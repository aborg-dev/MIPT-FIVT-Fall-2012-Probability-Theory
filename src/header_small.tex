% header.tex - default header for tex documents!

\documentclass[12pt]{article}
%\documentclass[11pt,a4paper,oneside]{book}
\frenchspacing
\pagestyle{plain}

\usepackage[papersize={4in,5in},margin=0.5cm]{geometry}
\usepackage{amssymb}
\usepackage{amsmath}
\usepackage[utf8]{inputenc}
\usepackage[russian]{babel}
\usepackage{cmap}
\usepackage{amsthm}
\usepackage[all]{xy}
\usepackage{mdwlist}
\usepackage{enumitem}
\usepackage{tikz}
\usepackage{epstopdf}

\hfuzz=18pt

\newcommand*\circled[1]{\tikz[baseline=(char.base)]{
            \node[shape=circle,draw,inner sep=2pt] (char) {#1};}}

\newcommand{\dd}{\; \mathrm{d}}
\newcommand{\setR}{\mathbb{R}}
\newcommand{\setRn}{\mathbb{R}^n}
\newcommand{\setRinf}{\mathbb{R}^{\infty}}
\newcommand{\setC}{\mathbb{C}}
\newcommand{\setN}{\mathbb{N}}
\newcommand{\setZ}{\mathbb{Z}}
\newcommand{\setQ}{\mathbb{Q}}
\newcommand{\setM}{\mathcal{M}}
\newcommand{\setL}{\mathcal{L}}
\newcommand{\setA}{\mathcal{A}}
\newcommand{\setF}{\mathcal{F}}

\newcommand{\walls}[1]{\left | #1 \right |} % |smth_vertically_large|
\newcommand{\pars}[1]{\left( #1 \right)} % (smth_vertically_large)
\newcommand{\class}[1]{[ #1 ]} % [smth_vertically_large]
\newcommand{\braces}[1]{\left\{ #1 \right\}} % {smth_vertically_large}

\newcommand{\condset}[2]{\braces{\, #1 \mid #2 \,}} % definition of set with condition

\newcommand{\expl}[1]{\walls{\text{#1}}} % explanation inside formula

%\newcommand{\argmin}{\operatornamewithlimits{arg\,min}}
\newcommand{\combus}[2]{\binom {#1}{#2}}
\newcommand{\combru}[2]{C_{#1}^{#2}}
\newcommand{\comb}[2]{\combru{#1}{#2}}
\newcommand{\limup}{\varlimsup\limits} % limit superior 
\newcommand{\limdown}{\varliminf\limits} % limit inferior
\newcommand{\toup}[1]{\xrightarrow{#1}}
\newcommand{\toae}{\toup{\text{\,п.н.}}} % almost everywhere convergence designation
\newcommand{\todown}[1]{\xrightarrow[#1]{}}

\newcommand{\equp}[1]{\stackrel{#1}{=}}

\newcommand{\mysum}{\sum\limits}

\newcommand{\conj}[1]{\overline{#1}} % complex conjugation
\newcommand{\comp}[1]{\overline{#1}} % set complement

\DeclareMathOperator{\cov}{cov}

\renewcommand{\emptyset}{\varnothing}
\renewcommand{\epsilon}{\varepsilon}
\renewcommand{\phi}{\varphi}
\renewcommand{\leq}{\leqslant}
\renewcommand{\geq}{\geqslant}
\renewcommand{\Im}{\mathop{\mathrm{Im}}\nolimits}
\renewcommand{\Re}{\mathop{\mathrm{Re}}\nolimits}

\DeclareMathOperator*{\argmin}{arg\,min}

\newcommand*{\hm}[1]{#1\nobreak\discretionary{}%
{\hbox{$\mathsurround=0pt #1$}}{}} % a\hm=b makes "=" carriable to the next line with duplication of the sign

\newtheorem{theorem}{Теорема}
\newtheorem*{theorem-star}{Теорема}

\newtheorem{statement}{Утверждение}
\newtheorem*{statement-star}{Утверждение}

\newtheorem{corollary}{Следствие}
\newtheorem*{corollary-star}{Следствие}

\newtheorem{lemma}{Лемма}
\newtheorem*{lemma-star}{Лемма}

\newtheorem{designation}{Обозначение}
\newtheorem*{designation-star}{Обозначение}

\newtheorem*{designations}{Обозначения}

\newtheorem{proposition}{Предложение}
\newtheorem*{proposition-star}{Предложение}

\newtheorem*{property-star}{Свойство}
\newtheorem{property}{Свойство}

\theoremstyle{remark}
\newtheorem*{remark}{Замечание}

\theoremstyle{definition}
\newtheorem{problem}{Задача}
\newtheorem{exercise}{Упражнение}

\theoremstyle{definition}
\newtheorem{definition}{Определение}
\newtheorem*{definition-star}{Определение}

\theoremstyle{definition}
\newtheorem{example}{Пример}
\newtheorem*{example-star}{Пример}

\theoremstyle{definition}
\newtheorem*{solution}{Решение}

%Style

\newcommand{\bigtitle}[1]{\title{\textbf{\underline{#1}}}}
\newcommand{\boldtitle}[1]{\title{\textbf{#1}}}

\newcommand{\mysection}[1]{\setcounter{secnumdepth}{-1} \section{#1} \setcounter{secnumdepth}{2} \setcounter{subsection}{0} \setcounter{corollary}{0} \setcounter{definition}{0}}


\renewcommand{\thesubsection}{\arabic{subsection}}
\newcommand{\tocstyle}{\setlength{\parindent}{0ex} \setlength{\parskip}{0ex}}
\newcommand{\mainstyle}{\setlength{\parindent}{0ex} \setlength{\parskip}{1ex}}

\mainstyle
\setcounter{secnumdepth}{2}
