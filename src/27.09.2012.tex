
\begin{lemma}[о $\pi$- и $\lambda$- системах]
	Система $\setF$ подмножеств $\Omega$ является $\sigma$-алгеброй \\
	$\Leftrightarrow$ она является $\pi$-системой и $\lambda$-системой одновременно.
	
	\begin{proof}~

		($\Rightarrow$) очевидно.

		($\Leftarrow$) Для $\forall A \in \setF$

		$\comp{A} = \Omega \setminus A \in \setF$ т.к. $\setF - \lambda$-система ($A \subset \Omega$ и $\Omega \in \setF$, свойство 2)
		
		Также имеется замкнутость относительно $\cap$ в $\setF$ 
		($\setF - \pi$-система) $\Rightarrow \setF$ является алгеброй

		Покажем, что она $\sigma$-алгебра:\\
		Пусть $\braces{ B_n, n \in \setN }$ - последовательность элементов из $\setF$, 
		Проверим, что $\bigcup\limits_{n} B_n \in \setF$\\
		
		Положим $A_m = \bigcup\limits_{n = 1}^{m} B_n$. \\
		Тогда $A_m \in \setF$ т.к $\setF$ -- алгебра. Кроме того $A_m \subset A_{m + 1}$ и
		$A_m \uparrow \bigcup\limits_n B_n = B$ \\
		
		Тогда в силу свойства 3) $\lambda$-системы, $B \in \setF$. Значит $F$ -- $\sigma$-алгебра
		
	\end{proof}
\end{lemma}

\begin{example}
	$\Omega = \braces{ 1, 2, 3, 4 }$\\
	$\setL = \braces{ \emptyset;\, (1, 2);\, (1, 3);\, 
	(1, 4);\, (2, 3);\, (2, 4);\, (3, 4);\, \Omega }$\\
	Тогда $\setL$ -- это $\lambda$-система, но не алгебра.
\end{example}

\begin{lemma}[о существовании минимальной системы]~

	Пусть $\setM$ -- система подмножеств $\Omega$.\\
	Тогда существует минимальная(по включению) алгебра 
	(или $\sigma$-алгебра, $\pi$-система, $\lambda$-система) содержащая $\setM$ и обозначаемая
	$\setL(\setM)\; (\sigma(\setM), \pi(\setM), \lambda(\setM))$
\end{lemma}

\begin{proof}
	Рассмотрим $\setF^* = 2^{\Omega}$ -- дискретная $\sigma$-алгебра. 
	Она является алгеброй($\sigma$-алгеброй, $\pi$-системой, $\lambda$-системой), 
	содержащей $\setM$, т.е множество интересующих нас систем не пусто.
	
	Рассмотрим $\alpha(\setM)\; (\sigma(\setM), \pi(\setM), \lambda(\setM))$ 
	-- пересечение всех алгебр ($\sigma$-алгебр, $\pi$-систем, $\lambda$-систем), содержащих $\setM$. 
	Тогда $\alpha(\setM)\; (\sigma(\setM), \pi(\setM), \lambda(\setM))$ тоже будет являться алгеброй 
	($\sigma$-алгеброй, $\pi$-системой, $\lambda$-системой), содержащей $\setM$.

	При этом она будет минимальной по включению.
\end{proof}

\begin{example}~

	\begin{enumerate}
		\item 
			Пусть $\setM = \condset{(a, b)}{a < b \in \setR}$ -- система интервалов.\\
			Тогда минимальная $\sigma$-алгебра, содержащая $\setM$, 
			называется борелевской $\sigma$-алгеброй на прямой и обозначается $B(\setR)$\\
			\begin{equation*}
				B(\setR) = \sigma(\setM)
			\end{equation*}
				
		\item 
			Рассмотрим в $\setR^n$ систему подмножеств вида
			\begin{align*}
				&\setM = \condset{B_1 \times \ldots \times B_n}{B_i \in B(\setR)}\\
				&\setM = \condset{(x_1, \ldots, x_n) \in \setR^n}{x_i \in B_i \quad \forall i = 1 \ldots n}
			\end{align*}
			Тогда минимальная $\sigma$-алгебра, содержащая $\setM$ называется 
			\emph{болевеской $\sigma$-алгеброй} в $\setR^n$  и обозначается $B(\setR^n)$
			 
		\item 
			$\setR^{\infty} = \condset{ (x_1, x_2, \ldots )}{x_n \in \setR \quad \forall n }$ 
			-- числовые последовательности.\\
			Для $\forall n\; \forall B_n \in B(\setR^n)$ введем
			\begin{equation*}
				\setM_n(B_n) = \condset{\vec{x} \in \setR^{\infty}, \vec{x} = (x_1, x_2, \ldots )}
														{(x_1, \ldots, x_n) \in B_n}
			\end{equation*} -- цилиндр с основанием $B_n$
		
		Минимальная $\sigma$-алгебра, содержащая все цилиндры, называется борелевской 
		$\sigma$-алгеброй в $\setR^{\infty}$ и обозначается $B(\setR^{\infty})$. Формально:
		\begin{equation*}
			B(\setR^{\infty}) = \sigma(\condset{\setM_n(B_n)}{n \in \setN, B_n \in B(\setR^n)})
		\end{equation*}
	\end{enumerate}
\end{example}

\begin{theorem}[о монотонных классах]~

	Пусть $\setM$ -- $\pi$-система на $\Omega$. Тогда $\lambda(\setM) = \sigma(\setM)$.

\end{theorem}

\begin{proof}
	Заметим, что $\sigma(\setM)$ -- $\sigma$-алгебра, содержащая $\setM \Rightarrow \sigma(\setM)$ -- 
	$\lambda$-система, содержащая $\setM \Rightarrow \lambda(\setM) \subset \sigma(\setM)$ 
	в силу минимальности.
	
	Согласно лемме о $\pi$- и $\lambda$-системах для того, чтобы доказать 
	$\sigma(\setM) \subset \lambda(\setM)$, достаточно проверить, что $\lambda(\setM)$ 
	является $\pi$-системой.
	
	Действительно, тогда $\lambda(\setM)$  будем $\sigma$-алгеброй, 
	содержащей $\setM \Rightarrow \sigma(\setM) \subset \lambda(\setM)$
	
	Рассмотрим следующую систему подмножеств:
	\begin{equation*}
		\setM_1 = \condset{ B \in \lambda(\setM)}
			{\forall A \in \setM \hookrightarrow A \cap B \in \lambda(\setM)}
	\end{equation*}
	Покажем, что $\setM_1$, является $\lambda$-системой,
	
	\begin{enumerate}
		\item 
			$\Omega \in \setM_1 ?$ Для $\forall A \in \setM$\\
			$\Omega \cap A = A \in \setM \subset \lambda(\setM) \Rightarrow \Omega \in \setM$.
		
		\item 
			Пусть $A, B \in \setM_1$ и $A \subset B$. 
			Верно ли, что $B \setminus A \in \setM_1 ?$

			Пусть $C \in \setM$. Тогда $(B \setminus A) \cap C = (B \cap C) \setminus (A \cap C)$

			Причем $A \setminus C \subset B \setminus C \Rightarrow$ по свойству 2) $\lambda$-системы\\
			получаем, что $(B \setminus A) \cap C \in \lambda(M) \Rightarrow B \setminus A \in \setM_1$
			
		\item 
			Пусть $\braces{B_n, n \in \setN}$ -- последовательность из $\setM_1$, 
			причем $B_n \uparrow B$. Верно ли, что $B \in \setM_1 ?$

			Для $\forall A \in \setM \quad (B_n \cap A) \uparrow (B \cap A)$. 
			Но $(B_n \cap A) \in \lambda(\setM) \Rightarrow (B \cap A) \in \lambda(\setM)$ 
			по свойству 3) $\lambda$-системы. $\Rightarrow B \in \setM_1$.

			Мы показали, что $\setM_1$ -- $\lambda$-система.

			В силу того, что $\setM$ -- $\pi$-система, $\setM \subset \setM_1 
			\Rightarrow \lambda(\setM) \subset \setM_1.$ 
			В силу минимальности. Но $\setM_1 \subset \lambda(\setM)$ по построению. \\
			Следовательно, $\lambda(\setM) = \setM_1$, т.е. 
			$\forall A \in \lambda(\setM) \quad \forall B \in \setM \quad A \cap B \in \lambda(\setM)$\\
			
			Рассмотрим систему
			\begin{equation*}
				\setM_2 = \condset{B \in \lambda(\setM)}
					{\forall A \in \lambda(\setM) \; A \cap B \in \lambda(\setM))}
			\end{equation*}

			Точно также проверяется, что $\setM_2$ -- это $\lambda$-система.
			Тогда $\sigma(\setM) \subset \setL$.\\
			В силу минимальности $\lambda(\setM) \subset \setL$. 
			По теореме $\sigma(\setM) = \lambda(\setM) \subset \setL$
	\end{enumerate}
\end{proof}

\mysection{Независимость событий}

\begin{definition}
	События $A$ и $B$ на вероятностном пространстве $(\Omega, \setF, P)$ 
	называются \emph{независимыми}, если
	\begin{equation*}
		P(A \cap B) = P(A) \, P(B)
	\end{equation*}
\end{definition}

\begin{exercise}
	Пусть $A$ и $B$ независимы. Тогда независимыми будут и такие пары:\\
	$\comp{A}, B \quad A, \comp{B} \quad \comp{A}, \comp{B}$
\end{exercise}
 
\begin{definition}
	Набор событий $A_1 \ldots A_n$ называются попарно независимыми, если $\forall i \neq j$ \\
	$A_i$ независимо с $A_j$.
\end{definition}

\begin{definition}
	События $A_1 \ldots A_n$ называются независимыми в совокупности, если $\forall k \leq n, 
	\forall i_1, \ldots i_k : 1 \leq i_1 < \ldots < i_k \leq n$ выполнено:
	\begin{equation*}
		P(A_{i_1} \cap \ldots \cap A_{i_k}) = P(A_{i_1}) \ldots P(A_{i_k})
	\end{equation*}
\end{definition}

\begin{definition}
	Системы событий $\setM_1 \ldots \setM_n, \; \setM_i \subset \setF$ 
	называются \emph{независимыми в совокупности}, 
	если $\forall A_1 \in \setM_1, \ldots, A_n \in \setM_n$ 
	события $A_1 \ldots A_n$ -- независимы в совокупности.
\end{definition}

\begin{lemma}[критерий независимости $\sigma$-алгебр]~

	Пусть $\setM_1$ и $\setM_2$ -- $\pi$-системы в $\setF$.
	Тогда $\sigma(\setM_1)$ и $\sigma(\setM_2)$ независимы 
	$\Leftrightarrow \setM_1$ и $\setM_2$ -- независимы.

\end{lemma}

\begin{proof}~

	$(\Rightarrow)$ очевидно из определения

	$(\Leftarrow)$ используем принцип подходящих множеств.

	Рассмотрим такую систему:
	\begin{equation*}
		\setL_1 = \condset{ A \in \sigma(\setM_2)}{A \text{ независимо с } \setM_1}
	\end{equation*}
	
	Проверим, что $\setL_1$ -- это $\lambda$-система.
	\begin{enumerate}
		\item 
			$\Omega \in \setL_1?$

			$P(\Omega \cap A) = P(A) = P(A) \, P(\Omega) 
			\Rightarrow $ независимы $\Rightarrow \Omega \in \setL_1$ 

		\item 
			Пусть $A, B \in \setL_1$, причем $A \subset B$. \quad $B \setminus A \in \setL_1 ?$\\
			Пусть $C \in \setM_1$. Тогда
			\begin{align*}
				&P(B \setminus A \cap C) = P((B \cap C) \setminus (A \cap C)) =
				P(B \cap C) - P(A \cap C) = \\
				&P(B) \, P(C) - P(A) \, P(C) = (P(B) - P(A)) P(C) = P(B \setminus A) \, P(C) \\
				&\Rightarrow B \setminus A \text{ независимо с } C 
				\Rightarrow \text{ независимо с } \setM_1 \Rightarrow B \setminus A \in \setL_1
			\end{align*}

		\item
			Пусть $B_n \uparrow B, B_n \in \setL_1$. Верно ли, что $B \in \setL_1$?

			Да:\\
			Пусть $A \in \setM_1$. Тогда $(B_n \cap A) \uparrow (B \cap A)$.
			\begin{align*}
				&P(B \cap A) = \expl{по теореме о непрерывности меры} = \lim_n P(B_n \cap A) =\\
				&= \expl{$B_n \in \setL_1 \Rightarrow B$ независимо с $A$} = \lim_n P(B_n) P(A) =
				\expl{по теореме о непрерывности} = P(B) P(A)\\
				&\Rightarrow B \text{ и } A \text{ непрерывны } \Rightarrow B \in \setL_1
			\end{align*}

			Значит $\setL_1$ -- $\lambda$-система. По условию мы знаем, что $\setM_2$ независима
			с $\setM_1 \Rightarrow \setM_2 \subset \setL_1 \Rightarrow$ по следствию из теоремы 
			о монотонности $\Rightarrow \sigma(\setM_2) \subset \setL_1
			\Leftrightarrow$ т.е. $\sigma(\setM_2)$ независимо с $\setM_1$ \\

			Рассмотрим по аналогии 
			\begin{align*}
				\setL_2 = \condset{A \in \setF}{A \text{ независимо с } \sigma(\setM_2)}
			\end{align*}

			Аналогично $\Rightarrow \setL_2$ -- $\lambda$-система.

			По теореме о монотонных классах, в силу того, что $\setM_1 \subset \setL_2$,
			получаем, что $\sigma(\setM_1) \subset \setL_2 
			\Rightarrow \sigma(\setM_1)$ независимо с $\sigma(\setM_2)$.
		
	\end{enumerate}
\end{proof}

\begin{corollary}
	Пусть $\setM_1 \ldots \setM_n$ -- $\pi$-системы в $\setF$. Тогда $\setM_1, \ldots, \setM_n$
	независимы в совокупности $\Leftrightarrow \sigma(\setM_1), \ldots \sigma(\setM_n)$ 
	независимы в совокупности.
\end{corollary}

\begin{definition}
	Пусть $\braces{\setM_{\alpha}}_{\alpha \in \mathfrak{A}}$ -- произвольный набор систем событий из 
	$\setF$. Тогда этот набор называется независимым в совокупности, 
	если $\forall n \, \forall \alpha_1 \ldots \alpha_n \in \mathfrak{A},\; \alpha_i \neq \alpha_j,$ 
	системы $\setM_{\alpha_1} \ldots \setM_{\alpha_n}$ независимыми в совокупности, 
	тоесть любой конечный поднабор независим.
\end{definition}

