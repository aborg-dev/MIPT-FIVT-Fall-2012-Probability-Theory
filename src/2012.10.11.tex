
\begin{enumerate}[label=\protect\circled{\arabic*},series=distributions,start=2]
	\item \bigtitle{Абсолютно непрерывные распределения}
\end{enumerate}

\begin{definition}
	Пусть $F(x)$ -- функция распределения вероятностой меры $P$ на $\setR$, 
	причем для $\forall x \in \setR$
	имеет место равенство
	\begin{equation*}
		F(x) = \int\limits_{-\infty}^{x} p(t)\, dt
	\end{equation*}
	где $p(t) \geq 0$ -- неотрицательная функция т.ч
	\begin{equation*}
		 \int\limits_{-\infty}^{+\infty} p(t)\, dt = 1
	\end{equation*}
	
	В этом случае вероятностная мера $P$ называется \emph{абсолютно непрерывной}, 
	а $F(x)$ - \emph{абсолютно непрерывной функцией распределения}.
	Функция $p(t)$ называется \emph{плотностью распределения} $P$ (или просто плотностью)
\end{definition}

\begin{example}~

	\begin{enumerate}
		\item 
			Равномерное распределение на отрезке $[a, b]$.
			\begin{equation*}
				p(x) =
				\begin{cases}
					\dfrac{1}{b - a},&x \in [a, b]\\
					0				,&\text{иначе}
				\end{cases}		
			\end{equation*}
			
			\begin{equation*}
				F(x) =
				\begin{cases}
					0, &x < a\\
					\dfrac{x - a}{b - a}, &x \in [a, b]\\
					1, &x \geq b
				\end{cases}
			\end{equation*}
		
		\item 
			Нормальное распределение (с параметрами ($a, \sigma^2$))
			\begin{equation*}
				p(x) = \frac{1}{\sqrt{2\pi \sigma^2}} e^{-\frac{(x - a)^2}{2 \sigma^2}},\;
				a \in \setR, \sigma > 0
			\end{equation*}
			Моделирование: измерения величины $a$ = $a$ + ошибка измерения.
		
		\item 
			Гамма распределение (с параметрами ($d, \lambda$))
			\begin{equation*}			
				p(x) = 			
				\begin{cases}
					\dfrac{\alpha^\lambda x^{\lambda - 1}}{\Gamma(x)} e^{-\alpha x}, 
					&x > 0, \quad \alpha, \lambda > 0\\
					0, &\text{иначе}
				\end{cases}
			\end{equation*}
			\begin{definition}
				\begin{align*}				
					&\Gamma(\lambda) = \int\limits_{0}^{+\infty} x^{\lambda - 1} e^{-x} dx 
						\quad \text{для $\lambda > 0$} \\
					&\Gamma(n) = (n - 1)!\\
					&\Gamma(\lambda + 1) = \lambda \Gamma(\lambda)\\
					&\Gamma \left( \frac{1}{2} \right) = \sqrt{\pi}
				\end{align*}
			\end{definition}
			
		\item Экспоненциальное распределение(или показательное) (с параметром $\lambda > 0$).
			\begin{align*}
				p(x) = 
				\begin{cases}
					\lambda e^{-\lambda x}, &x > 0\\
					0, &\text{иначе}
				\end{cases}
			\end{align*}

			\begin{align*}
				F(x) =
				\begin{cases}
					1 - e^{-\lambda x}, &x > 0\\
					0, \text{иначе}
				\end{cases}
			\end{align*}
			Моделирование: время ожидания(время работы приборов)
		
		\item Распределение Коши (с параметром $\Theta > 0$)
		
		\begin{align*}
			&p(x) = \frac{\Theta}{\pi (\Theta^2 + x^2)}\\
			&F(x) = \frac{1}{\pi} \arctan \left( \frac{x}{\Theta} \right) + \frac{1}{2}\\
		\end{align*}
	\end{enumerate}
\end{example}

\begin{enumerate}[resume*=distributions]
	\item \bigtitle{Сингулярные распределения}
\end{enumerate}

\begin{definition}
	Пусть $F(x)$ -- функция распределения на $\setR$.\\ 
	Точка $x_0 \in \setR$ называется \emph{точкой роста} для $F(x)$, если для $\forall \epsilon > 0$
	\begin{equation*}
		F(x_0 + \epsilon) - F(x_0 - \epsilon) > 0
	\end{equation*}
\end{definition}

\begin{definition}
	Множество $A \subset \setR$ называется множеством лебеговой меры нуль, если для $\forall \epsilon > 0 \quad \exists$ счетный набор интервалов $((a_k, b_k), k \in \setN)$ т.ч 
	\begin{align*}
		&\sum_k (b_k - a_k) \leq \epsilon\\
		&A \subset \bigcup_{k = 1}^{\infty} (a_k, b_k)
	\end{align*}
\end{definition}

\begin{example}
	$\forall$ счетное множество $\mathcal{X}$ имеет меру нуль.\\
	Пусть 
		\begin{align*}			
			&\mathcal{X} = \{ x_1, x_2, \ldots \}\\
			&(a_k, b_k) = \pars{x_k - \frac{\epsilon}{2^{k + 1}}, x_k + \frac{\epsilon}{2^{k + 1}}}\\
			&\sum_{k = 1}^{\infty} (b_k - a_k) = \sum_{k = 1}^{\infty} \frac{\epsilon}{2^k} = \epsilon
		\end{align*}	
\end{example}

\begin{definition}
	Функция распределения $F(x)$ называется \emph{сингулярной}, 
	если она непрерывна и её множество точек роста имеет лебегову меру нуль.
\end{definition}

\begin{theorem}[Лебег]
	Пусть $F(x)$ -- произвольная функция распределения. Тогда существует разложение вида
	\begin{equation*}
		F(x) = \alpha_1 F_1(x) + \alpha_2 F_2(x) + \alpha_3 F_3(x)
	\end{equation*}	 
	где 
	\begin{align*}
		&F_1 \text{ -- дискретная функция рапределения}\\
		&F_2 \text{ -- абсолютно непрерывная функция рапределения}\\
		&F_3 \text{ -- сингулярная функция рапределения}
	\end{align*}
	$\alpha_1, \alpha_2, \alpha_3 \geq 0,\quad \alpha_1 + \alpha_2 + \alpha_3 = 1$
\end{theorem}

\mysection{Вероятностные меры в $\setR^n$}

\begin{definition}
	Пусть $P$ -- вероятносная мера на $(\setR^n , B(\setR^n))$

	Тогда функция $F(\vec{x}), \vec{x} = (x_1, \ldots, x_n)$
	\begin{equation*}
		F(\vec{x}) = P((-\infty, x_1] \times \ldots \times (-\infty, x_n])
	\end{equation*}
	называется функцией распределения вероятностой меры $P$ в $\setR^n$.
\end{definition}

\begin{designations}
	Пусть $\vec{x}^{(k)} = (x_1^{(k)}, \ldots, x_n^{(k)}) \in \setR^n$

	Будем писать $\vec{x}^{(k)} \downarrow \vec{y} = (y_1, \ldots, y_n)$, если:\\
	$\forall i \quad x_i^{(k)} \geq x_i^{(k + 1)} \geq y_i$ и $x_i^{(k)} \to y_i$ при $k \to \infty$
\end{designations}

\begin{lemma}[свойства многомерной функции распределения]~

	Пусть $F(\vec{x})$ -- функция распределения вероятностной меры $P$ в $\setR^n$ Тогда:
	\begin{enumerate}
		\item	Если $\vec{x}^{(k)} \downarrow \vec{x}$, то $F(\vec{x}^{(k)}) \to F(\vec{x})$
		
		\item 
			$\lim\limits_{\forall i: \, x_i \rightarrow +\infty} F(\vec{x}) = 1$ и 
			$\forall i \lim\limits_{x_i \rightarrow -\infty}	F(\vec{x}) = 0$
		
		\item	
			Для $\forall i = 1 \ldots n\quad \forall a_i < b_i \in \setR$ введем оператор 
			\begin{equation*}
				\Delta_{a_i, b_i}^{i} F(\vec{x}) = 
					F(x_1, \ldots b_i, \ldots x_n) - F(x_1, \ldots a_i, \ldots x_n)
			\end{equation*}
		
	\end{enumerate}
	
	Тогда $\forall a_1 < b_1, \ldots, a_n < b_n$:
	\begin{equation*}
		\Delta_{a_1, b_1}^{1} \ldots \Delta_{a_n, b_n}^{n} F(\vec{x}) \geq 0
	\end{equation*}
\end{lemma}

\begin{proof}~
	\begin{enumerate}
		\item  
			Если $\vec{x}^{(k)} \downarrow \vec{x}$, то множество \\
			$(-\infty, x_1^{(k)}] \times \ldots \times (-\infty, x_n^{(k)}] \downarrow  
			(-\infty, x_1] \times \ldots \times	(-\infty, x_n]$\\
			 $\Rightarrow \expl{по непрерывности вероятностной меры} \Rightarrow$\\
			$F(\vec{x}^{(k)}) = P((-\infty, x_1^{(k)}] \times \ldots \times (-\infty, x_n^{(k)}]) 
			\todown{k \to \infty} P((-\infty, x_1] \times \ldots \times (-\infty, x_n]) = F(\vec{x})$
				
		\item 
			Если $x_1 \ldots x_n \rightarrow +\infty$, то 
			$(-\infty, x_1] \times \ldots \times (-\infty, x_n] \uparrow \setR^n$\\
			В силу непрерывности вероятностной меры:
			\begin{equation*}
				\lim\limits_{\forall i:\, x_i \to \infty} F(\vec{x}) = P(\setR^n) = 1
			\end{equation*}

			Если же $\vec{x}^{(k)} \rightarrow -\infty,\; k \to \infty$, 
			то $(-\infty, x_1] \times \ldots \times (-\infty, x_i^{(k)}] 
			\times \ldots \times (-\infty, x_n] \downarrow \emptyset$

			Отсюда в силу непрерывности вероятностной меры:
			\begin{equation*}
				\lim\limits_{x_i \to -\infty} F(\vec{x}) = P(\emptyset) = 0
			\end{equation*}
		
		\item Докажем, только для $n = 2$
			\begin{align*}
				&\Delta_{a_1 b_1}^1 \Delta_{a_2 b_2}^2 F(\vec{x}) = 
				\Delta_{a_1 b_1}^1 (F(x_1, b_2)) - F(x_1, a_2)) = 
				F(b_1, b_2) - F(b_1, a_2) - F(a_1, b_2) + F(a_1, a_2) =\\
				&= P((-\infty, b_1] \times (-\infty, b_2]) - P((-\infty, b_1] \times (-\infty, a_2]) -
				P((-\infty, a_1] \times (-\infty, b_2]) + \\
				&+ P((-\infty, a_1] \times (-\infty, a_2]) =
				P((a_1, b_1] \times (a_2, b_2]) - P((-\infty, a_1] \times (-\infty, a_2]) +\\
				&+ P((-\infty, a_1] \times (-\infty, a_2]) = P((a_1, b_1] \times (a_2, b_2]) \geq 0
			\end{align*}						
			
	\end{enumerate}
\end{proof}

\begin{theorem}[о взаимно однозначном соответствии]~

	Если $F(\vec{x}),\; \vec{x} \in \setR^n$, удовлетворяет свойствам 1) - 3) из леммы, то $\exists !$ вероятностная мера $P$ в $(\setR^n, B(\setR^n))$, для которой $F(\vec{x})$ является функцией распределения т.е. 
	\begin{align*}
		&\forall a_1 < b_1, \ldots, a_n < b_n\\
		&\Delta_{a_1 b_1}^1 \ldots \Delta_{a_n b_n}^n F(\vec{x}) = P((a_1, b_1] \times \ldots \times (a_n, b_n])
	\end{align*}
\end{theorem}
