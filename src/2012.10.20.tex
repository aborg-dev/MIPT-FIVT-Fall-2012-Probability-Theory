  
\mysection{Случайные элементы}
                  
\begin{definition}
	Пусть ($\Omega, \setF$) и $(E, \mathcal{E})$ -- два измеримых пространства. 
  Отображение $X\colon \Omega \to E$ называется случайным элементом,
   если оно является $\setF$ - измеримым. (или $\setF \setminus \mathcal{E}$ - измеримым) 
   т.е $\forall B \in \mathcal{E}$
  \begin{align*}
    \{ x \in B \} = X^{-1}(B) = \condset{\omega}{X(\omega) \in B} \in \setF.
  \end{align*}
\end{definition}

\begin{definition}~

  Если $(E, \mathcal{E})$ = $(\setR, B(\setR))$, то случайный элемент $X$ 
  называется \emph{случайной величиной}.\\

  Если $(E, \mathcal{E})$ = $(\setRn, B(\setRn))$, то $X$ называется \emph{случайным вектором}.\\
\end{definition}

\begin{lemma}[достаточное условие измеримости отображения]~

  Пусть $(\Omega, \setF)$ и $(E, \mathcal{E})$ -- два измеримых пространства, 
  $X\colon \Omega \to E$.
  Пусть $\setM \subset \mathcal{E}$ таково, что $\sigma(\setM) = \mathcal{E}$. 
  Тогда $X$ является случайным элементом $\iff$ для $\forall B \in \setM$

  \begin{align*}
    X^{-1}(B) = \condset{\omega}{X(\omega) \in B} \in \setF
  \end{align*}

\end{lemma}

\begin{proof}~

  $(\implies)$ очевидно из определения\\

  $(\impliedby)$ \\
  Рассмотрим систему множеств
  \begin{align*}
    D = \condset{B \in \mathcal{E}}{X^{-1}(B) \in \setF}
  \end{align*}
  Убедимся в том, что $D$ -- это $\sigma$-алгебра. 
  Операция прообраз сохраняет все теоретико-множественные операции.

  \begin{align*}
    &X^{-1}\pars{\bigcup_{\alpha} D_{\alpha}} = \bigcup_{\alpha} X^{-1} (D_{\alpha})\\\\
    &X^{-1}\pars{\bigcap_{\alpha} D_{\alpha}} = \bigcap_{\alpha} X^{-1} (D_{\alpha})\\\\
    &X^{-1}\pars{B \setminus A} = X^{-1} (B) \setminus X^{-1} (A)
  \end{align*}

  Тогда
  \begin{enumerate}
    \item 
      $X^{-1}(E) = \Omega \in \setF \implies E \in D$

    \item  
      $X^{-1}\pars{\bigcup\limits_{n = 1}^{\infty} D_n} = \{ D_n \in D \} = 
      \bigcup\limits_{n = 1}^{\infty} X^{-1}(D_n) \in \setF 
      \implies \bigcup\limits_{n = 1}^{\infty} D_n \in D$

    \item
      Если $B, A \in D$, то
      $X^{-1}(B \setminus A) = X^{-1}(B) \setminus X^{-1}(A) \in \setF 
      \implies B \setminus A \in D$
  \end{enumerate}

  $D$ -- $\sigma$-алгебра и по условию $\setM \subset D$
  $\implies$ в силу минимальности $\sigma(\setM) = \mathcal{E} \subset D$ (А значит $E = D$)\\
  т.е $\forall  B \in \mathcal{E} : X^{-1}(B) \in \setF$
  и, стало быть, $X$ -- случайный элемент.
\end{proof}

\begin{corollary}~

  \begin{enumerate}
    \item 
      $X$ -- случайная величина на $(\Omega, \setF)$
      $\iff \forall x \in \setR :
       \{ X \leq x \} = 
      \condset{\omega}{X(\omega) \leq x} \in \setF$

    \item $X = (X_1, \ldots, X_n)$ -- случайный вектор на $(\Omega, \setF)$
          $\iff \forall i : X_i$ -- случайная величина.
  \end{enumerate}

\end{corollary}

\begin{proof}~

  $(\implies)$ 1) и 2) очевидно из определения случайных величин и векторов\\

  $(\impliedby)$ \nolinebreak
  \begin{enumerate}
    \item 
      Рассмотрим систему $\setM = \condset{(-\infty; x]}{x \in \setR}$.\\
      Тогда  $\sigma(\setM) = B(\setR)$. По условию $X^{-1} (B) \in \setF$ для 
      $\forall B \in \setM$. По лемме о достаточном условии измеримости получим, 
      что $X$ -- случайная величина.

    \item 
      Рассмотрим систему $\setM = \condset{B_1 \times \ldots B_n}{B_i \in B(\setR)}$\\
      Тогда $\sigma(\setM) = B(\setRn)$
      \begin{align*}
        &X^{-1} (B_1 \times \ldots \times B_n) = 
        \condset{\omega}{X_1(\omega) \in B_1, \ldots, X_n(\omega) \in B_n}
        = \bigcap_{i = 1}^{n} X^{-1}_i (B_i) \in \setF
      \end{align*}

      $\implies X^{-1} (B) \in \setF$ для $\forall B \in \setM$. По лемме получаем, 
      что $X$ -- случайный вектор.
  \end{enumerate}

\end{proof}

\bigtitle{Смысл условия измеримости}

Случайные величины и векторы --- это численные и векторные характеристики случайных экспериментов. Нам нужно уметь вычилсять вероятности вида $P(\xi \leq x)$ или $P(\xi \in [a, b])$

Но $P$ задана формально только на $\sigma$-алгебре $\setF$\\
Значит, нам нужно требовать, чтобы события вида $\{ \xi \leq x \}$ и 
$\{\xi \in [a, b]\}$ лежали в $\setF$.

\mysection{Действия над случайными величинами и векторами}

\begin{definition}
  Функция $\phi\colon \setRn \to \setR^m$ называется борелевской, 
  если для $\forall B \in B(\setR^m)$
  \begin{align*}
    \phi^{-1} (B) = \condset{x}{\phi(x) \in B} \in B(\setRn)
  \end{align*}
\end{definition}

\begin{lemma}
  Пусть $\xi = (\xi_1, \ldots, \xi_n)$ -- случайный вектор.
  $\phi\colon \setRn \to \setR^m$ -- борелевская функция.\\
  Тогда $\phi(\xi)$ -- тоже случайный вектор.
\end{lemma}

\begin{proof}
  Пусть $B \in B(\setR^m)$. Тогда
  \begin{align*}
    &(\phi(\xi))^{-1} (B) = \condset{\omega}{\phi(\xi(\omega)) \in B}
    =\condset{\omega}{\xi(\omega) \in \phi^{-1} (B)} \in \setF 
    \quad\text{ (т.к $\phi^{-1}(B) \in B(\setRn)$) }
  \end{align*}
  $\implies \phi(\xi)$ -- случайный вектор.
\end{proof}

\begin{theorem}
  Любая непрерывная или кусочно-непрерывная функция является борелевской.
\end{theorem}

\begin{corollary}~

  Пусть $\xi$ и $\eta$ -- случайные величины, $c \in \setR$.

  Тогда $c\, \xi,\; \xi + c,\; \xi + \eta,\; \xi - \eta$ и $\dfrac{\xi}{\eta}$ 
  (считаем, что $\eta(\omega) \neq 0\; \forall \omega \in \Omega)$ -- тоже случайные величины.

\end{corollary}

\begin{proof}
  $\phi(x, y) = x y$ или $x + y$ -- непрерывные функции в $\setR^2 \implies$ борелевские.\\
  Константа $c$ -- случайная величина $\implies$ по лемме получаем, 
  что $c\, \xi,\; \xi + c,\; \xi + \eta,\; \xi - \eta$ --- случайные величины.\\

  Рассмотрим

  $\phi(x, y) = \begin{cases}
                      \dfrac{x}{y}, &y \neq 0\\
                      0, &y = 0
                    \end{cases}$

  Она тоже борелевская(кусочно-непрерывная) $\implies \phi(\xi, \eta) = \dfrac{\xi}{\eta}$ ---
  тоже случайная величина.
\end{proof}

\begin{lemma}[пределы случайной величины]~

  Пусть $\{ \xi_n, n \in \setN \}$ -- последовательность случайных величин.

  Тогда $\limup_n \xi_n,\; \limdown_n \xi_n,\; \sup\limits_n \xi_n,\; \inf\limits_n \xi_n$ -- 
  тоже случайная величина. \\
  (Они могут принимать значения $\pm\infty$)

\end{lemma}

\begin{proof}
  \begin{align*}
    &\condset{\omega}{\sup_n \xi_n (\omega) \leq x} = 
    \bigcap_{n = 1}^{\infty} \{ \xi_n \leq x \} \in \setF\\
    &\implies \sup_n \xi_n (\omega) \text{ -- случайная величина }\\
    &\condset{\omega}{\inf_n \xi_n (\omega) \geq x} = 
    \bigcap_{n = 1}^{\infty} \{ \xi_n \geq x \} \in \setF\\
    &\implies \inf_n \xi_n (\omega) \text{ -- случайная величина }\\
    &\condset{\omega}{\limup_n \xi_n (\omega) \leq x} = 
    \bigcap_{k = 1}^{\infty} \bigcup_{m = 1}^{\infty} 
    \bigcap_{n = m}^{\infty} \{ \xi_n \leq x + \frac{1}{k} \} \in \setF\\
    &\implies \limup_n \xi_n (\omega) \text{ -- случайная величина }\\
    &\condset{\omega}{\limdown_n \xi_n (\omega) \geq x} = 
    \bigcap_{k = 1}^{\infty} \bigcup_{m = 1}^{\infty} 
    \bigcap_{n = m}^{\infty} \{ \xi_n \geq x - \frac{1}{k} \} \in \setF\\
    &\implies \limdown_n \xi_n (\omega) \text{ -- случайная величина }\\
  \end{align*}
\end{proof}

\mysection{Характеристики случайных величин и векторов}

Распределение случайной величины вектора.

\begin{definition}
  Пусть $(\Omega, \setF, P)$ -- вероятностное пространство, $\xi$ - случайная величина на нем. Тогда распределением $\xi$ называется вероятностная мера $P_\xi$ на $(\setR, B(\setR))$, заданная по правилу.
  \begin{align*}
    P_\xi (B) = P(\xi \in B),\; B \subset B(\setR).
  \end{align*}
\end{definition}

\begin{definition}
  Пусть $\xi$ - случайный вектор размерности $n$ на $(\Omega, \setF, P)$. 
  Тогда его распределением $P_\xi$ называется вероятностая мера на ($\setRn, B(\setRn)$), заданная по правилу
    \begin{align*}
      P_\xi (B) = P(\xi \in B),\; B \in B(\setRn)
    \end{align*}
\end{definition}

\bigtitle{Функция распределения}

\begin{definition}
  Пусть $(\Omega, \setF, P)$ -- вероятностное пространство.
  $\xi$ - случайная велличина на нем. Тогда \emph{функцией распределения} случайной величины $\xi$ называется
    \begin{align*}
      F_\xi (x) = P(\xi \leq x)
    \end{align*}
\end{definition}

\begin{definition}
  Случайная величина $\xi$ называется
  \begin{itemize}
    \item 
      дискретной, если её функция распределения дискретная.

    \item 
      абсолютно непрерывной, если её функция распределения абсолютно непрерывна. 
      В этом случае
      \begin{align*}
        P(\xi \leq x) = F_\xi (x) = \int_{-\infty}^{x} p_\xi (t)\, dt
      \end{align*}
      и функция $p_\xi (t)$ называется плотностью случайной величины $\xi$.\\

    \item
      сингулярной, если её функция распределения сингулярна

    \item
      непрерывной, если её функция рапределение непрерывна.
  \end{itemize}
\end{definition}

\begin{definition}
  Пусть $\xi = (\xi_1, \ldots, \xi_n)$ -- случайный вектор на $(\Omega, \setF, P)$. 
  Тогда его \emph{функцией распределения} называется 
  \begin{align*}
    F_\xi (x_1, \ldots, x_n) = P(\xi_1 \leq x_1, \ldots, \xi_n \leq x_n).
  \end{align*}
\end{definition}

\bigtitle{Порожденная $\sigma$-алгебра}

\begin{definition}
  Пусть $\xi$ - случайная величина на $(\Omega, \setF, P)$. 
  Тогда \emph{$\sigma$-алгеброй $\setF_\xi$, порожденной $\xi$} называется
  \begin{align*}
    \setF_\xi = \condset{\{ \xi \in B \}}{B \in B(\setR)}
  \end{align*}
\end{definition}

\begin{definition}
  Если $\xi$ -- случайный вектор размерности $n$ на $(\Omega, \setF, P)$, 
  то $\sigma$-алгеброй, порожденной $\xi$ называется
  \begin{align*}
    \setF_\xi = \condset{\{ \xi \in B \}}{B \in B(\setRn)}
  \end{align*}
\end{definition}

Схема:
\begin{align*}
  (\Omega, \setF, P) &\toup{\xi} (\setR, B(\setR))\\
  P                  &\toup{\phantom{\xi}} P_\xi\\
  \setF_\xi              &\xleftarrow{\phantom{\xi}} B(\setR)
\end{align*}

\begin{definition}
  Пусть $\xi$ и $\eta$ -- случайные величины. 
  Будем говорить, что $\eta$ является \emph{$\setF_\xi$ - измеримой}, 
  если $\setF_\eta \subset \setF_\xi$.
\end{definition}

\begin{exercise}
  Пусть $\phi(x)$ -- борелевская функция, $\eta = \phi(\xi)$. Тогда $\eta$ -- $\setF_\xi$ - измерима.
\end{exercise}

\begin{theorem}
  Пусть $\eta$ - $F_\xi$ - измерима. 
  Тогда $\exists$ борелевская функция $\phi(x)$ т.ч\; $\eta = \phi(\xi)$
\end{theorem}

\begin{definition}~

  Пусть $A \in \setF$ -- событие на $(\Omega, \setF, P)$\\
  Тогда случайная величина\\
  \begin{align*}
    I_A = \begin{cases}
              1, & \omega \in A\\
              0, & \omega \not\in A
          \end{cases}
  \end{align*}

  называется индикатором события $A$
\end{definition}

\begin{definition}
  Случайная величина $\xi$ называется \emph{простой}, если она принимает конечное число значений. 

  Тогда $\exists$ набор $\{x_1, \ldots, x_n\}$ из различных чисел т.ч
  \begin{align*}
    \xi = \sum_{k = 1}^{n} x_k I_{A_k}
  \end{align*}
  где события $A_1 \ldots A_n$ -- разбиение $\Omega$. т.е $A_k = \{ \xi = x_k \}$
\end{definition}

