
\mysection{Неравенства}

\begin{enumerate}[label=\protect\circled{\arabic*},series=inequalities]

  \item
    \bigtitle{Неравенство Маркова}

    Пусть $\xi \geq 0$ -- неотрицательная случайная величина. 

    Тогда для $\forall \epsilon > 0:$ \quad
    $
      \boxed{
        P(\xi \geq \epsilon) \leq \frac{E \xi}{\epsilon}
      }
    $

    \begin{proof}
        $P(\xi \geq \epsilon) = E\, I \{ \xi \geq \epsilon \} 
        \leq E\pars{\dfrac{\xi}{\epsilon}\, I \{ \xi \geq \epsilon \}}
        \leq E\pars{\dfrac{\xi}{\epsilon}} = \dfrac{E \xi}{\epsilon}$
    \end{proof}

  \item
    \bigtitle{Неравенство Чебышева}

    Если $D\xi < +\infty$, то для $\forall \epsilon > 0:$
    $
      \boxed{
        P(|\xi - E \xi| \geq \epsilon) \leq \frac{D\xi}{\epsilon^2}
      }
    $

    \begin{proof}
      \begin{align*}
        P(|\xi - E \xi| \geq \epsilon) = P(|\xi - E\xi|^2 \geq \epsilon^2) 
        \leq \expl{нер-во Маркова} \leq \frac{E\walls{\xi - E\xi}^2}{\epsilon^2} = 
        \frac{D\xi}{\epsilon^2}
      \end{align*}
    \end{proof}

  \item
    \bigtitle{Неравенство Йенсена}

    Пусть $g(x)$ -- выпуклая вниз функция.
    Пусть $E\xi$ - конечно. Тогда
    \begin{align*}
      \boxed{E g(\xi) \geq g(E\xi)}
    \end{align*}

    \begin{proof}
      Т.к $g(x)$ -- выпуклая вниз функция, то $\forall x_0 \in \setR\; \exists \lambda(x_0):$ 
      т.ч. $\forall x \in \setR$ выполнено:
      \begin{align*}
        g(x) \geq g(x_0) + \lambda(x_0) (x - x_0)
      \end{align*}

      Положим $x = \xi,\; x_0 = E\xi$. Тогда
      \begin{align*}
        g(\xi) \geq g(E\xi) + \lambda(E\xi) (\xi - E\xi)
      \end{align*}

      Берем математическое ожидание от обеих частей:
      \begin{align*}
        E g(\xi) \geq g(E\xi) + \lambda(E\xi) E(\xi - E\xi) = g(E\xi)
      \end{align*}
    \end{proof}

\end{enumerate}

\mysection{Виды сходимостей случайных величин}

\begin{definition}~

  \begin{enumerate}
    \item 
      Последовательность случайных величин $\{ \xi_n,\; n \in \setN \}$ 
      \emph{сходится по вероятности} к случайной величине $\xi$ 
      (обозначение $\xi_n \toup{p} \xi$), если для $\forall \epsilon > 0:$
      \begin{align*}
        P\pars{|\xi_n - \xi| \geq \epsilon} \todown{n \to \infty} 0
      \end{align*}

    \item 
      Последовательность случайных величин $\{ \xi_n,\; n \in \setN \}$ 
      \emph{сходится с вероятностью 1} к случайной величине $\xi$ 
      (или сходится \emph{почти наверное}), если 
      \begin{align*}
        P(\omega : \lim_{n \to \infty} \xi_n (\omega) = \xi(\omega)) = 1
      \end{align*}

      Обозначения: $\xi_n \toae \xi,\; \xi_n \to \xi \text{ п.н.}$ или 
      $\xi_n \to \xi\; P\text{-п.н.}$

    \item 
      Последовательность случайных величин $\{ \xi_n,\; n \in \setN \}$ 
      \emph{сходится в среднем порядка $p > 0$} к случайной величине $\xi$
      (или \emph{сходится в пространстве $L^p$}), если
      \begin{align*}
        E|\xi_n - \xi|^p \todown{n \to \infty} 0
      \end{align*}
      
      Обозначение: $\xi_n \toup{L^p} \xi$

    \item 
      Последовательность случайных величин $\{ \xi_n, n \in \setN \}$ 
      \emph{сходится по распределению} к случайной величине $\xi$, 
      если для $\forall$ ограниченой непрерывной ф-ции $f(x)$ выполнено
      \begin{align*}
        E f(\xi_n) \todown{n \to \infty} E f(\xi)
      \end{align*}
      
      Обозначение: $\xi_n \toup{d} \xi$\\
  \end{enumerate}
\end{definition}

\begin{theorem}[Закон больших чисел в форме Чебышева]~

  Пусть $\{ \xi_n,\; n \in \setN \}$ -- последовательность попарно некоррелированных случайных величин, т.ч. $\forall n : D\xi_n \leq C$.

  Обозначим $S_n = \xi_1 + \ldots + \xi_n$. Тогда
  \begin{align*}
    \frac{S_n - E S_n}{n} \toup{P} 0, \quad n \to \infty
  \end{align*}

\end{theorem}

\begin{proof}
  \begin{align*}
    &P\pars{\walls{\frac{S_n - E S_n}{n}} \geq \epsilon} \leq \expl{нер-во Чебышева}
    \leq \frac{D\pars{\frac{S_n - E S_n}{n}}}{\epsilon^2} = 
    \frac{D(S_n - E S_n)}{n^2 \epsilon^2} =\\
    &= \frac{D S_n}{n^2 \epsilon^2} = \expl{$\xi_i$ и $\xi_j$ - некорр.}
    = \frac{\sum_{j = 1}^n D \xi_j}{n^2 \epsilon^2} \leq 
    \frac{C n}{n^2 \epsilon^2} \todown{n \to \infty} 0
  \end{align*}
\end{proof}

\begin{corollary}
  Пусть $\{ \xi_n,\; n \in \setN \}$ -- независимые случайные величины, т.ч. 
  $D \xi_n \leq C, \forall n$ и $E \xi_n = a, \forall n$.

  Тогда, обозначив $S_n = \xi_1 + \ldots + \xi_n$, получаем 
  \begin{align*}
    \frac{S_n}{n} \toup{P} a
  \end{align*}
\end{corollary}

\bigtitle{Смысл ЗБЧ:}

$\xi_1 \ldots \xi_n \ldots$ -- результаты независимых проведений одного и того же эксперимента. 

Тогда их среднее арифметическое сходится к среднему значению результата одного эксперимента $E\xi_i$

Если $\xi_i$ -- индикаторы наступления некоторого события $A$:
\begin{align*}
  \xi_i = I \{ A \text{ наступило в $i$-м эксперименте}\}
\end{align*}

то
\begin{align*}
  \frac{\xi_1 + \ldots + \xi_n}{n} \toup{P} E \xi_i = P(A)
\end{align*}

Таким образом ЗБЧ --- это принцип устойчивости частот постулировавшийся в начале курса.

\begin{lemma}[критерий сходимости почти наверное]~

  $\xi_n \toae \xi \quad \Leftrightarrow$ \quad для $\forall \epsilon > 0:$
  $P(\sup\limits_{k \geq n} |\xi_k - \xi| \geq \epsilon) \todown{n \to \infty} 0$

\end{lemma}

\begin{proof}~

  Обозначим $A_k^\epsilon = \{ |\xi_k - \xi| \geq \epsilon \}$ и
  $A^\epsilon = \bigcap\limits_{n = 1}^{\infty} \bigcup\limits_{k \geq n} A_k^\epsilon$

  Тогда 
  $\{ \xi_n \nrightarrow \xi \} = \bigcup\limits_{m = 1}^{\infty} A^{\frac{1}{m}}$

  Получаем
  \begin{align*}
    &P(\xi_n \nrightarrow \xi) = 0 \Leftrightarrow 
    P\pars{\bigcup_{m = 1}^{\infty} A^{\frac{1}{m}}} = 0 \Leftrightarrow 
    \forall m : P\pars{A^{\frac{1}{m}}} = 0 \Leftrightarrow 
    \forall \epsilon > 0 : P(A^\epsilon) = 0.\\
    &\text{Но $\bigcup\limits_{k \geq n} A_k^\epsilon \downarrow A^\epsilon$, поэтому }
    P(A^\epsilon) = \lim_{n \to \infty} P\pars{\bigcup_{k \geq n} A_k^\epsilon} = 0
    \Leftrightarrow P\pars{\bigcup_{k \geq n} A_k^\epsilon} \todown{n \to \infty} 0
  \end{align*}
  
  Оталось заметить, что $\bigcup\limits_{k \geq n}{A_k^\epsilon} = 
  \{ \sup\limits_{k \geq n} |\xi_k - \xi| \geq \epsilon \}$
\end{proof}

\begin{theorem}[взаимоотношение различных видов сходимости]~

  Выполнены соотношение
  \begin{enumerate}
    \item $\xi_n \toae \xi \Rightarrow \xi_n \toup{P} \xi$
    \item $\xi_n \toup{L^P} \xi \Rightarrow \xi_n \toup{P} \xi$
    \item $\xi_n \toup{P} \xi \Rightarrow \xi_n \toup{d} \xi$
  \end{enumerate}
\end{theorem}

\begin{proof}~

  \begin{enumerate}
    \item 
      Если $\xi_n \toae \xi$, то по лемме для $\forall \epsilon > 0:$
      \begin{align*}
        &P(\sup_{k \geq n} |\xi_k - \xi| \geq \epsilon) \todown{n \to \infty} 0,
        \quad\text{ но событие } \{ |\xi_n - \xi| \geq \epsilon \} \subset 
        \{ \sup_{k \geq n} |\xi_k - \xi| \geq \epsilon \}\\
        &\Rightarrow P(|\xi_n - \xi| \geq \epsilon) \leq 
        P(\sup_{k \geq n} |\xi_k - \xi| \geq \epsilon) \todown{n \to \infty} 0
      \end{align*}

    \item 
      $P(|\xi_n - \xi| \geq \epsilon) = P(|\xi_n - \xi|^P \geq \epsilon^P)
      \leq \expl{нер-во Маркова} \leq \dfrac{E|\xi_n - \xi|^P}{\epsilon^P} \todown{n \to \infty} 0$

    \item 
      Пусть $f(x)$ - ограниченная непрерывная функция, $|f(x)| \leq C, \forall x \in \setR$. 

      Пусть $\epsilon > 0$ -- фиксировано. Возьмем такое $N$, что
      \begin{align*}
        P(|\xi| > N) \leq \frac{\epsilon}{4 C}
      \end{align*}

      Функция $f(x)$ равномерно непрерывна на $[-N, N]$, т.е $\exists \delta > 0 :$
      $\forall x, y$ с условием $|x| \leq N$ и $|x - y| \leq \delta$ выполнено
      \begin{align*}
        |f(x) - f(y)| \leq \frac{\epsilon}{2}
      \end{align*}

      Рассмотрим следующее разбиение $\Omega$
      \begin{align*}
        &A_1 = \{ |\xi_n - \xi| \leq \delta, |\xi| \leq N \}\\
        &A_2 = \{ |\xi_n - \xi| \leq \delta, |\xi| > N \}\\
        &A_3 = \{ |\xi_n - \xi| > \delta\}\\
      \end{align*}

      Тогда
      \begin{align*}
        |E f(\xi_n) - E f(\xi)| \leq E |f(\xi_n) - f(\xi)| = 
        E(|f(\xi_n) - f(\xi)| (I_{A_1} + I_{A_2} + I_{A_3}))\; \circled{$\leq$}
      \end{align*}

      Если выполнено $A_1$, то $|f(\xi_n) - f(\xi)| \leq \frac{\epsilon}{2}$
      $\Rightarrow E|f(\xi_n) - f(\xi)| I_{A_1} \leq 
      \frac{\epsilon}{2} E I_{A_1} \leq \frac{\epsilon}{2}$

      Если выполнено $A_2$ или $A_3$, то $|f(\xi_n) - f(\xi)| \leq 2 C$

      \begin{align*}
        \Rightarrow\; &\circled{$\leq$}\; \frac{\epsilon}{2} + 2 C E(I_{A_2} + I_{A_3}) =
        \frac{\epsilon}{2} + 2C (P(A_2) + P(A_3)) \leq\\
        &\leq \frac{\epsilon}{2} + 2C P(|\xi| > N) + 2C P(|\xi_n - \xi| > \delta) 
        \leq \epsilon + 2C P(|\xi_n - \xi| > \delta)
      \end{align*}

      По условию $P(|\xi_n - \xi| > \delta) \todown{n \to \infty} 0$

      Значит в силу произвольности $\epsilon > 0$, 
      $E f(\xi_n) \to E f(\xi)$, т.е. $\xi_n \toup{d} \xi$
  \end{enumerate}

\end{proof}

\begin{remark}
  Сходимость по распределению случайных величин --- это, на самом деле, сходимость их распределений.
  \begin{align*}
    \xymatrix{
      \text{п.н.} \ar@{=>}[dr]  &           &\\
      &             P \ar@{=>}[r]  &d\\
      L^P \ar@{=>}[ur] &            &
    }
  \end{align*}

  Обратных стрелок нигде нет. Можно привести контрпримеры.
\end{remark}

\mysection{Усиленный закон больших чисел для случайных величин с ограниченными дисперсиями}

\begin{definition}
  Последовательность $\{ x_n,\; n \in \setN \}$ чисел из $\setR$ 
  называется \emph{фундаментальной}, если
  \begin{align*}
    |x_n - x_m| \to 0, \quad n, m \to +\infty
  \end{align*}
\end{definition}

\begin{theorem}[критерий Коши]
  Последовательность $\{ x_n,\; n \in \setN \}$ сходится $\Leftrightarrow$ она фундаментальна.
\end{theorem}

\begin{theorem}[критерий Коши сходимости почти наверное]
  Последовательность $\{ \xi_n,\; n \in \setN \}$ сходится почти наверное 
  $\Leftrightarrow \{ \xi_n, n \in \setN \}$ фундаментальна с вероятностью 1.
\end{theorem}

\begin{proof}~

  $(\Rightarrow)$ Пусть $\xi_n \toae \xi$.

  Тогда если $\omega \in \condset{\omega}{\lim\limits_{n \to \infty} \xi_n(\omega) \xi(\omega)}$, 
  то $\omega \in \{ \{ \xi_n(\omega) \} \text{ -- фундаментальна} \}$\\

  $\Rightarrow P(\{ \xi_n \} \text{ -- фундаментальна}) \geq 
  P(\lim\limits_{n \to \infty} \xi_n = \xi) = 1$\\

  $(\Leftarrow)$ Обозначим $A = \{ \{ \xi_n \} \text{ -- фундаментальна} \}$

  Тогда $\forall \omega \in A$ у $\xi_n(\omega)\; \exists$ предел $\xi(\omega)$
  \begin{align*}
    \xi(\omega) := \lim_{n \to \infty} \xi_n(\omega),\quad \text{ если $\omega \in A$}
  \end{align*}

  Если же $\omega \not\in A$, то положим $\xi(\omega) := 0$

  Тогда $\xi_n I_A \to \xi \Rightarrow \xi$ -- случайная величина(как предел случайных величин)
  \begin{align*}
    &P(\xi_n \to \xi) \leq P(\{ \xi_n \to \xi \} \cap A) = P(A) = 1\\
    &\Rightarrow \xi_n \toae \xi
  \end{align*}

\end{proof}

\begin{lemma}[критерий фундаментальности с вероятностью 1]~

  Последовательность $\{ \xi_n,\; n \in \setN \}$ фундаментальна с вероятностью 1 
  $\Leftrightarrow$ для $\forall \epsilon > 0:$
  \begin{align*}
    P(\sup_{k \geq n} |\xi_k - \xi_n| \geq \epsilon) \todown{n \to \infty} 0
  \end{align*}

\end{lemma}

\begin{proof}
  Полностью аналогично док-ву критерия сходимости почти наверное.
\end{proof}

\begin{theorem}[Колмогоров-Хинчин, достаточное условие для сходимости ряда почти наверное]~

  Пусть $\{ \xi_n,\; n \in \setN \}$ -- последовательность независимых случайных величин т.ч. 
  $E \xi_n = 0, \forall n$  и $E \xi_n^2 < +\infty, \forall n$

  Тогда если сходится $\sum\limits_n E\xi_n^2 < +\infty$, 
  то ряд $\sum\limits_n \xi_n$ сходится почти наверное.

\end{theorem}

