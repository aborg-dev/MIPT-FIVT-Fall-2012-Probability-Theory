
\mysection{Введение}

Предмет изучения теории вероятностей: \\
Математический анализ случайных явлений.\\

Эксперименты бывают:
\begin{itemize}
	\item Детерминированный результат (изучают другие науки)
	\item Случайный результат (теория вероятностей)\\
\end{itemize}

Одиночные результаты случайных экспериментов не позволяют обнаружить закономерности,
однако при большом числе результатов однородных случайных экспериментов обнаруживается \hbox{\emph{устойчивость частот}}. \\

\begin{example}
	Подбрасывание монетки:
	
	Бюфорон, XVIII век, 4040 подбрасываний, 2048 раз выпал орел, частота 0,508\ldots
	
	Пирсон, XIX век, 24000 подбрасываний, 12012 раз выпал орел, частота 0,5005\ldots
\end{example}

\boldtitle{Принцип устойчивости частот:}

Частота осуществления какого-либо исхода в последовательности однородных случайных экспериметов сходится
к некоторому числу $p \in [0, 1]$.

Пусть $A$ - некоторое событие, $U_n(A)$ - количетсво появлений $А$ в результатах случайных экспериментов после $n$ испытаний. Тогда 
\begin{equation*}
	\frac{U_n(A)}{n} \todown{n \to \infty} p(A) \text{ -- вероятность события } A.
\end{equation*}

Однако с математической точки зрения это неудобно. Нужно предложить другое определение вероятности, для которого будет наблюдаться устойчивость частот.\\

\mysection{Вероятностное пространство}

В основе теории вероятностей лежит понятие вероятностного пространства $(\Omega, \setF, P)$ 
(т.н ``тройки Колмогорова'')

\begin{enumerate}[label=\protect\circled{\arabic*},series=kolm_triple]

\item
	$\Omega$ --- \emph{пространство элементарных событий}.\\
	$\omega \in \Omega$ --- называется \emph{элементарным событием}.\\
	В результате случайного эксперимента получаем один и ровно один элемент $\Omega$.

\item
	$\setF$ --- $\sigma$-алгебра подмножеств на $\Omega$.\\
	Элементы $\setF$ называются \emph{событиями}.\\
	$\forall A \in \setF \implies A \subset \Omega$.

\end{enumerate}

\begin{definition}
	Система подмножеств $\setF$ множества $\Omega$ называется \emph{алгеброй}, если:

	\begin{enumerate}
		\item $\Omega \in \setF$
		\item $\forall A,B \in \setF \implies A \cap B \in \setF$
		\item $\forall A,B \in \setF \implies A \bigtriangleup B \in \setF$\\
	\end{enumerate}

\end{definition}

\begin{exercise}
	Алгебра замкнута относительно операций:
	\begin{enumerate}
		\item $A, B \in \setF \implies A \cup B \in \setF$
		\item $A, B \in \setF \implies A \setminus B \in \setF$
		\item $A \in \setF \implies \comp{A} \in \setF$\\
	\end{enumerate}
\end{exercise}

\begin{definition}
 	$\comp{A} = \Omega \setminus A$, называется дополнительным событием к событию $A$.\\
\end{definition}

\begin{example}~
	\begin{enumerate}
		\item $\setF_* = \braces{\emptyset, \Omega}$ --- тривиальная алгебра
		\item $\setF^* = 2^\Omega$ (все подмножества $\Omega$) --- дискретная алгебра
		\item $\setF = \braces{\emptyset, A, \comp{A}, \Omega}$ --- алгебра ``порожденная'' $A$
		\item Конечные объединения подмножеств вида 
					$[a, b), (-\infty; c), [d, +\infty)$ образуют алгебру.\\
	\end{enumerate}
\end{example}

\begin{definition}
	Система подмножеств $\setF$ множества $\Omega$ называется $\sigma$-алгеброй, если:
	\begin{enumerate}
		\item $\setF$ --- алгебра
		\item $\forall \braces{A_n, n \in \setN}, A_n \in \setF \; \forall n 
					\implies \bigcup\limits_{n = 1}^{+\infty} A_n \in \setF$\\
	\end{enumerate}
\end{definition}

\begin{exercise}
	Условие $\bigcup\limits_n A_n \in \setF$ можно заменить на $\bigcap\limits_n A_n \in \setF$
\end{exercise}

\begin{example}~
	\begin{enumerate}
		\item $\setF_* $ --- тривиальная $\sigma$-алгебра
		\item $\setF^* $ --- дискретная $\sigma$-алгебра
		\item $\forall$ конечная алгебра является $\sigma$-алгеброй.
		\item $[a, b), (-\infty; c), [d, +\infty)$ --- не $\sigma$-алгебра.\\
	\end{enumerate}
\end{example}

\begin{enumerate}[resume*=kolm_triple]
	\item
		$P$ - \emph{вероятностная мера} на $(\Omega, \setF)$
\end{enumerate}

\begin{definition}
	Пара $(\Omega, \setF)$ множества $\Omega$ с заданной на нем $\sigma$-алгеброй $\setF$ называется \emph{измеримым пространством}. 
\end{definition}

\begin{definition}
	Отображение $P\colon \setF \rightarrow [0;1]$ \\
	называется вероятностной мерой(или вероятностью) на $(\Omega, \setF)$, если:

	\begin{enumerate}
		\item $P(\Omega) = 1$
		\item Для $\forall$ последовательности $\braces{A_n, n \in \setN}, A_n \in \setF \; \forall n
		\text{ такой, что }  \forall i \neq j: \; A_i \cap A_j = \emptyset$ \\
		выполнено свойство счетной аддитивности:
		\begin{equation*}
			P\pars{\bigsqcup\limits_{n=1}^{\infty} A_n} = \sum\limits_{n=1}^{\infty} P(A_n)
		\end{equation*}
	\end{enumerate}
\end{definition}

\newpage

\begin{statement}~
	\begin{enumerate}
		\item $P(\emptyset) = 0$
		\item Если $A \cap B = \emptyset, \text{ то } P(A \cup B) = P(A) + P(B) $ (свойство конечной 			аддитивности)
		\item $P(\comp{A}) = 1 - P(A)$
		\item $P(A \cup B) = P(A) + P(B) - P(A \cap B)$
		\item $\forall A_1, \ldots, A_m \in \setF \\ 
				P\pars{\bigcup\limits_{n = 1}^{m} A_n} \leq \sum\limits_{n = 1}^{m} P(A_n) $
		\item Если $A \subset B, $ то $P(A) \leq P(B)$ \\
	\end{enumerate}

	\begin{proof}~
		\begin{enumerate}
			\item 
				$\displaystyle \forall n \; A_n = \emptyset \implies
				P\pars{\bigsqcup_{n = 1}^{\infty} A_n} = \sum_{n = 1}^{\infty} P(A_n) = 
				\sum_{n = 1}^{\infty} P(\emptyset) < +\infty
				\implies P(\emptyset) = 0$

			\item 
				$\displaystyle A_1 = A,\; A_2 = B,\; A_3 = A_4 = \ldots = A_n = \ldots = \emptyset \\ 
				P\pars{\bigsqcup_{n = 1}^{\infty} A_n} = P(A \cup B) = \sum_{n = 1}^{\infty} P(A_n) = P(A) + P(B)$

			\item $\Omega = A \sqcup \comp{A} \implies \expl{по 2} \implies 
						1 = P(A) + P(\comp{A})$

			\item 
				$A \cup B = A \sqcup (B \setminus (A \cap B)) \\
					\implies P(A \cup B) = P(A) + P(B \setminus (A \cap B))\\\\
				B = (A \cap B) \sqcup (B \setminus (A \cap B)) \\
					\implies P(B) = P(A \cap B)  + P(B \setminus (A \cap B))$

				Осталось вычесть одно равенство из другого.

			\item 
				Если $m = 2$ --- то это пункт 4). \\
				По индукции\\
				$\displaystyle P\pars{\bigcup_{n = 1}^m A_n} 
				\leq P(A_m) + P\pars{\bigcup_{n = 1}^{m - 1} A_n} 
				\leq \text{|индукция|} \leq P(A_m) + \sum_{n = 1}^{m - 1} P(A_n) = \sum_{n = 1}^{m} P(A_n)$

			\item Следует из 4).
		\end{enumerate}			
	\end{proof}
\end{statement}

\begin{definition}
	Будем обозначать $A_n \downarrow A$ при $n \to +\infty $, 
	если для последовательности событий $\braces{A_n, n \in \setN}$ выполнены свойства: 

	\begin{enumerate}
		\item $A_n \supset A_{n+1} \supset \ldots$
		\item $A = \bigcap\limits_{n}^{\infty} A_n$
	\end{enumerate}
\end{definition}

\begin{theorem}[О непрерывности в нуле вероятностной меры]
	Пусть $(\Omega, \setF)$ - измеримое пространство, 
	а $P\colon \setF \rightarrow [0, 1]$ удовлетворяет двум свойствам:
	\begin{enumerate}
		\item $P(\Omega) = 1$
		\item $P$ - конечно-аддитивна. 
	\end{enumerate}
	Тогда $P$ - вероятностная мера $\iff P$ - непрерывна в нуле(т.е если $A_n  \downarrow \emptyset$, то $P(A_n) \rightarrow 0).$
\end{theorem}

\begin{proof}~

	$(\implies)$ Пусть $P$ - вероятностная мера, а $A_n \downarrow \emptyset.$

	Рассмотрим $B_m = A_m \setminus A_{m+1}.$ Тогда в силу $\bigcap\limits_n A_n = \emptyset
	\implies \bigsqcup\limits_{m = n}^{\infty} B_m = A_n$

	Тогда в силу счетной аддитивности $P(A_n) = \sum\limits_{m = n}^{\infty} P(B_m)$

	Но ряд $P(A_1) = \sum\limits_{m=1}^{\infty} P(B_m) $ сходится
	$\implies \sum\limits_{m=n}^{\infty} P(B_m)$ есть остаток сходящего ряда 
	$\implies P(A_n) \rightarrow 0$\\

	$(\Leftarrow)$ Пусть $P$ непрерывна в нуле. 

	Покажем её счетную аддитивность:

	Пусть ${A_n, n \in \setN} $ т.ч $A_n \in F\; \forall n$ и 
	$A_i \cap A_j = \emptyset$ при $i \neq j$\\
	Рассмотрим $B_m = \bigsqcup\limits_{n=m}^{+\infty} A_n.$ 
	Тогда $B_m \supset B_{m+1} \supset \ldots$

	Покажем, что $\bigcap\limits_m B_m = \emptyset $. \\
	Пусть $\omega \in \bigcap\limits_m B_m 
	\implies \omega \in B_1 \implies \exists k: \omega \in A_k 
	\implies \omega \not\in B_{k+1}$. Противоречие.

	Следовательно, $\bigcap\limits_m B_m = \emptyset$ и в силу непрерывности в нуле 
	$P(B_m) \to 0$.

	Далее $\displaystyle P\pars{\bigsqcup_{n=1}^{\infty} A_n} = 
	P\pars{\bigsqcup_{n=1}^{m} A_m \sqcup B_{m+1}} 
	= \expl{конечная аддитивность} = \\
	= \sum_{n=1}^{m} P(A_n) + P(B_{m + 1}) \to \sum_{n = 1}^{\infty} P(A_n),\; m \to \infty\\
	\implies P\pars{\bigsqcup_n A_n} = \sum_n P(A_n)$
\end{proof}

\begin{corollary}[непрерывность вероятностной меры]~
	\begin{enumerate}
		\item Если $A_n \downarrow A, \text{ то } P(A_n) \to P(A)$
		\item Если $A_n \uparrow A$ (т.е $A_n \subset A_{n + 1} \subset \ldots $,
		и $A = \bigcup\limits_n A_n$, то $P(A_n) \to P(A)$	
	\end{enumerate}
\end{corollary}

\begin{proof}~
	\begin{enumerate}
		\item Надо рассмотреть $B_n = A_n \setminus A$
		\item Надо рассмотреть $B_n = \comp{A_n}$
	\end{enumerate}
\end{proof}

