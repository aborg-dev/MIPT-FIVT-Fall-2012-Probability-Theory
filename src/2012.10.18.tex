
\bigtitle{Примеры многомерных функций распределения}

\begin{example}
	\begin{enumerate}
		\item 
			Пусть $F_1(x_1), \ldots, F_n(x_n)$ -- одномерные функции распределения. 
			Тогда
			\begin{equation*}
				F(x_1, \ldots, x_n) = F_1(x_1) \ldots F_n(x_n)
			\end{equation*}
			--- многомерная функция распределения в $\setRn$.\\
						
			Заметим, что 
			\begin{align*}
				\Delta_{a_1 b_1}^1 \ldots \Delta_{a_n b_n}^n F(x_1, \ldots x_n) = 
				\prod\limits_{k = 1}^n (F_k(b_k) - F_k(a_k)))	\geq 0
			\end{align*}
			
			Если $F_i(x_i) = x_i$, для $\forall i = 1 \ldots n$ при $x_i \in [0, 1]$, то
			\begin{equation*}
				F(x_1, \ldots x_n) =
				\begin{cases}
					0, &\text{если } \exists i : x_i < 0\\					
					\prod\limits_{i=1}^n (x_i I\{x_i \in [0, 1]\} + I\{x_i \geq 1\}), &\text{иначе}
				\end{cases}
			\end{equation*}
			
			Такая $F$ соответствует для меры Лебега на $[0, 1]^n$.
		
		\item 
			Пусть $f(t_1, \ldots t_n),\; t_i \in \setR$ - функция в $\setRn$ т.ч\\
			$\int\limits_{\setR^n} f(t_1, \ldots, t_n)\; d t_1 \ldots d t_n = 1$ и 
			$f(t_1, \ldots, t_n) \geq 0$\\
			
			Тогда
			\begin{align*}
				&F(x_1, \ldots x_n) = \int\limits_{-\infty}^{x_1} \ldots 
															\int\limits_{-\infty}^{x_n} f(t_1 \ldots t_n)\; d t_1 \ldots d t_n\\
			\end{align*}
			--- многомерная функция распределения
			
			\begin{align*}
				&\Delta_{a_1 b_1}^1 \ldots \Delta_{a_n b_n}^n F(x_1, \ldots x_n) = \int\limits_{a_1}^{b_1} \ldots 									\int\limits_{a_n}^{b_n} f(t_1 \ldots t_n)\; d t_1 \ldots d t_n \geq 0		
			\end{align*}
			
			В этом случае $f(t_1 \ldots t_n)$ называется плотностью функции распределения $F(x_1 \ldots x_n)$ (или просто плотностью). Ясно, что 
			\begin{align*}
				f(x_1 \ldots x_n) = \frac{\partial^n}{\partial x_1 \ldots \partial x_n} F(x_1 \ldots x_n)\\
			\end{align*}
			
	\end{enumerate}
\end{example}


\bigtitle{Вероятностные меры в $\setRinf = \setR^{\setN}$}

Пусть $P$ -- вероятностная мера в ($\setRinf, B(\setRinf))$.
Для $\forall B_n \in B(\setRn)$ введем
\begin{align*}
	\setF_n (B_n) = \condset{\vec{x} \in \setRinf}{(x_1, \ldots, x_n) \in B_n}
\end{align*}
--- цилиндр с основанием $B_n$

Тогда $P_n(B_n) = P(\setF_n(B_n))$ является вероятностной мерой в ($\setRn, B(\setRn)$).
При этом имеет место свойство согласованности:

\begin{align*}
	P_{n + 1}(B_n \times \setR) = P_n(B_n)
\end{align*}

\begin{theorem}[Колмоговора, о мерах в $\setRinf$]~

	Пусть $\forall n$ задана вероятностная мера $P_n$  в $(\setRn, B(\setRn))$, 
	причем для $\{ P_n, n \in \setN \}$ выполнено свойтсво согласованности.\\
	Тогда $\exists !$ вероятностная мера $P$ в $(\setRinf, B(\setRinf))$,
	т.ч. $\forall n \, \forall B_n \in B(\setRn)$:
	\begin{align*}
		P_n(B_n) = P(\setF_n(B_n))
	\end{align*}
\end{theorem}

\mysection{Случайные величины в дискретных вероятностных пространствах}

Пусть $(\Omega, P)$ -- дискретное вероятностное пространство.

\begin{definition}
	Отображение $\xi\colon \Omega \to \setR$ называется \emph{случайной величиной}.

	Т.к $\Omega$ не более чем счетно, 
	то $\xi$ принимает не более чем счетное число значений ($a_1, a_2, \ldots$)

	Введем события $A_i = \condset{\omega}{\xi(\omega) = a_i}$ 
	-- состоит в том, что $\xi$ приняло значение $a_i$.
	
	$p_i = P(A_i) = P(\xi = a_i) \quad$ и $\quad \sum\limits_i p_i = 1 = \sum\limits_i P(A_i)$
\end{definition}

\begin{definition}
	Набор значений $(a_1, a_2, \ldots)$ и вероятностей $(p_1, p_2, \ldots)$, с которыми эти значения принимаются, вместе образуют распределение случайной величины $\xi$.
\end{definition}

\begin{remark}
	$\xi_1 \ldots \xi_n$ -- случайные величины, $\phi\colon \setRn \to \setR$ -- функция, 
	то $\phi(\xi_1, \ldots, \xi_n)$ -- тоже случайная величина.
\end{remark}

\begin{definition}
	Пусть $\xi$ -- случайная величина со значениями $(a_1, a_2, \ldots )$ и 
	$\eta$ -- случайная величина со значениями $(b_1, b_2, \ldots)$. 
	Случайные величины $\xi$ и $\eta$ называются независимыми, если\\
	$\forall i \, \forall j$ события $\{ \xi = a_i \}$ и  $\{ \eta = b_j \}$ независимы, т.е 
	\begin{align*}
		P(\xi = a_i, \eta = b_j) := P(\{ \xi = a_i \} \cap \{ \eta = b_j \}) = P(\xi = a_i) P(\eta = b_j)
	\end{align*}
\end{definition}

\begin{definition}
	Пусть $\xi_1, \ldots \xi_n$ - случайные величины, 
	$\xi_i$ принимает значения $(a_1^{(i)}, a_2^{(i)}, \ldots)$. 
	Тогда $\xi_1, \ldots \xi_n$ называют \emph{независимыми в совокупности}(взаимно независимыми), 
	если $\forall j_1, \ldots j_n$ выполнено:
	\begin{align*}
		P(\xi_1 = a_{j_1}^{(1)}, \ldots, \xi_n = a_{j_n}^{(n)}) = 
		\prod\limits_{k = 1}^n P(\xi = a_{j_k}^{(k)})
	\end{align*}
\end{definition}

\begin{example}~
	\begin{enumerate}
		\item 
			Бросок игральной кости.

			$\eta$ -- число очков, выпавшее на кости.\\
			Распределение $\eta$ -- равномерное на $\{ 1, \ldots 6 \}$\\
			
		\item 
			Пусть $A \subset \Omega$ -- событие. Тогда случайная величина
			\begin{equation*}
				I_A(\omega) =
				\begin{cases}
					1, & \omega \in A\\
					0, & \omega !\in A
				\end{cases}		
			\end{equation*}
			Называется индикатором события $A.$\\
			Другое обозначение: $I\{ A \}$.
		
		\item 
			$\xi$ называется \emph{биномиальной случайной величиной}, если
			она принимает значения $\{1, 2, \ldots, n\}$ и
			\begin{align*}
				P(\xi = k) = \comb{n}{k} p^k (1 - p)^{n - k},\; k = 0, \ldots, n
			\end{align*}

			Обозначение: $\xi \sim Bin(n, p)$
		
		\item 
			$\xi$ называется пуассоновской случайной величиной, $\xi \sim Pois(\lambda)$, 
			если $\xi$ принимает значения в $\setZ_+$ и
			\begin{align*}
				P(\xi = k) = \frac{\lambda^k e^{-\lambda}}{k!}, k \in \setZ_+
			\end{align*}
			$\lambda > 0$ -- параметр распределения.
	\end{enumerate}
\end{example}

\begin{exercise}~
	\begin{enumerate}
		\item $I_A$ и $I_B$ независимы $\Leftrightarrow$ $A$ и $B$ независимы
		\item 
			$\xi_1, \ldots, \xi_n$ -- с.в. Тогда они независимы в совокупности 
			$\Leftrightarrow$ $\forall x_1, \ldots, x_n \in \setR$:
			\begin{align*}
				P(\xi_1 = x_1, \ldots, \xi_n = x_n) = \prod\limits_{k = 1}^{n} P(\xi_k = x_k)
			\end{align*}

		\item
			Если $\xi$ и $\eta$ -- независимы, и $\xi \sim Bin(n, p)$, $\eta \sim Bin(m, p)$, то
			$\xi + \eta \sim Bin(n + m, p)$.

		\item
			Если $\xi \sim Pois(\lambda_1)$, $\eta \sim Pois(\lambda_2)$, то
			$\xi + \eta \sim Pois(\lambda_1 + \lambda_2)$.

	\end{enumerate}
\end{exercise}

\begin{definition}
	Пусть $\xi$ -- случайная величина. Её \emph{математическим ожиданием} называют 
	\begin{align*}
		E \xi = \sum_{\omega \in \Omega} \xi(\omega) P(\omega)
	\end{align*}
	Если ряд в правой части сходится абсолютно.
\end{definition}

\begin{example}
	В классической модели $\Omega$ -- конечно и $P(\omega) = 
	\frac{1}{|\Omega|}$ для $\forall \omega \in \Omega$. Тогда
	\begin{align*}
		E \xi = \sum_{\omega \in \Omega} \frac{\xi(\omega)}{|\Omega|} = 
		\frac{1}{|\Omega|} \sum\limits_{\omega \in \Omega} \xi(\omega)
	\end{align*}
	--- среднее арифметическое значений.
\end{example}

\begin{lemma}[свойства математического ожидания]~

	\begin{enumerate}
		\item Линейность 
			\begin{align*}
				E(a \xi + b \eta) = a E \xi + b E \eta, \quad a, b \in \setR
			\end{align*}
			
		\item Пусть $\eta$ принимает значения $(a_1, a_2, \ldots)$. Тогда
			\begin{align*}
				E \xi = \sum_i a_i P(\xi = a_i)
			\end{align*}
		
		\item Пусть $\xi$ - принимает значения $(a_1, a_2, \ldots)$\\
			Тогда для $\forall$ функции $\phi(x)$:
			\begin{align*}
				E \phi(\xi) = \sum_i \phi(a_i) P(\xi = a_i)
			\end{align*}
		
		\item Если $\xi \leq \eta$, то $E \xi \leq E \eta$
		
		\item Если $\xi$ и $\eta$ - независимы, то
			\begin{align*}
				E \xi \eta = E \xi E \eta
			\end{align*}
	\end{enumerate}
\end{lemma}

\begin{proof}~

	\begin{enumerate}
		\item Очевидно из определения.
		\item 
			\begin{align*}
				&E \xi = \sum_{\omega \in \Omega} \xi(\omega) P(\omega) = 
				\sum_i \sum_{\omega : \xi(\omega)	= a_i} \xi(\omega) P(\omega) = \\
				&\sum_i \sum_{\omega : \xi(\omega) = a_i} a_i P(\omega) = 
				\sum_i a_i \sum_{\omega : \xi(\omega) = a_i} P(\omega) = \sum_i a_i P(\xi = a_i)
			\end{align*}
			
		\item Аналогично 2)
		
		\item Очевидно из определения.
		
		\item \begin{align*}
				&E \xi \eta = \sum_{\omega \in \Omega} \xi(\omega) \eta(\omega) P(\omega) = 
				\sum_{i, j} \sum_{\substack{\omega : \xi(\omega) = a_i\\\;\eta(\omega) = b_j}} 
				\eta(\omega) \xi(\omega) P(\omega) = \sum\limits_{i, j} a_i b_j 
				\sum\limits_{\substack{\omega : \xi(\omega) = a_i\\\;\eta(\omega) = b_j}} P(\omega) =\\
				&= \sum\limits_{i, j} a_i b_j P(\xi = a_i, \eta = b_j) = \expl{независимость} = 
				\sum\limits_{i, j} a_i b_j P(\xi = a_i) P(\eta = b_j) = \\
				& = \pars{ \sum\limits_i a_i P(\xi = x_i) } \pars{ \sum\limits_j b_j P(\eta = b_j) } 
				= E \xi E \eta
			\end{align*}						
	\end{enumerate}
\end{proof}

\begin{corollary}
	Для матожидания $E \xi$ (и $E \phi(\xi)$) достаточно знать распределение случайной величины $\xi$.
\end{corollary}

\begin{definition}
	$E \xi^k$ -- \emph{момент порядка $k$} случайной величины $\xi$ ($k$-й момент)\\
	
	$E(\xi - E\xi)^k$ -- \emph{центральный момент порядка $k$} случайной величины $\eta$ 
	($k$-й центральный момент).\\
	
	$E\xi(\xi - 1) \ldots (\xi - k + 1)$ -- \emph{факториальный момент порядка $k$}
	 случайной величины $\eta$, $k \in \setN$\\
	
	$D\xi = E(\xi - E\xi)^2$ -- \emph{дисперсия} случайной величины $\xi$
\end{definition}

\begin{lemma}[свойства дисперсии]~
	\begin{enumerate}
		\item $D\xi = E\xi^2 - (E\xi)^2$
		\item $D\xi \geq 0$
		\item $D(c\, \xi) = c^2 D\xi$
		\item $D\xi = 0 \Leftrightarrow P(\xi = E\xi) = 1$
	\end{enumerate}
\end{lemma}

\begin{statement}
	Если $\xi$ -- биномиальная: $\xi \sim Bin(n, p)$, то $D\xi = np (1 - p)$
\end{statement}

\begin{definition}
	Пусть $\xi$ и $\eta$ - две случайные величины. Ковариацией случайных величин $\xi$ и $\eta$ называется
	\begin{align*}
		\cov(\xi, \eta) = E(\xi - E\xi) (\eta - E\eta)
	\end{align*}
	Если $\cov(\xi, \eta) = 0$, то $\xi$ и $\eta$ называются некоррелированными.
\end{definition}

\begin{enumerate}
	\item $\cov(\xi, \eta) = E \xi \eta - E \xi E \eta$

	\item Если $\xi$ и $\eta$ независимы, то они не коррелируют. (обратное неверно!)

	\item $D\xi = \cov(\xi, \xi)$
\end{enumerate}

\begin{statement}
	\begin{equation*}
		D(\xi + \eta) = D\xi + D\eta + 2 \cov(\xi, \eta)
	\end{equation*}
\end{statement}

\begin{proof}
	\begin{align*}
		&D(\xi + \eta) = E(\xi + \eta - E(\xi + \eta))^2 = E(\xi - E\xi)^2 + E(\eta - E\eta)^2
		+ 2 E(\xi - E\xi)(\eta - E\eta) = \\ 
		&= D\xi + D\eta + 2 \cov(\xi, \eta)
	\end{align*}
\end{proof}

\begin{corollary}
	Если $\xi$ и $\eta$ независимы, то $D(\xi + \eta) = D\xi + D\eta$
\end{corollary}

