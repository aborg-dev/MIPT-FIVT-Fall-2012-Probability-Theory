
\begin{lemma}[Неравенство Колмогорова]~

  Пусть $\xi_1 \ldots \xi_n$ -- независимые с.в.

  $E \xi_i = 0$ и $E \xi_i^2 < +\infty$. 
  Обозначим $S_k = \xi_1 + \ldots + \xi_k$

  Тогда
  \begin{align*}
    P\pars{\max_{1 \leq k \leq n}|S_k| \geq \epsilon} \leq \frac{E S_n^2}{ \epsilon^2}
  \end{align*}
\end{lemma}

\begin{proof}
  Обозначим $A = \braces{\max\limits_{1 \leq k \leq n}{|S_k|} \geq \epsilon }$
    
  Разделим $A$ на следующие части:

  $A_k = \{|S_k| \geq \epsilon $ и $|S_i| < \epsilon$ для $i = 1 \ldots k - 1\}$.

  Тогда $A_k \cap A_j = \emptyset$ при $k \neq j$ и $A = \bigsqcup\limits_{k = 1}^{n} A_k$

  Рассмотрим 
  \begin{align*}
    E S_n^2 \geq E(S_n^2 I_A) = E \sum_{k = 1}^{n} (S_n^2 I_{A_k}) = \sum_{k = 1}^n E(S_n^2 I_{A_k})
  \end{align*}

  \begin{align*}
    E S_n^2 I_{A_k} &= E(S_k + \xi_{k + 1} + \ldots + \xi_n)^2 I_{A_k} =\\
    &= E S_k^2 I_{A_k} + 2 E S_k (\xi_{k + 1} + \ldots + \xi_n) I_{A_k} + 
    E(\xi_{k + 1} + \ldots + \xi_n)^2 I_{A_k}
  \end{align*}

  Но $I_{A_k}$ завиcит от $S_1 \ldots S_k \implies S_k I_{A_k}$ 
  не зависит от $\xi_{k + 1} \ldots \xi_n$

  Следовательно, второе слагаемое

  \begin{align*}
    &E S_k I_{A_k} (\xi_{k + 1} + \ldots + \xi_n) = 
    E S_k I_{A_k} E(\xi_{k + 1} + \ldots \xi_n) = 0 \quad (\forall i: E \xi_i = 0)\\
    &\implies E S_n^2 I_{A_k} = E S_k^2 I_{A_k} + E(\xi_{k + 1} + \ldots + \xi_n)^2 I_{A_k} 
    \geq E S_k^2 I_{A_k} \geq \epsilon^ 2 E I_{A_k} = \epsilon^2 P(A_k)
  \end{align*}

  т.к $S_k \geq \epsilon$ на $A_k$.

  В итоге
  \begin{align*}
    E S_n^2 \geq \sum_{k = 1}^{n} E (S_n^2 I_{A_k}) \geq 
    \epsilon^2 \sum_{k = 1}^{n} P(A_k) = \epsilon^2 P(A)
  \end{align*}
\end{proof}

\begin{proof}[Док-во теоремы Колмогорова-Хинчина]~

  Обозначим $S_n = \sum\limits_{k = 1}^{n} \xi_k$. 
  Тогда $\sum\limits_{k = 1}^{\infty} \xi_k$ сходится п.н. 
  $\iff$ (критерий Коши) $\iff$

  $\iff$ $\{ S_n,\; n \in \setN \}$ фундаментальна с вероятностью 1 
  $\iff$ (критерий фундаментальности) $\iff$

  $\iff$ для $\forall \epsilon > 0: P(\sup\limits_{k \geq n} |S_k - S_n| \geq \epsilon) 
  \todown{n \to \infty} 0$

  Оценим её: Рассмотрим.
  \begin{align*}
    &P(\sup_{k \geq n}{|S_k - S_n| \geq \epsilon}) = 
    P(\bigcup_{k \geq n} \{ |S_k - S_n| \geq \epsilon \}) = \expl{непрерывность вер. меры} =\\
    &= \lim_{N \to \infty} P(\bigcup_{k = n}^{N + n} \{ |S_k - S_n| \geq \epsilon \})
    = \lim_{N \to \infty} P(\max_{1 \leq k \leq N} |S_{k + n} - S_n| \geq \epsilon)
    = \expl{нер-во Колмогорова} \leq \\
    &\leq \lim_{N \to \infty} \frac{E(S_{n + N} - S_n)^2}{\epsilon^2} = 
	\expl{независимость, $E\xi_i = 0$}
    = \lim_{N \to \infty} \frac{\sum_{k = n + 1}^{n + N} E \xi_k^2}{\epsilon^2} 
    = \frac{\sum_{k = n + 1}^{\infty} E \xi_k^2}{\epsilon^2} \todown{n \to \infty} 0
  \end{align*}

  т.к. это остаток сходящегося ряда (по условию $\sum\limits_n E \xi_n^2 < +\infty$)

\end{proof}



\begin{lemma}[Тёплиц]~

  Пусть последовательность $x_n \to x,\; \{a_n,\; n \in \setN\}$ т.ч. $a_n \geq 0$ и 
  $b_n = \sum\limits_{j = 1}^{n} a_j \uparrow +\infty$.

  Тогда 
  \begin{align*}
    \frac{1}{b_n} \sum\limits_{j = 1}^{n} a_j x_j \todown{n \to \infty} x
  \end{align*}
  
  \begin{proof}
    Пусть $\epsilon > 0$ -- произвольное. Возьмём $n_0 = n_0(\epsilon)$ т.ч. 
    $\forall n > n_0: |x - x_n| < \frac{\epsilon}{2}$

    Далее, возьмем $n_1 > n_0$, т.ч. 
    \begin{align*}
      \frac{1}{b_{n_1}} \sum_{j = 1}^{n_0} a_j |x_j - x| < \frac{\epsilon}{2}
    \end{align*}

    Тогда для $\forall n > n_1$
    \begin{align*}
      &\walls{\pars{\frac{1}{b_n} \sum_{j = 1}^{n} a_j x_j} - x} = 
      \walls{\frac{1}{b_n} \sum_{j = 1}^{n} a_j (x_j - x)} \leq \\
      &\leq \frac{1}{b_n} \sum_{j = 1}^{n_0} a_j |x_j - x| + 
      \frac{1}{b_n} \sum_{j = n_0 + 1}^{n} a_j |x_j - x| \leq 
      \frac{\epsilon}{2} + \frac{\epsilon}{2}\; \frac{1}{b_n} \sum_{j = n_0 + 1}^{n} a_j 
      \leq \epsilon
    \end{align*}
  \end{proof}

\end{lemma}

\begin{lemma}[Кронекер]~

  Пусть ряд $\sum\limits_n x_n$ сходится.

  $\{a_n,\; n \in \setN\}$ -- некоторая последовательность $a_n \geq 0$ т.ч. 
  $b_n = \sum\limits_{j = 1}^{n} a_j \uparrow +\infty$

  Тогда 
  \begin{align*}
    \frac{1}{b_n} \sum\limits_{j = 1}^{n} b_j x_j \todown{n \to \infty} 0
  \end{align*}

  \begin{proof}
    Обозначим $S_n = x_1 + \ldots + x_n$. Тогда $\{ S_n \}$ сходится.
    
    \begin{align*}
      \sum_{j = 1}^{n} b_j x_j = \sum_{j = 1}^{n} b_j (S_j - S_{j - 1}) 
      = b_n S_n - \sum_{j = 1}^{n} S_{j - 1} (b_j - b_{j - 1}) = 
      b_n S_n - \sum_{j = 1}^{n} S_{j - 1} a_j
    \end{align*}

    Делим на $b_n$:
    \begin{align*}
      &\frac{1}{b_n} \sum_{j = 1}^{n} b_j x_j = S_n - \frac{1}{b_n} \sum_{j = 1}^{n} S_{j - 1} a_j\\
      &S_n \todown{n \to \infty} \sum_{n = 1}^{\infty} x_n = S
    \end{align*}

    А по лемме Тёплица:
    \begin{align*}
      \frac{1}{b_n} \sum_{j = 1}^{n} S_{j - 1} a_j \todown{n \to \infty} S
    \end{align*}

    $\implies$ их разность стремится к нулю.
  \end{proof}

\end{lemma}

\begin{theorem}[Усиленный закон больших чисел в форме Колмогорова-Хинчина]~

  Пусть $\{ \xi_n,\; n \in \setN \}$ -- независимые с.в. т.ч. $D \xi_n < +\infty \forall n$.

  Пусть последовательность $\{ b_n,\; n \in \setN \}$ т.ч. $b_n > 0, b_n \uparrow +\infty$ и
  \begin{align*}
    \sum_{n = 1}^{\infty} \frac{D \xi_n}{b_n^2} < +\infty
  \end{align*}

  Обозначим $S_n = \xi_1 + \ldots \xi_n$.
  Тогда 
  \begin{align*}
    \boxed{\frac{S_n - E S_n}{b_n} \toae 0} \quad (\text{при } n \to \infty)
  \end{align*}

  \begin{proof}
    Рассмотрим 
    \begin{align*}
      \frac{S_n - E S_n}{b_n} = 
      \frac{1}{b_n} \sum_{k = 1}^{n} b_k \pars{\frac{\xi_k - E \xi_k}{b_k}}
    \end{align*}

    Далее с.в. $\eta_k = \frac{\xi_k - E \xi_k}{b_k}$ -- независимы и $E \eta_k = 0$

    Тогда
    \begin{align*}
      \sum_{k = 1}^{\infty} E \eta_k^2 = 
      \sum_{k = 1}^{\infty} E \pars{\frac{\xi_k - E \xi_k}{b_k}}^2 = 
      \sum_{k = 1}^{\infty} \frac{D \xi_k}{b_k^2} < +\infty
    \end{align*}

    $\implies$ по теореме о сходимости ряда, ряд $\sum\limits_k \eta_k$ сходится п.н.
    
    Но по лемме Кронекера 
    $\frac{1}{b_n} \sum\limits_{k = 1}^{n} b_k \pars{\frac{\xi_k - E \xi_k}{b_k}}$ 
    сходится к нулю для всех $\omega$, для которых
    \begin{align*}
      \sum_{k = 1}^{\infty} \frac{\xi_k - E \xi_k}{b_k} = \sum_{k = 1}^{\infty} \eta_k
    \end{align*}
    сходится. А этот ряд сходится.

    \begin{align*}
      \Longrightarrow \frac{S_n - E S_n}{b_n} \toae 0
    \end{align*}
  \end{proof}
\end{theorem}

\begin{corollary}
  Пусть $\{ \xi_n,\; n \in \setN \}$ -- независимые случайные величины т.ч. 
  $D \xi_n \leq C \; \forall n \in \setN$

  Обозначим $S_n = \xi_1 + \ldots + \xi_n$.

  Тогда 
  \begin{align*}
    \frac{S_n - E S_n}{n} \toae 0
  \end{align*}

  Если, к тому же, $E \xi_i = a \forall i$, то 
  \begin{align*}
    \frac{S_n}{n} \toae a
  \end{align*}

  \begin{proof}
    Возьмем $b_n = n \implies b_n > 0,\; b_n \uparrow +\infty$.

    Тогда
    \begin{align*}
      \sum_n \frac{D \xi_n}{b_n^2} = \sum_n \frac{D\xi_n}{n^2} \leq 
      \sum_n \frac{c}{n^2} < +\infty
    \end{align*}

    Согласно УЗБЧ
    \begin{align*}
      \frac{S_n - E S_n}{n} \toae 0,\quad (n \to \infty)
    \end{align*}

    Если же $E \xi_n = a$, то $E S_n = n - a$
    \begin{align*}
      \frac{S_n}{n} - a \toae 0 \iff \frac{S_n}{n} \toae a
    \end{align*}

  \end{proof}
\end{corollary}

Смысл УЗБЧ: обоснование феномена устойчивости частот появлений событий в последовательностях независимых экспериментов.

Если $\xi_i = I\{ \text{событие $A$ произошло в $i$- том эксперимете} \}$ то

\begin{align*}
  \nu_n(A) = \frac{\xi_1 + \ldots + \xi_n}{n} \toae E \xi_1 = P(A)
\end{align*}

\mysection{Предельный переход под знаком $E$}

Вопрос: $\xi_n \toae \xi \implies  E\xi_n \rightarrow E\xi$?

\begin{theorem}[О монотонной сходимости]~

  Пусть $\{\xi_n,\; n \in \setN\}, \xi, \eta$ -- с.в.

  \begin{enumerate}
    \item 
      Если $\xi_n \uparrow \xi,\; \xi_n \geq \eta, \forall n \in \setN$ и
      $E \eta > -\infty,$ то $E\xi = \lim\limits_{n \to \infty} E \xi_n$
      
    \item 
      Если $\xi_n \downarrow \xi, \xi_n \leq \eta, \forall n \in \setN$ и $E \eta < +\infty,$
      то $E\xi = \lim\limits_{n \to \infty} E \xi_n$
  \end{enumerate}

\end{theorem}

\begin{theorem}[лемма Фату]~

  Пусть $\{\xi_n, n \in \setN\}, \eta$ -- с.в.,\; $E\eta $ - конечно

  \begin{enumerate}
    \item
      Если $\xi_n \geq \eta, \forall n \in \setN$, то \quad
      $\limdown_n E \xi_n \geq E \limdown_n \xi_n$ 

    \item
      Если $\xi_n \leq \eta, \forall n \in \setN$, то \quad
      $\limup_n E \xi_n \leq E \limup_n \xi_n$ 

    \item
      Если $|\xi_n| \leq \eta, \forall n \in \setN$, то \quad
      $E \limdown_n \xi_n \leq \limdown_n E \xi_n \leq \limup_n E\xi_n \leq E \limup_n \xi_n$ 
  \end{enumerate}

  \begin{proof}~

    \begin{enumerate}
      \item
        Обозначим $\psi_n = \inf\limits_{k \geq n} \xi_k$. Тогда
        $\psi_n \uparrow \limdown_n \xi_n$ и $\psi_n \geq \eta, \forall n \in \setN$.

        По теореме о монотонной  сходимости получаем
        \begin{align*}
          \lim_n E \psi_n = E  \limdown_n \xi_n
        \end{align*}

        Осталось заметить, что 
        \begin{align*}
          E \limdown_n \xi_n = \lim_n E \psi_n = \limdown_n E \psi_n \leq \limdown_n E \xi_n
        \end{align*}

        т.к. $\xi_n \geq \psi_n, \forall n$

      \item
        Следует из $1)$ заменой $\xi_n$ на $- \xi_n$

      \item
        Сразу следует из $1)$ и $2)$
    \end{enumerate}
  \end{proof}
\end{theorem}

\begin{theorem}[Лебега о мажорируюмой сходимости]~

  Пусть $\{ \xi_n,\; n \in \setN \}$ -- последовательность с.в. т.ч. $\xi_n \toae \xi$
  и для $\forall n : |\xi_n| \leq \eta$, причем $E \eta$ конечно.

  Тогда $E \xi = \lim\limits_n E\xi_n$ и, более того,
  $E |\xi_n - \xi| \to 0$ (т.е. $\xi_n \toup{L^1} \xi)$

  \begin{proof}
    Заметим, что $\xi = \lim\limits_n \xi_n = \limdown_n \xi_n = \limup_n \xi_n$ п.н.

    $\implies$ по лемме Фату

    \begin{align*}
      &E \xi = E \limdown_n \xi_n \leq \limdown_n E \xi_n \leq \limup_n E\xi_n \leq 
      E \limup_n \xi_n = E\xi\\
      &\implies \lim_n E\xi_n = E\xi
    \end{align*}

    Конечность $E \xi$ следует из того, что $|\xi| \leq \eta$ п.н. и конечности $E \eta$

    Для обоснования сходимости в $L^1$ достаточно взять $\psi_n = |\xi_n - \xi|$. 

    Тогда $|\psi_n| \leq 2 |\eta|$ п.н.
    и $\psi_n \toae 0 \implies E \psi_n \to 0$

  \end{proof}
\end{theorem}

\mysection{Усиленный закон больших чисел для с.в. с конечным математическим ожиданием}

\begin{definition}
  Пусть $\{ A_n,\; n \in \setN \}$ -- последовательность событий. 

  Тогда событием $\{ A_n \text{ бесконечное число} \}$ = $\{ A_n \text{ б.ч} \}$ наз. событие, 
  заключающееся в том, что произошло бесконесное число событий в последовательности
  $\{A_n,\; n \in \setN\}$. Формально:
  \begin{align*}
    \{ A_n \text{ б.ч.} \} = \bigcap_{n = 1}^{\infty} \bigcup_{k \geq n} A_k
  \end{align*}
\end{definition}

\begin{lemma}[Борель-Кантелли]~

  \begin{enumerate}

    \item
      Если $\sum\limits_n P(A_n) < +\infty$, то $P(A_n \text{ б.ч.}) = 0$

    \item
      Если $\sum\limits_n P(A_n) = +\infty$ и все $A_n$ - независимые, то $P(A_n \text{ б.ч.}) = 1$

  \end{enumerate}

  \begin{proof}~

    \begin{enumerate}
      \item
        \begin{align*}
          P(A_n \text{ б.ч.}) = P\pars{\bigcap_{n = 1}^{\infty} \bigcup_{k \geq n} A_k} = 
          \lim_{n \to \infty} P\pars{\bigcup_{k \geq n} A_k}
          \leq \lim_{n \to \infty} \sum_{k \geq n} P(A_k) = 0
        \end{align*}

      \item
        \begin{align*}
          P(A_n \text{ б.ч.}) = \lim_{n \to \infty} P\pars{\bigcup_{k \geq n} A_k} = 
          \lim_{n \to \infty} \pars{1 - P\pars{\bigcap_{k \geq n} \comp{A}_k}}
        \end{align*}

        Но
        \begin{align*}
          &P\pars{\bigcap_{k \geq n} \comp{A}_n} = \expl{непр. вер. меры} =
          \lim_{N \to \infty} P\pars{\bigcap_{k \geq n}^{N} \comp{A}_k} = 
          \lim_{N \to \infty} \prod_{k = n}^{N} P(\comp{A}_k) = \\
          &= \lim_{N \to \infty} \prod_{k = n}^{N} (1 - P(A_k)) 
          \leq \lim_{N \to \infty} \prod_{k = n}^{N} e^{-P(A_k)} = 
          \lim_{N \to \infty} e^{-\sum\limits_{k = n}^N P(A_k)} = 
          e^{-\sum\limits_{k = n}^{+\infty} P(A_k)} = 0
        \end{align*}
        т.к. $\forall n : \sum\limits_{k = n}^{\infty} P(A_k) = +\infty$\\

        $\implies P(A_n \text{ б.ч.}) = 1$
    \end{enumerate}
  \end{proof}
\end{lemma}

