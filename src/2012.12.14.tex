
\mysection{Характеристические функции}

\begin{definition}
  Характеристической функцией с.в. $\xi$ называется
  \begin{align*}
    \phi_{\xi} (t) = E e^{i t \xi}, \quad t \in \setR
  \end{align*}
\end{definition}

\begin{remark}
  Характеристическая функция, вообще говоря, явл. комплекснозначной. 
  Мы понимаем $E e^{i t \xi}$ как
  \begin{align*}
    E e^{i t \xi} = E \cos(t\xi) + i E \sin(t \xi)
  \end{align*}
\end{remark}

\begin{definition}
  Пусть $F(x),\; x \in \setR$ -- функция распределения на $\setR$\\
  Её характеристической функцией наз.
  \begin{align*}
    \phi(t) = \int\limits_{\setR} e^{i t \xi} dF(x)
  \end{align*}

  Если $P$ -- вероятностная мера на $(\setR, B(\setR))$, то её характеристической ф-ей наз.
  \begin{align*}
    \phi(t) = \int\limits_{\setR} e^{it\xi} P(dx)
  \end{align*}
\end{definition}

\begin{corollary}
  $\phi_\xi (t)$ -- х.ф. с.в. $\xi \Leftrightarrow  \phi_{\xi} (t)$ -- х.ф. $F_{\xi}(x)$
  $\Leftrightarrow \phi_{\xi} (t)$ -- х.ф. $P_{\xi}$ (распр. $\xi$)

  \begin{proof}
    \begin{align*}
      \phi_{\xi} (t) = E e^{i t \xi} = \int\limits_{\setR} e^{i t x} P_{\xi} (dx) 
      = \int\limits_{\setR} e^{i t x} dF_{\xi} (x)
    \end{align*}
  \end{proof}

\end{corollary}

\begin{definition}
  Пусть $\xi = (\xi_1, \ldots, \xi_n) $ -- случайный вектор.
  Его характеристической функцией наз.
  \begin{align*}
    \phi_{\xi} (t) = E e^{i \langle t \xi \rangle}, \text{ где } t = (t_1, \ldots, t_n) \in \setRn, 
    \text{ а } \langle t, \xi \rangle = \sum_{i = 1}^{n} t_i \xi_i
  \end{align*}
\end{definition}

\begin{definition}
  Пусть $F(x), \; x \in \setR$ -- функция распр. в $\setRn$.

  Её х.ф. наз.
  \begin{align*}
    \phi(t) = \int\limits_{\setRn} e^{i \langle t, x \rangle} dF(x),\quad t \in \setRn
  \end{align*}

  Если $P$ -- вероятносная мера в $\setRn$ , то её х.ф. наз
  \begin{align*}
    \phi(t) = \int\limits_{\setRn} e^{i \langle t, x \rangle} P(dx),\quad t \in \setRn
  \end{align*}
\end{definition}

\begin{corollary}
  Если $\xi = (\xi_1, \ldots, \xi_n)$ -- сл. вектор, то
  $\phi_\xi(t)$ -- х.ф. $\xi \Leftrightarrow \phi_\xi(t)$ -- х.ф. $F_\xi (x), x~\in~\setRn
  \Leftrightarrow \phi_\xi (t)$ -- х.ф. $P_\xi$
\end{corollary}

\begin{example}~
  \begin{enumerate}
    \item
      $\xi \sim Bern(p)$, бернуллевская с.в.,
      $P(\xi = 1) = p,\quad \; P(\xi = 0) = 1 - p$.

      Тогда 
      \begin{align*}
        \phi_\xi (t) = E e^{i t \xi} = e^{i t} P (\xi = 1) + e^{i t 0} P(\xi = 0) 
        = p e^{it} + 1 - p
      \end{align*}

    \item
      $\xi \sim Pois(\lambda)$, пуассоновская с.в.
      \begin{align*}
        \phi_\xi (t) = E e^{i t \xi} = \sum_{k = 0}^{\infty} e^{i t k} P(\xi = k) 
        = \sum_{k = 0}^{\infty} e^{i t k} \frac{\lambda^k}{k!} e^{-\lambda} 
        = \pars{\sum_{k = 0}^{\infty} \frac{(e^{i t} \lambda)^k}{k!}} e^{-\lambda}
        = e^{\lambda(e^{it} - 1)}
      \end{align*}

    \item
      $\xi \sim Exp(\lambda)$ экспоненциальная с.в.
      \begin{align*}
        \phi_\xi (t) = E e^{i t \xi} = \int\limits_{0}^{+\infty} e^{i t x} \lambda e^{-\lambda x} dx 
        = \lambda \int\limits_{0}^{+\infty} e^{(it - \lambda) x} dx = \frac{\lambda}{\lambda - i t}
      \end{align*}
  \end{enumerate}
\end{example}

\bigtitle{Основные свойства характеристических функций}

\begin{enumerate}[label=\protect\circled{\arabic*},series=charfunc_properties]
  \item
    Пусть $\phi(t)$ -- х.ф. с.в. $\xi$. 

    Тогда $|\phi(t)| \leq \phi(0) = 1, \; \forall t \in \setR$
    \begin{proof}
      \begin{align*}
        |\phi(t)| = |E e^{i t \xi}| \leq E |e^{i t \xi}| = 1 = \phi(0)
      \end{align*}
    \end{proof}

  \item 
    Пусть $\phi(t)$ -- хар. ф. с.в. $\xi$, а $\eta = a \xi + b, \; a, b \in \setR$.
    Тогда
    \begin{align*}
      \phi_{\eta} (t) = e^{i t b} \phi_{\xi} (t a)
    \end{align*}

    \begin{proof}
      \begin{align*}
        \phi_{\eta} (t) = E e^{i t \eta} = E e^{i t (a \xi + b)} 
        = e^{i t b} E \phi_{i (a t) \xi} = e^{i t b} \phi_\xi (at)
      \end{align*}
    \end{proof}

  \item
    Пусть $\phi(t)$ -- х.ф.с.в. $\xi$. 
    Тогда $\phi(t)$ равномерно непрерывна на $\setR$.

    \begin{proof}
      \begin{align*}
        |\phi(t + h) - \phi(t)| = \walls{E e^{i (t + h) \xi} - E e^{i t \xi}} 
        = \walls{E(e^{i(t + h)\xi} - e^{i t \xi})} = \walls{E(e^{i t \xi} (e^{i h \xi} - 1))}
        = E |e^{i h \xi} - 1|
      \end{align*}
      
      При $h \to 0, \; e^{i h \xi} - 1 \to 0$ п.н. 

      Кроме того, $E|e^{i h \xi} - 1| \leq 2 \Rightarrow$
      по теореме Лебега о мажорируемой сходимости:

      $E|e^{i h \xi} - 1| \todown{h \to 0} 0$ 
      $\Rightarrow \phi(t)$  равномерно непрерывна на $\setR$.
    \end{proof}

  \item
    Пусть $\phi(t)$ -- х.ф. с. в. $\xi$. Тогда $\phi(t) = \conj{\phi(-t)}$

    \begin{proof}
      \begin{align*}
        \phi(t) = E e^{i t \xi} = E e^{conj{-i t \xi}} = \conj{E e^{-i t \xi}} = \conj{\phi(-t)}
      \end{align*}
    \end{proof}

  \item
    Пусть $\phi(t)$ -- х.ф. с.в. $\xi$. 
    Тогда $\phi(t)$ -- действительнозначная $\Leftrightarrow$ распределение $\xi$ 
    симметрично, т.е. $\forall B \in B(\setR)$
    \begin{align*}
      P(\xi \in B) = P(\xi \in -B)
    \end{align*}

    \begin{proof}~

      $(\Leftarrow)$ Пусть распр. $\xi$ -- симметрично. 
      Тогда $\xi \equp{d} -\xi \Rightarrow$
      \begin{align*}
        &E sin(t\xi) = E sin(-t\xi) = -E sin(t\xi)\\
        &\Rightarrow E sin(t\xi) = 0 \Rightarrow \phi(t) = E e^{i t \xi} = E cos(t \xi) \in \setR
      \end{align*}
      -- действительнозначная.\\

      $(\Rightarrow)$ Пусть $\phi(t) \in \setR, \, \forall t \in \setR$.
      Тогда по свойствам \circled{2} и \circled{4}.
      \begin{align*}
        \phi(t) = \phi_\xi (t) = \overline{\phi_\xi (-t)} = \phi_{\xi} (-t) = \phi_{-\xi} (t)
      \end{align*}
      т.е. у $\xi$ и у $-\xi$  одинаковая х.ф. 
      $\Rightarrow$ по теореме о единственности функции распр. $\xi$ и $-\xi$ совпадают.

      $\Rightarrow \xi \equp{d} -\xi$ и, значит, для $\forall B \in B(\setR):$
      \begin{align*}
        P(\xi \in B) = P(-\xi \in B) = P(\xi \in -B)
      \end{align*}
    \end{proof}

    \item
      Пусть $\xi_1, \ldots, \xi_n$ -- независимые с.в., $S_n = \xi_1 + \ldots + \xi_n$
      Тогда 
      \begin{align*}
        \phi_{S_n} (t) = \prod_{k = 1} \phi_{\xi_k} (t)
      \end{align*}

      \begin{proof}
        \begin{align*}
          &\phi_{S_n} (t) = E e^{i S_n t} = E e^{i \xi_1 t} \ldots e^{i \xi_n t} 
          = \expl{с.в независимы $\Rightarrow$ $e^{\text{с.в}}$ независимы} =\\
          &= \pars{E e^{i \xi t}} \ldots \pars{E e^{i \xi_n t}} 
          = \prod_{k = 1}^{n} \phi_{\xi_k} (t)
        \end{align*}
      \end{proof}

\end{enumerate}

\begin{theorem}[о производных х.ф.]~

  Пусть $E |\xi|^n < +\infty,\; n \in \setN$. 
  Тогда для $\forall r \leq n: \exists \phi_{\xi}^{(r)}(t)$, причем

  \begin{enumerate}
    \item 
      $\phi_{\xi}^{(r)} (t) = \int\limits_{\setR} (i x)^r e^{i t x} P_{\xi} (dx)$

    \item
      $E \xi^r = \dfrac{\phi_{\xi}^{(r)} (0)}{i^r}$

    \item
      $
        \phi_{\xi} (t) 
        = \sum\limits_{k = 0}^{n} \dfrac{(i t)^k}{k!} E \xi^k + \dfrac{(i t)^n}{n!} \epsilon_n (t)
      $\\

      где $|\epsilon_n (t)| \leq 3 E |\xi|^n$ и $\epsilon_{n} (t) \to 0$, при $t \to 0$
  \end{enumerate}

\end{theorem}
  
\begin{proof}~

  \begin{enumerate}
    \item
      Заметим, что $E|\xi|^r$ конечно для $\forall r \leq n$ т.к. $|\xi|^r \leq |\xi|^n + 1$

      Рассмотрим 
      \begin{align*}
        \frac{\phi_\xi(t + h) - \phi_\xi (t)}{h} = \frac{E e^{i (t + h) \xi} - E e^{i t \xi}}{h}
        = E \pars{e^{i t \xi}\, \frac{e^{i h \xi} - 1}{h}}
      \end{align*}
      При $h \to 0, \; \dfrac{e^{i h \xi} - 1}{h} \to i \xi$ п.н., 
      кроме того $\walls{ \dfrac{e^{i h \xi} - 1}{h} } \leq |\xi|$

      $\Rightarrow$ по теореме Лебега.
      \begin{align*}
        E \pars{e^{i t \xi}\, \frac{e^{i h \xi} - 1}{h}} \todown{h \to 0} E(i \xi e^{i t \xi})
        = \int\limits_{\setR} (i x) e^{i t x} P_\xi (dx) = \phi'_{\xi} (t)
      \end{align*}

      Установление формулы для $\phi_{\xi}^{(r)}$ при $r > 1$ проводится по индукции аналогично.

    \item
      Формула $E \xi^k = \dfrac{\phi_{\xi}^{(r)} (0)}{i^r}$ сразу следует 
      из формулы для $\phi_\xi^{(r)}$

    \item
      Имеет место разложение:
      \begin{align*}
        e^{i y} = \sum_{k = 0}^{n - 1} \frac{(i t)^k}{k!} 
        + \frac{(i y)^n}{n!} (\cos \theta_1 y + i \sin \theta_2 y)
      \end{align*}
      где $|\theta_1| \leq 1, |\theta_2| \leq 1$.

      Тогда 
      \begin{align*}
        &e^{i t \xi(\omega)} = \sum_{k = 0}^{n - 1} \frac{(i t \xi)^k}{k!} 
        + \frac{(i t \xi)^n}{n!} (\cos (\theta_1(\omega) t \xi(\omega)) 
        + i \sin (\theta_1(\omega) t \xi(\omega)))\\
        &\Rightarrow \phi_{\xi} (t) = E e^{i t \xi} 
        = \sum_{k = 0}^{n - 1} \frac{(i t)^k}{k!} E \xi^k
        + \frac{(i t)^n}{n!} E (\xi^n (\cos (\theta_1 t \xi) + i \sin (\theta_1 t \xi)) =\\
        &= \sum_{k = 0}^{n} \frac{(i t)^k}{k!} E \xi^k + \frac{(i t)^n}{n!} \epsilon_n (t)
      \end{align*}
      где $\epsilon_n (t) = E (\xi^n (\cos (\theta_1 t \xi) + i \sin (\theta_1 t \xi) - 1))$
      
      Легко увидеть, что $|\epsilon_n (t)| \leq 3 E |\xi|^n$ и
      $E (\xi^n (\cos (\theta_1 t \xi) + i \sin (\theta_1 t \xi) - 1)) \to 0,\; t \to 0$

      По теореме Лебега, $\epsilon_n(t) \todown{t \to 0} 0$
  \end{enumerate}
\end{proof}

\begin{theorem}[о разложении в ряд х.ф.]~

  Пусть $\xi$ -- с.в. такова, что $E |\xi|^n < +\infty$ для $\forall n \in \setN$.

  Если для некоторго $T > 0$ выполнено
  \begin{align*}
    \limup_n \pars{ E \frac{|\xi|^n}{n!} }^{\frac{1}{n}} < \frac{1}{T},
  \end{align*}
  то для $\forall t: |t| < T$, выполнено
  \begin{align*}
    \phi_{\xi} (t) = \sum_{n = 0}^{\infty} \frac{(i t)^n}{n!}
  \end{align*}

\end{theorem}
 
\begin{proof}~
  
  Пусть $t_0$ такое, что $|t_0| < T$.
  Тогда
  \begin{align*}
    \limup_{n \to \infty} \pars{ E \frac{|\xi|^n |t_0|^n}{n!} }^{\frac{1}{n}} 
    < \frac{|t_0|}{T} < 1
  \end{align*}

  По принципу Коши ряд
  \begin{align*}
    \sum_{k = 0}^{\infty} \frac{E |\xi|^n |t_0|^n}{n!} \text{ сходится.}
  \end{align*}

  Рассмотрим $t$ т.ч. $|t| < |t_0|:$
  \begin{align*}
    \phi_{\xi} (t) = \sum_{k = 0}^{n} \frac{(i t)^k}{k!} E \xi^k + \frac{(i t)^n}{n!} \epsilon_n (t)
    &&(*)
  \end{align*}
  
  Но $|R_n(t)| \leq \dfrac{|t|^n}{n!} 3 E |\xi|^n \todown{n \to \infty} 0$

  Устремляя $n \to \infty$ в $(*)$ получаем
  \begin{align*}
    \phi_{\xi} (t) = \sum_{n = 0}^{\infty} \frac{(i t)^n}{n!} E \xi^n
  \end{align*}

  В силу произвольности $t_0$ с условием $|t_0| < T$, получаем, что разложение \\
  верно для всех $t \in (-T, T)$
\end{proof}

\begin{example}
  Пусть $\xi \sim N(0, 1)$. Тогда $\phi_{\xi} = e^{\frac{-t^2}{2}}$

  \begin{proof}
    Посчитаем моменты с.в. $\xi$.
    \begin{align*}
      E \xi^m = \int\limits_{\setR} x^m \frac{1}{\sqrt(2 \pi)} e^{\frac{-x^2}{2}} dx
    \end{align*}

    Если $m$ - нечетно, то $E \xi^m = 0$ 

    Если же $m$ - четно, то 
    \begin{align*}
      &E \xi^m = 2 \int\limits_{0}^{+\infty} x^m \frac{1}{\sqrt{2 \xi}} e^{-\frac{x^2}{2}} dx 
      = \expl{$y = \frac{x^2}{2}$} = 2 \int\limits_{0}^{+\infty} (2 y)^{m / 2} 
        \frac{1}{\sqrt{2 \pi}} e^{-y} \frac{dy}{\sqrt{2 y}}\\
      &= 2^{\frac{m}{2}} \frac{1}{\sqrt{\pi}} \int\limits_{0}^{+\infty} y^{\frac{m-1}{2}} e^{-y} dy 
      = 2^{\frac{m}{2}} \frac{1}{\sqrt{\pi}} \Gamma\pars{\frac{m + 1}{2}}
      = 2^{\frac{m}{2}} \frac{1}{\sqrt{\pi}} \frac{m - 1}{2} \cdot \frac{m - 3}{2} \cdot 
        \ldots \cdot \frac{1}{2} \cdot \Gamma\pars{\frac{1}{2}} \\
      &= 2^{\frac{m}{2}} \frac{1}{\sqrt{\pi}} \frac{(m - 1)!!}{2^{m/2}} \sqrt{\pi}
      = (m - 1)!!
    \end{align*}

    Рассмотрим 
    \begin{align*}
      &\limup_n \pars{\frac{E |\xi|^n}{n!}}^{\frac{1}{n}} 
      = \limup_n \pars{\frac{E |\xi|^{2n}}{(2n)!}}^{\frac{1}{2n}}
      = \limup_n \pars{\frac{(2n - 1)!!}{(2n)!}}^{\frac{1}{2n}}
      = \limup_n \pars{\frac{1}{(2n)!!}}^{\frac{1}{2n}}\\
      &= \limup_n \pars{\frac{1}{2^n n!}}^{\frac{1}{2n}}
      = \expl{ф-ла Стирлинга}
      = \limup_n \pars{\frac{e^n}{2^n n^n}}^{\frac{1}{2n}} = 0 < \frac{1}{T},\; \forall T
    \end{align*}

    $\Rightarrow \phi_\xi (t)$ разлагается в ряд на всей прямой.

    Осталось его посчитать
    \begin{align*}
      &\phi_\xi (t) = \sum_{k = 0}^{\infty} \frac{(i t)^k}{k!} E \xi^k
      = \sum_{m = 0}^{\infty} \frac{(i t)^{2m}}{(2m)!} E \xi^{2m}
      = \sum_{m = 0}^{\infty} \frac{(-t^2)^m}{(2m)!} (2m - 1)!!\\
      &= \sum_{m = 0}^{\infty} \frac{(-t^2)^m}{(2m)!!}
      = \sum_{m = 0}^{\infty} \frac{(-t^2)^m}{2^m m!}
      = \sum_{m = 0}^{\infty} \pars{\frac{-t^2}{2}}^m \cdot \frac{1}{m!} = e^{-t^2/2}
    \end{align*}
  \end{proof}

  \begin{corollary}
    Пусть $\xi \sim N(a, \sigma^2)$. Тогда 
    \begin{align*}
      \phi_{\xi} (t) = e^{i t a - \frac{t^2 \sigma^2}{2}}
    \end{align*}

    \begin{proof}
      Если $\xi \sim N(a, \sigma^2), $ то $\eta = \frac{\xi - a}{\sigma} \sim N(0, 1)$

      $\Rightarrow \phi_{\xi}(t) = e^{i t a} \phi_{\eta} (t \sigma) = e^{i t a - \frac{t^2 \sigma^2}{2}}$
    \end{proof}
  \end{corollary}
\end{example}

\begin{theorem}[единственности]~

  Пусть $F(x), G(x)$ -- функции распределения на прямой. 
  Если характеристические функции $F$ и $G$ совпадают, то $F = G$.

  \begin{proof}
    Пусть $a < b \in \setR$. Для $\forall \epsilon > 0$
    рассмотрим функцию $f_{\epsilon} (x):$

    Докажем, что 
    \begin{align*}
      \int\limits_{\setR} f_\epsilon (x) d F(x) = \int\limits_{\setR} f_\epsilon dG(x)
    \end{align*}

    Рассмотрим отрезок $[-n, n], \; n \in \setN$ т.ч.
    $[-n, n] \supset [a, b + \epsilon]$. 

    По теореме Вейерштрасса $f_\xi (x)$ равномерно приближается тригонометрическими многочленами
    от $\frac{x \pi}{n}$, т.е.
    \begin{align*}
      \exists f_\epsilon^n (x) = \sum_{k \in K} a_k e^{i \frac{k \pi x}{n}}, \; a_k \in \setR, 
      \text{ $K$ -- конечное подмно-во $\setZ$}
    \end{align*}
    т.ч.
    $\walls{f_\epsilon(x) - f_\epsilon^{n} (x)} \leq \frac{1}{n},\; \forall x \in [-n, n]$

    Заметим, что $f_\epsilon^n (x)$ явл. периодической с периодом $2n$

    $\Rightarrow$  т.к. $| f_\epsilon^n (x) | \leq 2$ для $\forall x \in [-n, n]$, то 
    $|f(x)| \leq 1$ и $|f_{\epsilon}^n (x)| \leq 2$, для $\forall x \in \setR$.

    По условию $\forall t \in \setR$
    \begin{align*}
      &\int\limits_{\setR} e^{i t x} dF(x) = \int\limits_{\setR} e^{i t x} dG(x)\\
      &\Rightarrow \int\limits_{\setR} f_{\epsilon}^n (x) dF(x) 
      = \int\limits_{\setR} f_{\epsilon}^n (x) dG(x)
    \end{align*}

    Теперь оценим:
    \begin{align*}
      &\walls{ \int\limits_{\setR} f_{\epsilon} (x) dF(x) 
      - \int\limits_{\setR} f_{\epsilon} (x) dG(x) } 
      \leq \walls{ \int\limits_{\setR} f_{\epsilon}^n (x) dF(x) 
      - \int\limits_{\setR} f_{\epsilon}^n (x) dG(x) } + \\
      &+ \int\limits_{\setR} (f_{\epsilon} (x) - f_{\epsilon}^n (x)) dF(x) 
      - \int\limits_{\setR} (f_{\epsilon} (x) - f_{\epsilon}^n (x)) dG(x) \leq\\
      &\leq \frac{1}{n} \int\limits_{[-n, n]} dF(x) + \frac{1}{n} \int\limits_{[-n, n]} dG(x)
      + 2 \pars{\int\limits_{\setR \setminus [-n, n]} dF(x) 
      + \int\limits_{\setR \setminus [-n, n]} dG(x)} \leq\\
      &\leq \frac{2}{n} + 2 \pars{\int\limits_{-\infty}^{-n} dF(x) + \int\limits_{n}^{+\infty} dF(x) 
      + \int\limits_{-\infty}^{-n} dG(x) + \int\limits_{n}^{+\infty} dG(x)} =\\
      &= \frac{2}{n} + 2 ( F(-n) + 1 - F(n) + G(-n) + 1 - G(n) ) \to 0, \text{ при } n \to \infty
    \end{align*}

    Отсюда получаем, что $\forall \epsilon$
    \begin{align*}
      \int\limits_{\setR} f_\epsilon (x) dF(x) = \int\limits_{\setR} f_\epsilon (x) dG(x)
    \end{align*}

    При $\epsilon \to 0, f_{\epsilon}(x) \to I_{(a, b]} (x)$
    
    При этом $|f_{\epsilon} (x)| \leq 1$ для $\forall x \in \setR \Rightarrow$ по теореме Лебега
    \begin{align*}
      \int\limits_{\setR} f_\epsilon (x) dF(x) \todown{\epsilon \to 0} 
      \int\limits_{\setR} I_{(a, b]} dF(x) = F(b) - F(a)
    \end{align*}

    Следовательно, для $\forall a < b:$
    \begin{align*}
      F(b) - F(a) = G(b) - G(a)
    \end{align*}

    Устремим $a \to -\infty, \Rightarrow \forall x \in \setR$ 
    \begin{align*}
      F(x) = G(x)
    \end{align*}
  \end{proof}
\end{theorem}

\begin{example}
  Пусть $\xi_1, \xi_2$ -- нез. с.в., $\xi_i \sim N(a_i, \sigma_{i}^2$.
  Тогда $\xi_1 + \xi_2 \sim N(a_1 + a_2, \sigma_{1}^2 + \sigma_{2}^2)$.
  \begin{proof}
    х.ф. 
    \begin{align*}
      &\phi_{\xi_j}(t) = e^{i a_j t - \frac{1}{2} \sigma_j^2 t^2}\\
      &\Rightarrow \phi_{\xi_1 + \xi_2} (t) = \expl{нез.} = \phi_{\xi_1}(t) \phi_{\xi_2} (t) 
      = e^{i(a_1 + a_2) t - \frac{1}{2} t^2 (\sigma_1^2 + \sigma_2^2)}
    \end{align*}
    -- х.ф $N(a_1 + a_2, \sigma_1^2 + \sigma_2^2)$

    По теореме о единственности $\xi_1 + \xi_2 \sim N(a_1 + a_2, \sigma_1^2 + \sigma_2^2)$
  \end{proof}
\end{example}

