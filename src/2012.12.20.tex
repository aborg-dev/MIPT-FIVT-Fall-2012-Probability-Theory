

\begin{theorem}[критерий независимости компонент случайного вектора]

  Пусть $\xi = (\xi_1, \ldots, \xi_n)$ -- случайный вектор. 
  Тогда $\xi_1 \ldots \xi_n$ -- независимы в совокупности
  $\iff$ х.ф. вектора $\xi$ распадается в произведение х.ф. $\xi_j$ :

  \begin{align*}
    \phi_{\xi} (t_1, \ldots, t_n) = \phi_{\xi_1} (t_1) \cdot \ldots \cdot \phi_{\xi_n} (t_n)
  \end{align*}

  \begin{proof}~
  
    $(\implies)$ Пусть $\xi_1 \ldots \xi_n$ -- независимы.
    Тогда 
    \begin{align*}
      &\phi_{\xi} (t_1 \ldots t_n) = E e^{i \sum_{k = 1}^{n} \xi_k t_k} 
      = E (e^{i t_1 \xi_1} \ldots e^{i t_n \xi_n})
      = \prod_{k = 1}^{n} E e^{i t_k \xi_k} = \prod_{k = 1}^{n} \phi_{\xi_k} (t_k)\\
    \end{align*}

    $(\Leftarrow)$
    Пусть $F_1, \ldots, F_n$ -- функции распределения $\xi_1, \ldots, \xi_n$.

    Рассмотрим $G(x_1, \ldots, x_n) = F_1(x_1) \cdot \ldots \cdot F_n(x_n)$
    Посчитаем её х.ф.:
    \begin{align*}
      &\int\limits_{\setR^n} e^{i \langle t, x \rangle} dG(x) 
      = \int\limits_{\setR^n} e^{i \langle t, x \rangle} dF_1(x_1) \ldots dF_n(x_n)
      = \expl{теорема Фубини} =\\
      &= \prod_{k = 1}^{n}\; \int\limits_{\setR} e^{i t_k x_k} dF_k(x_k)
      = \prod_{k = 1}^{n} \phi_{\xi_k}(t_k) = \phi_{\xi} (t_1 \ldots t_n)
    \end{align*}
    
    Но $\phi$ -- х.ф. вектора $\xi$.
    $\implies$ она является х.ф. ф.р. $F_{\xi} (x_1 \ldots x_n).$ 

    По теореме о единственности 
    \begin{align*}
      F_{\xi} (x_1 \ldots x_n) = G(x_1 \ldots x_n) = F_1(x_1) \ldots F_n(x_n)
    \end{align*}

    По критерию независимости для ф.р. получаем, 
    что $\xi_1 \ldots \xi_n$ независимы в совокупности.\\
  \end{proof}
  
\end{theorem}

\begin{theorem}[формула обращения]~

  Пусть $\phi(t)$ -- х.ф. ф.р. $F(x)$
  Тогда
  \begin{enumerate}
    \item
      Для $\forall a < b, \; a, b \in \setC(F)$ -- точки непрерывности $F(x)$, выполнено:
      \begin{align*}
        F(b) - F(a) = \frac{1}{2 \pi} \lim_{c \to +\infty} 
        \int\limits_{-c}^{c} \frac{e^{- i t b} - e^{- i t a}}{it} \phi(t) dt
      \end{align*}

    \item
      Если $\int\limits_{\setR} |\phi(t)| dt < +\infty$, 
      то y $F(x) \; \exists$ плотность $f(x)$ и
      \begin{align*}
        f(x) = \frac{1}{2\pi} \int\limits_{\setR} e^{-i t x} \phi(t) dt
      \end{align*}
  \end{enumerate}

\end{theorem}

\begin{example}
  Пусть $\xi$ имеет распр. Коши
  \begin{align*}
    p_{\xi} (x) = \frac{1}{\pi (1 + x^2)}
  \end{align*}
  Найти х.ф. $\xi$.

  \begin{proof}
    Пусть $\eta$ имеет распр. Лапласа, $p_{\eta} (x) = \frac{1}{2} e^{-|x|}$

    Тогда $\phi_{\eta} (t) = \frac{1}{1 + t^2}$, и
    $\int\limits_{\setR} |\phi(t)| dt < +\infty$

    $\implies$ по формуле обращения 
    \begin{align*}
      &p_{\eta}(x) = \frac{1}{2 \pi} \int\limits_{\setR} e^{- i t x} \phi(t) dt 
      = \frac{1}{2} \int\limits_{\setR} e^{- i t x} \frac{1}{\pi (1 + t^2)} dt
      = \frac{1}{2} \phi_\xi (-x)\\
      &\implies  \phi_{\xi} (t) = e^{-|t|}
    \end{align*}
  \end{proof}
\end{example}

\bigtitle{Как понять, является ли функция характеристической?}

\begin{definition}
  Функция $(\phi(t), t \in \setR)$ наз. \emph{неотрицательно определенной},
  если $\forall t_1 \ldots t_n \in \setR \; z_1 \ldots z_n \in \setC$ выполнено:
  \begin{align*}
    \sum_{i, j = 1}^{n} \phi(t_i - t_j) z_i \conj{z_j} \geq 0
  \end{align*}
\end{definition}

\begin{theorem}[Бонхер - Хинчин]~

  Пусть $\phi(t), t \in \setR$ -- непрерывна в нуле и $\phi(0) = 1$. 
  Тогда $\phi(t)$ явл. хар. функцией $\iff \phi(t)$ неотрицательно определена.

  \begin{proof}
    $(\implies)$ Пусть $\phi(t)$ -- х.ф. с.в. $\xi$.
    Тогда $\forall t_1 \ldots t_n \in \setR, \; \forall z_1 \ldots z_n \in \setC$

    \begin{align*}
      &\sum_{j, k = 1}^{n} \phi(t_j - t_k) z_j \conj{z_k} 
      = \sum_{j, k = 1}^{n} E e^{i (t_j - t_k) \xi} z_j \conj{z_k} 
      = E \pars{\sum_{j, k = 1}^{n} e^{i t_j \xi} z_j e^{-i t_k \xi} \conj{z_k}} = \\
      &= E \pars{\sum_{j, k = 1}^{n} (e^{i t_j \xi} z_j) \conj{(e^{i t_k \xi} z_k)}}
      = E \pars{\sum_{j}^{n} (e^{i t_j \xi} z_j)} 
      \cdot \conj{\pars{\sum_{k = 1}^{n} e^{i t_k \xi} z_k)}}
      = E \walls{\sum_{j = 1}^{n} e^{i t_j \xi} z_j }^2 \geq 0
    \end{align*}
  \end{proof}
\end{theorem}

\begin{corollary}
  Если $\phi(t)$  и $\psi(t)$ -- две х.ф., 
  то $\forall \alpha \in (0, 1):$ 
  \begin{align*}
    \alpha \phi(t) + (1 - \alpha) \psi(t) \text{ -- тоже х.ф.}
  \end{align*}
\end{corollary}

\begin{theorem}[непрерывности]~

  Пусть $\{ F_n(x),\; n \in \setN \}$ -- последовательность ф.р. на $\setR$, 
  а $\{ \phi_n(t),\; n \in \setN \}$ -- последовательность их х.ф.

  Тогда
  \begin{enumerate}
    \item
      Если $F_n \toup{w} F$, где $F(x)$ -- ф.р. на $\setR$, 
      то для $\forall t \in \setR: \phi_n(t) \to \phi(t)$ при $n \to \infty$, 
      где $\phi(t)$ - х.ф. $F(x)$

    \item
      Пусть для $\forall t \in \setR \quad \exists$ предел $\lim\limits_{n \to \infty} \phi_n (t)$,
      причем $\phi(t) = \lim\limits_{n \to \infty} \phi_n(t)$ непрерывна в нуле. 
      Тогда $\exists$ ф.р. $F(x)$ т.ч. $F_n \toup{w} F$ и $\phi(t)$ - х.ф. $F(x)$
  \end{enumerate}

  \begin{proof}
    \begin{enumerate}
      \item
        Если $F_n \toup{w} F$, то $\forall f(x)$ -- огр. непр. выполнено:
        \begin{align*}
          \int\limits_{\setR} f(x) dF_n(x) \todown{n \to \infty} \int\limits_{\setR} f(x) dF(x)
        \end{align*}

        Функции $\cos{tx}$ и $\sin{tx}$ -- огр. и непр., тогда 
        \begin{align*}
          &\phi_n (t) = \int\limits_{\setR} e^{i t x} dF_n(x) 
          = \int\limits_{\setR} \cos{tx}\, dF_n(x) + i \int\limits_{\setR} \sin{tx}\, dF_n(x)
          \todown{n \to \infty}\\
          &\todown{n \to \infty} \int\limits_{\setR} \cos{tx}\, dF(x) 
          + i \int\limits_{\setR} \sin{tx}\, dF(x) = \phi(t)
        \end{align*}
    \end{enumerate}
  \end{proof}
\end{theorem}

\begin{corollary}
  С.в. $\xi_n \toup{d} \xi \iff \forall t \in \setR:
  \phi_{\xi_n} (t) \to \phi_\xi (t)$

  \begin{proof}
    $\xi_n \toup{d} \xi \iff F_{\xi_n} \toup{w} F_{\xi} \iff 
    \phi_{\xi_n} (t) \to \phi_\xi (t)$ для $\forall t \in \setR$.
  \end{proof}
\end{corollary}

\begin{theorem}[Центральная предельная теорема]~

  Пусть $\{ \xi_n,\; n \in \setN \}$ -- 
  последовательность независимых одинаково распределенных с.в. т.ч. 
  $0 < D \xi_n < +\infty$. 

  Обозначим $S_n = \xi_1 + \ldots + \xi_n$
  Тогда
  \begin{align*}
    \frac{S_n - E S_n}{\sqrt{D S_n}} \toup{d} N(0, 1)
  \end{align*}

  \begin{proof}~

    Обозначим $a = E \xi_i, \sigma^2 = D \xi_i$. 
    Рассмотрим $\eta_i = \frac{\xi_i - a}{\sigma}
    \implies E \eta_i = 0, D \eta_i = E \eta_i^2 = 1$

    Тогда
    \begin{align*}
      T_n = \frac{S_n - E S_n}{\sqrt{D S_n}} = \expl{независимость}
      = \frac{S_n - na}{\sqrt{n} \sigma} = \frac{\eta_1 + \ldots \eta_n}{\sqrt{n}}
    \end{align*}

    Рассмотрим х.ф. $\eta_i:$
    \begin{align*}
      \phi_{\eta_i} (t) = \phi(t) = 1 + E \eta_i (i t) + 
      \frac{1}{2} E \eta_i^2 (it)^2 + \underset{(t \to 0)}{o(t^2)};
    \end{align*} 

    Отсюда получаем, что 
    \begin{align*}
      &\phi_{T_n} (t) = \phi_{\eta_1 + \ldots + \eta_n} (\frac{t}{\sqrt{n}}) 
      = \expl{независимость} = \pars{\phi\pars{\frac{t}{\sqrt{n}}}}^n 
      = \pars{1 - \frac{t^2}{2n} + o\pars{\frac{t^2}{n}}}^n 
      \todown{n \to \infty} e^{- \frac{t^2}{2}}
    \end{align*}
    Но $e^{-\frac{t^2}{2}}$ -- х.ф. $N(0, 1)$ 
    $\implies$ по теорема непрерывности мы получаем, что 
    \begin{align*}
      T_n = \frac{S_n - E S_n}{\sqrt{D S_n}} \toup{d} N(0, 1)
    \end{align*}
  \end{proof}
\end{theorem}

\begin{corollary}
  В условиях ЦПТ для $\forall x \in \setR$ выполнено
  
  \begin{align*}
    P\pars{\frac{S_n - E S_n}{\sqrt{D S_n}} \leq x} 
    \todown{n \to \infty} \int\limits_{-\infty}^{x} \frac{1}{\sqrt{2\pi}} e^{-\frac{y^2}{2}} dy
  \end{align*}

  \begin{proof}
    По ЦПТ $T_n = \frac{S_n - E S_n}{\sqrt{D S_n}} \toup{d} \xi \sim N(0, 1) 
    \iff F_{T_n} \implies F_{\xi}$, где $F_{\xi} (x)$  -- ф.р. $N(0, 1)$, т.е.
    $\forall x \in \setR:$
    \begin{align*}
      F_{T_n} \todown{n \to \infty} F_{\xi} (x) 
      = \int\limits_{-\infty}{x} \frac{1}{\sqrt{2 \pi}} e^{-\frac{y^2}{2}} dy
    \end{align*}
  \end{proof}
\end{corollary}

\begin{corollary}
  В условиях ЦПТ, если $E \xi_i = a, D \xi_i = \sigma^2$, то
  \begin{align*}
    \sqrt{n} \pars{\frac{S_n}{n} - a} \toup{d} N(0, \sigma^2)
  \end{align*}

  \begin{proof}
    \begin{align*}
      \sigma T_n = \sigma \frac{S_n - E S_n}{\sqrt{D S_n}} 
      = \sigma \frac{S_n - na}{\sqrt{n} \sigma} = \sqrt{n} \pars{\frac{S_n}{n} - a}
    \end{align*}

    Но $T_n \toup{d} N(0, 1) \implies \sigma T_n \toup{d} \sigma N(0, 1) = N(0, \sigma^2)$

    $\implies \sqrt{n} \pars{\frac{S_n}{n} - a} \toup{d} N(0, \sigma^2)$
  \end{proof}
\end{corollary}

\begin{theorem}[Теорема Берри - Эссен]~

  Пусть $\{ \xi_n,\; n \in \setN \}$ -- нез. с.в., $E |\xi_i|^3 < +\infty$,
  
  $E \xi_i = a,\; D \xi_i = \sigma^2 > 0$. 

  Обозначим $S_n = \xi_1 + \ldots + \xi_n, \; T_n = \frac{S_n - E S_n}{\sqrt{D S_n}}$.

  Тогда
  \begin{align*}
    \sup\limits_{x \in \setR} |F_{T_n} (x) - \Phi(x)| 
    \leq C \frac{E |\xi_1 - a|^3}{\sigma^3 \sqrt{n}}
  \end{align*}
  где $C$ -- абс. константа. Вместо $\xi_1$ можно взять любую из $\xi_1 \ldots \xi_n$.
\end{theorem}

Что можно сказать про $C$?
\begin{enumerate}
  \item 
    $C \geq \frac{1}{\sqrt{2 \pi}} \approx 0,399$ (Эссен)

  \item
    Текущий рекорд $\forall n \forall \xi: C \leq 0.48$
\end{enumerate}

\begin{example}
  Складываются $10^4$ чисел, каждое из которых было вычислено с точностью $10^{-6}$.
  Найти в каких пределах с вероятностью $0.99$ лежит суммарная ошибка, считая,
  что все ошибки независимы и распределены $R(-10^{-6}, 10^{-6})$

  \begin{proof}
    $\xi_i \sim R(-10^{-6}, 10^{-6})$ -- нез. с. в.

    $E \xi_i = a = 0, \; D \xi_i = \sigma^2 = 10^{-12} \frac{2}{3},\; 
    S_n = \xi_1 + \ldots + \xi_n$.

    Согласно ЦПТ:
    \begin{align*}
      P\pars{ \walls{ \frac{S_n - E S_n}{\sqrt{D S_n}} } \leq u } 
      \sim P(|\eta| \leq u), \text{где } \eta \sim N(0, 1)
    \end{align*}

    Из таблицы значений $\Phi(x) 
    = \int\limits_{-\infty}{x} \frac{1}{\sqrt{2 \pi}} e^{-\frac{y^2}{2}} dy$

    Получаем, что при $u = 2.58$
    \begin{align*}
      &P(|\eta| \leq u) \geq 0.99\\
      &\implies P\pars{|S_n| \leq 2.58 \sqrt{D S_n}} \geq 0.99\\
      &P\pars{|S_n| \leq 2.58 \sqrt{\frac{2}{3}} \cdot 10^{-6}} \geq 0.99
    \end{align*}

    Суммарная ошибка: $2.58 \sqrt{\frac{2}{3}} \cdot 10^{-6}$
  \end{proof}

\end{example}

