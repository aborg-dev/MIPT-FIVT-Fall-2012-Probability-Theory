

\begin{theorem}[Фубини]~

  Пусть $(\Omega, \setF, P)$ -- прямое произведение 
  $(\Omega, \setF_1, P_1)$ и $(\Omega, \setF_2, P_2)$.

  Пусть с.в $\xi\colon \Omega \to \setR$ т.ч.
  $\int\limits_{\Omega} |\xi(\omega_1, \omega_2)| dP < + \infty$

  Тогда интегралы
  $\int\limits_{\Omega_1} \xi(\omega_1, \omega_2) dP_1$ и 
  $\int\limits_{\Omega_2} \xi(\omega_1, \omega_2) dP_2$ определены почти наверное
  относительно $P_2$ и $P_1$, являются измеримыми отностительно $\setF_2, \setF_1$ соотв., 
  и кроме того,
  \begin{align*}
    \int\limits_{\Omega} \xi(\omega_1, \omega_2) dP 
    = \int\limits_{\Omega_1} \int\limits_{\Omega_2} \xi(\omega_1, \omega_2) dP_2 dP_1 
    = \int\limits_{\Omega_2} \int\limits_{\Omega_1} \xi(\omega_1, \omega_2) dP_1 dP_2
  \end{align*}

  Смысл теоремы: Двойной интеграл = повторному интегралу

\end{theorem}

\begin{statement}
  Пусть $\xi$, $\eta$ -- независ. с.в. 

  Тогда $(\setR^2, B(\setR^2), P_{(\xi, \eta)})$ явл. прямым произведением 
  $(\setR^2, B(\setR^2), P_{\xi})$ и $(\setR^2, B(\setR^2), P_{\eta})$

  \begin{proof}
    \begin{align*}
      &\setR^2 = \setR \times \setR\\
      &B(\setR^2) = B(\setR) \otimes B(\setR)\\
      &P_{(\xi, \eta)}(B_1 \times B_2) = P_{\xi} (B_1) \cdot P_{\eta}(B_2) ?
    \end{align*}

    Действительно,
    \begin{align*}
      &P_{(\xi, \eta)}(B_1 \times B_2) = P((\xi, \eta) \in B_1 \times B_2) 
      = P(\xi \in B_1, \eta \in B_2) = \expl{независимость} =\\
      &= P(\xi \in B_1) \cdot P(\eta \in B_2) = P_{\xi}(B_1) \cdot P_{\eta}(B_2).
    \end{align*}
  \end{proof}
\end{statement}

\begin{lemma}[О свертке распределений]~

  Пусть $\xi, \eta$ -- нез. с.в. с ф.р. $F_\xi$ и $F_\eta$.

  Тогда:
  \begin{enumerate}
    \item 
      \begin{align*}
        F_{\xi + \eta} (z) = \int\limits_{\setR} F_{\xi} (z - x) dF_{\eta} (x) 
        = \int\limits_{\setR} F_{\eta} (z - x) dF_{\eta}(x)
      \end{align*}

    \item
      Если $\xi$ имеет плотность $f_{\xi} (x),\quad \eta$ --  плотность $f_{\eta}(x)$, 
      то $\xi + \eta$ имеет плотность
      \begin{align*}
        f_{\xi + \eta} (z) = \int\limits_{\setR} f_{\xi} (z - x) f_{\eta}(x) dx 
        = \int\limits_{\setR} f_{\eta} (z - x) f_{\xi} (x) dx
      \end{align*}

  \end{enumerate}

  \begin{proof}~

    \begin{enumerate}
      \item
        \begin{align*}
          &F_{\xi + \eta} (z) = P(\xi + \eta \leq z) = E I \{ \xi + \eta \leq z \} 
          = \expl{ф-ла замены переменных} =\\
          &= \int\limits_{\setR^2} I \{ x + y \leq z \} P_{(\xi, \eta)}(dx, dy) 
          = \int\limits_{\setR^2} I \{ x + y \leq z \} P_{\xi} (dx) P_{\eta} (dy)
          = \expl{теор. Фубини} =\\
          &= \int\limits_{\setR} 
            \pars{\int\limits_{\setR} I \{ x + y \leq z \} P_{\xi}(dx)} P_{\eta} (dy)
          = \int\limits_{\setR} P(\xi + y \leq z) P_{\eta} (dy) 
          = \int\limits_{\setR} F_{\eta} (z - y) dF_\eta (dy) 
        \end{align*}

      \item
        \begin{align*}
          &F_{\xi + \eta}(z) 
          = \int\limits_{\setR^2} I \{ x + y \leq z \} P_\xi (dx) P_\eta (dy) 
          = \int\limits_{\setR^2} I \{ x + y \leq z \} f_\xi (x) f_\eta (y) dx dy =\\
          &= \expl{$t = x + y, x' = x$}
          = \int\limits_{\setR^2} I \{ t \leq z \} f_\xi (x') f_\eta (t - x') dx' dt
          = \expl{теорема Фубини} =\\
          &= \int\limits_{-\infty}^{z} \pars{\int\limits_{\setR} f_\xi(x') f_\eta(t - x')dx'}dt
          = \int\limits_{-\infty}^{z} f_{\xi + \eta}(t) dt
        \end{align*}
    \end{enumerate}

  \end{proof}
\end{lemma}

Замечание:

Если $\xi_1 \ldots \xi_n$ -- незав. с.в., 
то $P_{(\xi_1, \ldots, \xi_n)} = P_{\xi_1} \otimes \ldots \otimes P_{\xi_n},$

$dF_{\xi_1, \ldots, \xi_n}(x_1 \ldots x_n) = dF_{\xi_1}(x_1) \ldots dF_{\xi_n}(x_n)$ 

и если $\xi_i$ имеет плотность $f_{\xi_i}(x_i)$, 
то вектор $\xi = (\xi_1, \ldots, \xi_n)$ тоже имеет плотность
\begin{align*}
  f_{\xi} (x_1 \ldots x_n) = f_{\xi_1}(x_1) \cdot \ldots \cdot f_{\xi_n} (x_n) 
  = \frac{\partial^n}{\partial{x_1} \ldots \partial{x_n}} F_{\xi} (x_1 \ldots x_n)\\
\end{align*}

\mysection{Слабая сходимость вероятностных мер}

\begin{definition}
  Последовательность $\{ F_n (x),\; n \in \setN \}$ функций распределения 
  на $\setR$ назыв. \emph{слабо сходящейся} к функции распределения $F(x)$, если 
  $\forall f(x)$ -- огр. непрер. функции на $\setR$
  \begin{align*}
    \int\limits_{\setR} f(x) dF_n (x) \todown{n \to \infty} \int\limits_{\setR} f(x) dF(x)
  \end{align*}

  \begin{designation}
    $F_n \toup{w} F$
  \end{designation}
\end{definition}

\begin{corollary}
  С.в. $\xi_n \toup{d} \xi \Leftrightarrow F_{\xi_n} \toup{w} F_{\xi}$

  \begin{proof}
    \begin{align*}
      E f(\xi_n) = \expl{замена переменной} 
      = \int\limits_{\setR} f(x) dF_{\xi_n} (x) \todown{n \to \infty} E f(\xi) 
      = \int\limits_{\setR} f(x) dF_{\xi}(x)
    \end{align*}
  \end{proof}

\end{corollary}

\begin{definition}
  Последовательность $\{ F_n(x),\; n \in \setN  \}$ --
  функций распределения на $\setR$ называется \emph{сходящейся в основном}
  к функции распределения $F(x)$, если
  $\forall x \in \setC(F):$
  \begin{align*}
    F_n(x) \todown{n \to \infty} F(x)
  \end{align*}
  где $\setC(F)$ -- множество точек непр. функции $F(x)$

  \begin{designation}
    $F_n \Rightarrow F$
  \end{designation}
\end{definition}

Пусть $\{ P_n,\; n \in \setN \}$, $P$ -- вероятностная мера в $(\setR^m, B(\setR^m))$

\begin{definition}
  Последовательность $P_n$ наз. \emph{слабо сходящейся} к вер. мере $P$, 
  если $\forall f(x)$ -- огранич. непр. ф-ии в $\setR^m$ выполнено:
  \begin{align*}
    \int\limits_{\setR^m} f(x) P_n(dx) \todown{n \to \infty} \int\limits_{\setR^m} f(x) P(dx)
  \end{align*}

  \begin{designation}
    $P_n \toup{w} P$
  \end{designation}

\end{definition}

\begin{corollary}
  С.в. $\xi_n \toup{d} \xi \Leftrightarrow 
  F_{\xi_n} \toup{w} F_\xi \Leftrightarrow 
  P_{\xi_n} \toup{w} P_\xi$
\end{corollary}

\begin{definition}
  Последовательность $P_n$ сходится к вер. мере $P$ в основном, 
  если для $\forall A \in B(\setR^m)$  с условием $P(\partial{A}) = 0$ выполнено:
  \begin{align*}
    P_n(A) \todown{n \to \infty} P(A)
  \end{align*}
\end{definition}

\begin{designation}
  $P_n \Rightarrow P$
\end{designation}

\begin{theorem}[Александров]~

  Для вер. мер в $\setR^m$ следующие условия эквивалентны

  \begin{enumerate}
    \item $P_n \toup{w} P$
    \item $\limup_n P_n (A) \leq P(A), \quad \forall \text{ замкнутого } A$
    \item $\limdown_n P_n (A) \geq P(A), \quad \forall \text{ открытого } A$
    \item $P_n \Rightarrow P$
  \end{enumerate}
\end{theorem}

\begin{theorem}[Эквивательность пределений сходимости]~

  Пусть $\{ P_n,\; n \in \setN \}, P$ -- вероятностные меры на $\setR$,
  $\{ F_n(x),\; n \in \setN \}, F(x)$ -- соответств. им функции распределения. 

  Тогда следующие условия эквивалентны:
  \begin{enumerate}
    \item $P_n \toup{w} P$
    \item $P_n \Rightarrow P$
    \item $F_n \toup{w} P$
    \item $F_n \Rightarrow F$
  \end{enumerate}

  \begin{proof}
    По теореме Александрова достаточно проверить, что $(2)$ эквивалентно $(4)$.

    $(2) \Rightarrow (4):$ 

    Пусть $x \in \setC(F)$

    Тогда $\partial((-\infty; x]) = \{ x \}$.

    Значит,
    \begin{align*}
      F_n(x) = P_n((-\infty; x]) \todown{P_n \to P} P((-\infty; x]) = F(x)
    \end{align*}

    $(4) \Rightarrow (2):$ 

    Для установления $(2)$ по теореме Александрова достаточно проверить, 
    что $\limdown_n P_n(A) \geq P(A), \forall A$ -- откр. из $\setR$

    Пусть $A \subset \setR$ -- открыто, 
    тогда $A = \bigsqcup\limits_{k = 1}^{\infty} I_k$, 
    где $I_k = (a_k, b_k)$ -- непересек. интервалы.

    Для $\forall \varepsilon > 0$
    выберем $I_{k}' = (a_{k}', b_{k}'] \subset I_{k}$,
    т.ч. $a_k', b_k'$ -- точки непрерывности $F(x)$ и
    \begin{align*}
      P(I_k') \geq P(I_k) - \frac{\epsilon}{2^k}
    \end{align*}

    Такой выбор $(a_k', b_k']$ возьмем в силу непр. вер. меры и того факта, 
    что $F(x)$ имеет не более чем счетное  число точек разрыва. 
    Тогда
    \begin{align*}
      &\limdown_n P_n(A) = \limdown_n \sum_{k = 1}^{\infty} P_n (I_k) \geq |\forall N|
      \geq \limdown_n \sum_{k = 1}^{N} P_n(I_k) \geq \sum_{k = 1}^{N} \limdown_n P_n(I_k)
    \end{align*}

    Устремим $N \to \infty:$
    \begin{align*}
      &\limdown_n P_n(A) \geq \sum_{k = 1}^{\infty} \limdown_n P_n(I_k) 
      \geq \sum_{k = 1}^{\infty} \limdown_n P_n(I_k') \;
      \circled{=} 
    \end{align*}
    Но $P_n(I_k') = P((a_k', b_k'])) = F_n(b_k') - F_n(a_k')
    \todown{n \to \infty} F(b_k') - F(a_k')$, 
    так как $a_k', b_k'$ -- точки непр. $F(x)$. Значит $F_n \Rightarrow F$
    \begin{align*}
      \circled{=} \sum_{k = 1}^{\infty} (F(b_k') - F(a_k')) 
      = \sum_{k = 1}^{\infty} P((a_k', b_k']) 
      \geq \sum_{k = 1}^{\infty} (P(I_k) - \frac{\epsilon}{2^k}) = P(A) - \epsilon
    \end{align*}

    В силу произвольности $\epsilon > 0$, $\limdown_n P_n(A) \geq P(A)$

  \end{proof}

\end{theorem}

\begin{corollary}
  Пусть $\{ \xi_n,\; n \in \setN \}$, $\xi$ -- с.в. 
  Тогда $\xi_n \toup{d} \xi$
  $\Leftrightarrow$ $F_{\xi_n}(x) \todown{n \to \infty} F_{\xi} (x)$ 
  для $\forall x \in \setC (F_\xi)$\\
\end{corollary}

\bigtitle{Смысл сходимости по распределению:}
 
Это апроксимация распределений.

Пусть $\eta$ -- нек. с.в. со "сложным" распр. (сложно вычислить ф.р. $\eta$). 

Пусть $\xi_n \toup{d} \xi$, где распр. $\xi$ "легко" вычислить(или оно известно). 

Если $\xi_m \equp{d} \eta$ для достаточно большого номера $m$, 
то ф.р. $\eta$ можно апроксимировать ф.р. $\xi$.\\

\mysection{Предельные теоремы для схемы Бернулли}

Описание модели: проводим большое число независимых однородных случ. экспериментов, в которых мы фиксируем "успех" или "неудачу".

Нас интересует распределение числа успехов при проведении большого числа экспериментов.

Математическая модель:

$\{ \xi_n,\; n \in \setN \}$ -- нез. с.в.
$P(\xi_n = 1) = p,\; P(\xi_n = 0) = 1 - p = q$

\begin{definition}
  Распределение $\xi_n$ наз. распр. Бернулли.

  Обозначим $S_n = \xi_1 + \ldots + \xi_n$ -- число "успехов" после проведения $n$ испытаний.
\end{definition}

\begin{theorem}[Бернулли, 1703, ЗБЧ]
  $\dfrac{S_n}{n} \toup{p} p$
\end{theorem}

Несмотря на то, что распр. $S_n$ известно, практическое вычисление вероятностей вида 
$P(a \leq S_n \leq b)$ при очень больших $n$ затруднительно.

\begin{theorem}[Пуассон]~

  Если $n p(n) \to \lambda > 0$, то $\forall k \in \setZ_+$
  \begin{align*}
    P(S_n = k) \todown{n \to \infty} \frac{\lambda^k e^{-\lambda}}{k!}
  \end{align*}

\end{theorem}

\begin{proof}
  \begin{align*}
    &P(S_n = k) = \comb{n}{k} p^k (1 - p)^{n - k} 
    = \frac{1}{k!} (np)^k \frac{n (n - 1) \ldots (n - k + 1)}{n^k} \, (1 - p)^n (1 - p)^{-k}\\
    &= \frac{1}{k!} (\lambda + o(1))^k e^{-\lambda} 
      \todown{n \to \infty} \frac{1}{k!} \lambda^k e^{-\lambda}
  \end{align*}
\end{proof}

\begin{corollary}
  Если $\xi_n \sim Bin(n, p(n))$, где $n p(n) \to \lambda > 0$, 
  то $\xi_n \toup{d} \eta \sim Pois(\lambda)$

  \begin{proof}
    $\xi_n \toup{d} \eta \Leftrightarrow 
    \forall x \in \setC(F_{\eta}): F_{\xi_n}(x) \to F_{\eta}(x)$
    Но $\xi_n$ и $\eta$ принимает значения $\setZ_+ 
    \Rightarrow \forall x \in \setR \setminus \setZ_+:$

    \begin{align*}
      F_{\xi_n} (x) = \sum_{\substack{k \leq x\\k \in \setZ_+}} 
        P(\xi_n = k) \to \expl{по теор. Пуассона} \to
        \sum_{\substack{k \leq x\\k \in \setZ_+}} P(\eta = k) = F_{\eta}(x)
    \end{align*}

  \end{proof}
\end{corollary}

\begin{theorem}[Муавр-Лаплас]~

  Пусть $p = const$, $S_n \sim Bin(n, p)$. 
  Обозначим для $\forall -\infty \leq a \leq b \leq +\infty$

  \begin{align*}
    P_n (a, b) = P\pars{a \leq \frac{S_n - np}{\sqrt{npq}} \leq b}
  \end{align*}

  Тогда имеет место сходимость:
  \begin{align*}
    \sup_{-\infty \leq a \leq b \leq +\infty} 
      \walls{P_n(a, b) - \int\limits_{a}^{b} \frac{1}{\sqrt{2\pi}} e^{-\frac{x^2}{2}} dx} 
      \todown{n \to \infty} 0
  \end{align*}
\end{theorem}

\begin{corollary}
  В условиях теоремы Муавра-Лапласа
  \begin{align*}
    \frac{S_n - np}{\sqrt{npq}} \toup{d} \eta \sim N(0, 1)
  \end{align*}

  \begin{proof}
    Обозначим $\xi_n := \frac{S_n - np}{\sqrt{npq}}$

    Тогда $\xi_n \toup{d} \eta \sim N(0, 1) \Leftrightarrow \forall x \in \setR$
    \begin{align*}
      F_{\xi_n} (x) \todown{n \to \infty} F_{\eta} (x) 
      = \int\limits_{-\infty}^{x} \frac{1}{\sqrt{2\pi}} e^{-\frac{y^2}{2}} dy
    \end{align*}

    Но теорема Муавра-Лапласса именно это и утверждает
  \end{proof}
\end{corollary}

