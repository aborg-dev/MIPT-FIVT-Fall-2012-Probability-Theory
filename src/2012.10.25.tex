
\begin{definition}
  Пусть $\xi$ -- случайная величина. 

  Тогда обозначим:
  $\xi^+ = \max(\xi, 0)$  и $\xi^- = \max(-\xi, 0)$

  Ясно, что $\xi = \xi^+ - \xi^-,\;  |\xi| = \xi^+ + \xi^-$
\end{definition}

\begin{theorem}[о приближении простыми]~

  Пусть $\xi$ -- случайная величина. Тогда
  \begin{enumerate}
    \item 
      Если $\xi \geq 0, $ то $\exists$ последовательность $\{ \xi_n, n \in \setN \}$ простых неотрицательных случайных величин, т.ч $\xi_n \uparrow \xi$
      (т.е. $\forall \omega \in \Omega : \xi_n(\omega) \leq \xi_{n+1}(\omega)$
      и $\xi(\omega) = \lim\limits_{n \to \infty} \xi_n(\omega)$) 
      и $\xi_n$ явл. $\setF_\xi$ - измеримыми.

    \item 
      Если $\xi$ -- произвольная случайная величина, то $\exists$ последовательность 
      $\{ \xi_n, n \in \setN \}$ простых $\setF_\xi$ - измеримых случайных величин т.ч.  
      $|\xi_n| \leq |\xi| \; \forall n $ и $\xi_n(\omega) \to \xi(\omega)$
  \end{enumerate}
\end{theorem}

\begin{proof}~
  \begin{enumerate}
    \item 
      Положим
      \begin{align*}
        \xi_n(\omega) = \sum_{k = 1}^{n 2^n} \frac{k - 1}{2^n} \;
        I \braces{ \frac{k - 1}{2^n} \leq \xi(\omega) < \frac{k}{2^n} } + n I\{\xi(\omega) \geq n\}
      \end{align*}
      Легко видеть, что $\xi_n \uparrow \xi$ и $\xi_n$ является $\setF_\xi$ - измеримым
      (т.к. $\braces{\frac{k - 1}{2^n} \leq \xi < \frac{k}{2^n}} \in \setF_\xi$)

	by А.Д.: не понял примера. Мой(возможно похож/такой же): 
	
	$\xi_n = \phi_{n}(\xi)$, где $[\phi_{n}(x)] = min(n,[x])$, \{$\phi_{n}(x)$\} = максимальное число, не превосходящее \{$x$\}, представимое в виде $\frac{k}{2^n}$. \\
	Очевидно, что $\xi_n \leq \xi$. \\
	Ясно, что $\xi_n \uparrow \xi$, ибо целая часть в какой-то момент совпадет, а дробные части - двоично-рациональные приближения. \\
	Покажем, что функции $\phi_n$ - борелевские, тогда $\xi_n$ - $\setF_\xi$-измеримы.
	Найдем прообраз $B \in B(\setR)$. Это тоже самое, что прообраз двоично-рациональных точек из $B$. В каждую двоично-рациональную дробь $a$ переходит не более чем счетное кол-во полуинтервалов вида [$a + k$, $a + k + \frac{1}{2^n}$), где $k$ - какое-то целое.(Бесконечно, только если [$a$] = $n$, иначе не более 1 полуинтервала). Но тогда искомый прообраз - счетное объединение полуинтервалов, а значит лежит в $B(\setR)$

    \item 
      Пусть $\xi = \xi^+ - \xi^-$ и пусть $\{\eta_n,\; n \in \setN \}$ -- последовательность
      простых $\setF_\xi$ - измеримых с.в. т.ч. $\eta_n \uparrow \xi^+$, а
      $\{ \zeta_n,\; n \in \setN \}$ -- последовательность простых $\setF_\xi$ - измеримых
      т.ч. $\zeta_n \uparrow \xi^{-}$

      Положим $\xi_n = \eta_n - \zeta_n$.\\
      Тогда $\xi_n \to \xi \quad \forall \omega \in \Omega$ и 
      $|\xi_n| = |\eta_n| + |\zeta_n| \leq |\xi^+| + |\xi^-| = |\xi|$
   \end{enumerate}
\end{proof}

\bigtitle{Математическое ожидание случайных величин}

Пусть $(\Omega, \setF, P)$ -- вероятностное пространство, $\xi$ - случайная величина на нем. 
Что такое $E \xi$?

Простые случайные величины.

Пусть $\xi$ -- простая случайная величина, т.е. 
\begin{align*}
  \xi = \sum_{k = 1}^{n} x_k I_{A_k},
\end{align*}
где $x_1 \ldots x_n$ -- различные числа, $A_1, \ldots, A_n$ 
-- разбиение $\Omega$, т.е. $A_k = \{ \xi = x_k \}$

\begin{definition}
  Для простой случайной величины $\xi$ её математическим ожиданием называют
  \begin{align*}
    E\xi = \sum_{k = 1}^n x_k P(A_k)
  \end{align*}
\end{definition}

\bigtitle{Свойства математического ожидания для простых случайных величин}

\begin{enumerate}
  \item $\xi = c = const \implies E\xi = c$
  \item 
    Линейность
    \begin{align*}
      E(a\xi + b\eta) = a E \xi + b E \eta, \quad a, b \in \setR
    \end{align*}

    \begin{proof}
      Обозначим $\zeta = a\xi + b\eta$, пусть $\xi$ принимает значения $x_1 \ldots x_n$, 
      $\eta$ -- значения $y_1 \ldots y_m$, $\zeta$ -- значения $z_1 \ldots z_l$

      Обозначим $C_{k, j} = \{ \xi = x_k, \eta = y_j \}$.\\
      Тогда 
      \begin{align*}
        &E\zeta = \sum_{i = 1}^l z_i P(\zeta = z_i) = \sum_{i = 1}^l z_i 
        \sum_{\substack{k,j:\\ a x_k + b y_j = z_i}} P(\xi = x_k, \eta = y_j) =\\
        &\sum_{i = 1}^l \sum_{\substack{k,j:\\ a x_k + b y_j = z_i}} (a x_k + b y_j) 
        P(\xi = x_k, \eta = y_j) =\\
        &\sum_{k = 1}^n \sum_{j = 1}^m (a x_k + b y_j) P(\xi = x_k, \eta = y_j) =\\
        &\sum_{k = 1}^n a x_k P(\xi = x_k) + \sum_{j = 1}^m b y_j P(\eta = y_j) = a E \xi + b E \eta
      \end{align*}

    \end{proof}

  \item Если $\xi \geq 0$, то $E \xi \geq 0$
    \begin{proof}
      Если $\xi \geq 0$, то все $x_k \geq 0 \implies E \xi \geq 0$
    \end{proof}

  \item Если $\xi \leq \eta$, то $E \xi \leq E \eta$
    \begin{proof}
      Рассмотрим $\zeta = \eta - \xi \geq 0$. По свойству 3\\
        \begin{align*}
          0 \leq E \zeta = E (\eta - \xi) = E \eta - E \xi
        \end{align*}
    \end{proof}
\end{enumerate}


\bigtitle{Неотрицательные случайные величины}

\begin{definition}
  Пусть $\xi$ -- неотрицательная случайная величина, 
  а $\{ \xi_n,\, n \in \setN \}$ -- $\forall$ последовательность неотрицательных 
  простых случайных величин, т.ч. $\xi_n \uparrow \xi$. 

  Тогда $E \xi_n \leq E \xi_{n + 1} \implies \exists$ предел $E \xi_n$ и
  \begin{align*}
    E \xi := \lim_{n \to \infty} E \xi_n
  \end{align*}
\end{definition}

\begin{lemma}
  Пусть $\{ \xi_n,\, n \in \setN \}$ и $\eta$ -- простые неотрицательные случайные вечилины, 
  причем $\xi_n \uparrow \xi \geq \eta$. Тогда 
  \begin{align*}
    \lim_{n \to \infty} E \xi_n \geq E \eta
  \end{align*}
\end{lemma}

\begin{proof}
  Пусть $\epsilon > 0$ фиксировано. 
  Рассмотрим $A_n = \condset{\omega}{\xi_n - \eta \geq -\epsilon}$

  Тогда
  \begin{align*}
    &E\xi_n = E(\xi_n I_{A_n}) + E(\xi_n I_{\comp{A}_n}) \geq E((\eta - \epsilon) I_{A_n}) = \\
    &E \eta - E\eta I_{\comp{A}_n} - \epsilon E I_{A_n} \geq 
    E\eta - c\, P(\comp{A}_n) - \epsilon P(A_n);\\
  \end{align*}
  где $c = \max\limits_{\omega \in \Omega} \eta(\omega)$. \\
  Заметим, что $A_n = \{ \xi_n \geq \eta - \epsilon \} \uparrow \Omega$ 
  т.к. $\xi_n \uparrow \xi \geq \eta \implies P(A_n) \rightarrow P(\Omega) = 1$\\
  Значит
  \begin{align*}
    \lim_{n \to \infty} E\xi_n \geq \lim_{n \to \infty} (E\eta - c\, P(\comp{A}_n) - \epsilon P(A_n)) = E\eta - \epsilon
  \end{align*}
  В силу произвольности $\epsilon$: $\lim\limits_{n \to \infty} E \xi_n \geq E \eta$
\end{proof}

\begin{corollary}
  Определение математического ожидания для неотрицательных случайных величин корректно.
\end{corollary}

\begin{proof}
  Пусть $\xi \geq 0$ и $\xi_n \uparrow \xi$,\; $\eta_n \uparrow \xi$ -- последовательность простых неотрицательных случайных величин. Тогда $\forall m$ в силу леммы
  \begin{align*}
    &\lim_{n \to \infty} E \xi_n \geq E \eta_m\\
    &\implies \lim_{n \to \infty} E \xi_n \geq \lim_{m \to \infty} E \eta_m
  \end{align*}
  Меняем $\xi$ и $\eta$ местами в рассуждении.
  \begin{align*}
    &\lim_{m \to \infty} E \eta_m \geq \lim_{n \to \infty} E \xi_n\\
    &\implies \lim_{n \to \infty} E \xi_n = \lim_{m \to \infty} E \eta_m
  \end{align*}
\end{proof}

\begin{remark}
  Если $\xi$ -- неотрицательная с.в., то
  \begin{align*}
    E\xi = \sup_{\eta: \eta \leq \xi} E\eta, \quad \text{ где $\eta$ -- неотриц. простая с.в.}
  \end{align*}
\end{remark}

\bigtitle{Произвольные случайные величины}

\begin{definition}
  Пусть $\xi$ -- произвольная случайная величина, $\xi = \xi^+ - \xi^-$

  \begin{enumerate}
    \item 
      Если $E \xi^+$ и $E \xi^-$ -- конечны, то 
      $\boxed{E\xi := E \xi^+ - E \xi^-}$

    \item 
      Если $E \xi^+ = +\infty$ и $E \xi^-$ -- конечно, то $\boxed{E\xi := +\infty}$

    \item
      Если $E \xi^+$ конечно и $E \xi^- = +\infty$, то $\boxed{E\xi := -\infty}$

    \item 
      Если $E \xi^+$ = $E \xi^-$ $= +\infty$, то $E \xi \text{ не существует(не определено)}$

  \end{enumerate}
\end{definition}

\begin{remark}~
  \begin{enumerate}
    \item
      Математическое ожидание случайной величины это интеграл Лебега по вероятностной мере~$P$
      \begin{align*}
        E\xi := \int\limits_{\Omega} \xi dP = \int\limits_{\Omega} \xi(\omega) P(d\omega)
      \end{align*}

    \item
      $E\xi$ -- конечно $\iff E|\xi|$ -- конечно.

    \item
      Множество случ. величин $\xi$ на $(\Omega, \setF, P)$ с условием: $E\xi$ -- конечно, образует
      пространство $L^1(\Omega, \setF, P)$. Далее мы убедимся, что это линейное пространство.

  \end{enumerate}
\end{remark}

\bigtitle{Свойства математического ожидания}

\begin{enumerate}[label=\protect\circled{\arabic*},series=mean_properties]

  \item
    Пусть $\xi$ -- случайная величина, $E \xi$ - конечно.

    Тогда для $\forall c \in \setR \; E(c\,\xi)$ конечно и
    \begin{align*}
      E(c\,\xi) = c E \xi
    \end{align*}

    \begin{proof}
      Для простых $\xi$, доказано ранее. Пусть $\xi \geq 0$. 

      Если $c \geq 0$, то возьмем последовательность простых неотрицательных 
      случайных величин $\xi_n$ т.ч. $\xi_n \uparrow \xi$.
      Тогда $c\, \xi_n \uparrow c\, \xi \implies$
      \begin{align*}
        &E (c\,\xi) = \lim_{n \to \infty} E(c\,\xi_n) = 
        c \lim_{n \to \infty} E \xi_n = c E \xi\\
      \end{align*}

      Если $c < 0$, то $c\,\xi = - (c\,\xi)^- = -(-c\,\xi)$

      $\implies E(c\,\xi) = - E(c\,\xi)^- = - E((-c)\,\xi) = c E \xi$\\

      Пусть $\xi$ - произвольная, $c \geq 0$

      Тогда 
      \begin{align*}
        E(c\,\xi) = E(c\,\xi)^+ - E(c\,\xi)^- = 
        E c\,\xi^+ - E c\,\xi^- = c (E \xi^+ - E \xi^-) = c E \xi
      \end{align*}


      Для $c < 0$ действуем аналогично.
    \end{proof}

  \item
    Если $\eta \leq \xi$ и $E \eta, E\xi$ - конечны, то 
    \begin{align*}
      E \eta \leq E \xi
    \end{align*}

    \begin{proof}
      Для простых $\xi$ и $\eta$ - доказано. Пусть $\xi$ и $\eta$ - неотрицательны. 
      Тогда
      \begin{align*}
        &E \eta = \sup_{\mu : \mu \leq \eta} E \mu
        \leq \sup_{\mu : \mu \leq \xi} E \mu = E \xi
      \end{align*}

      Пусть $\xi$ и $\eta$ - произвольные.

      Тогда $\xi^+(\omega) \geq \eta^+(\omega)$ и $\xi^-(\omega) \leq \eta^-(\omega)$

      $\implies E \eta = E \eta^+ - E \eta^- \leq E \xi^+ - E \xi^- = E \xi$
    \end{proof}

  \item 
    Если $E \xi$ - конечно, то
    \begin{align*}
      |E \xi| \leq E |\xi|
    \end{align*}

    \begin{proof}
      $|\xi| = \xi^+ + \xi^- \implies E|\xi|$ -- конечно.

      По свойству 2
      \begin{align*}
        &E(-|\xi|) \leq E \xi \leq E |\xi|
        \implies -E|\xi| \leq E\xi \leq E|\xi| \implies |E\xi| \leq E|\xi|
      \end{align*}
    \end{proof}

  \item 
    Аддитивность\\
    Пусть $\xi$ и $\eta$ - случайные величины. $E \xi$ и $E \eta$ - конечны.

    Тогда $E (\xi + \eta)$ - конечно и
    \begin{align*}
      E (\xi + \eta) = E \xi + E \eta
    \end{align*}

    \begin{proof}
      Для простых доказано ранее. Пусть $\xi$ и $\eta$ -- неотрицательные случайные величины.
      Возьмем $\xi_n, \eta_n$ - последовательности простых неотрицательных случайных величин, 
      т.ч. $\xi_n \uparrow \xi\;\eta_n \uparrow \eta$. Тогда $\xi_n + \eta_n \uparrow \xi + \eta$
      \begin{align*}
        E (\xi + \eta) = \lim_{n \to \infty} E(\xi_n + \eta_n) = 
        \lim_{n \to \infty} E \xi_n + \lim_{n \to \infty} E \eta_n = E \xi + E \eta
      \end{align*}
     
      Пусть $\xi$ и $\eta$ - произвольные случайные величины.

      Тогда $(\xi + \eta)^+ \leq (\xi^+ + \eta^+)$

      Обозначим $\delta = (\xi^+ + \eta^+) - (\xi + \eta)^+ \geq 0$.

	  По доказанному, $E\delta = E\xi^+ + E\eta^+ - E(\xi + \eta)^+$

	  Рассмотрим $(\xi + \eta)^- = (\xi + \eta)^+ - (\xi + \eta) = \xi^+ + \eta^+ - \delta - (\xi + \eta) = \xi^- + \eta^- - \delta$.

	  Следовательно $E(\xi + \eta)^- = E\xi^- + E\eta- - E\delta$

	  Тогда $E(\xi + \eta) = E(\xi + \eta)^+ - E(\xi + \eta)^- = E\xi^+ + E\eta^+ - E\delta- E\xi^- + E\eta^- + E\delta = E\xi + E\eta$

    \end{proof}

\end{enumerate}

