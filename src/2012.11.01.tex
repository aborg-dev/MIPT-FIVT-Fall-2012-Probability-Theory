
\begin{enumerate}[label=\protect\circled{\arabic*},series=mean_properties,start=5]

  \item
    \begin{enumerate}[label=\arabic*)]
      \item 
        Пусть $|\xi| \leq \eta$  и $E \eta$ - конечно. Тогда $E \xi$ конечно.
      
      \item 
        Пусть $\xi \leq \eta$ и $E \eta < +\infty$, тогда $E \xi < +\infty$\\
        Если $\xi \geq \eta$ и $E \eta > -\infty$, то $E \xi > -\infty$.
      
      \item 
        Если $E \xi$ конечно и $A \in \setF$, то $E (\xi I_A)$ тоже конечно.
    \end{enumerate}

    \begin{proof}~
      \begin{enumerate}[label=\arabic*)]
        \item 
          $\xi^- , \xi^+ \leq \eta \Rightarrow
          E \xi^+ = \sup\limits_{0 \leq \xi \leq \xi^+} 
          E \xi \leq \sup\limits_{0 \leq \xi \leq \eta} E \xi = E \eta < +\infty$

          Аналогично с $E \xi^-$. Тогда $E \xi = E \xi^+ - E \xi^-$ -- тоже конечно\\

        \item 
          $\xi^+ \leq \eta^+$ и $E \eta^+ < +\infty \Rightarrow$ по доказанному в 1), что \\
          $E \xi^+ < +\infty \Rightarrow E \xi < +\infty$\\

        \item
          $(\xi I_A)^+ = \xi^+ I_A \leq \xi^+ \Rightarrow E(\xi I_A)^+$ -- конечно.

          Аналогично, $E(\xi I_A)^-$ -- тоже конечно.
      \end{enumerate}
    \end{proof}
\end{enumerate}

\begin{definition}
  Говорят, что событие $A$ происходит почти наверное, если $P(A) = 1$
\end{definition}

\begin{enumerate}[resume*=mean_properties]

  \item
    $\xi = 0$ п.н. Тогда $E \xi = 0$

    \begin{proof}
      Пусть $\xi$ -- простая случайная величина.
      \begin{align*}
        \xi = \sum_{k = 1}^{n} x_k I_{A_k}, \quad 
        \text{где $x_1, \ldots x_n$ --- различные и $A_1 \ldots A_n$ --
        разбиение $\Omega$ : $A_k = \{ \xi = x_k \}$}
      \end{align*}

      Тогда, если $x_k \neq 0$, то $A_k = \{ \xi = x_k \} \subset \{ \xi \neq 0 \}$
      \begin{align*}
        &\Rightarrow P(A_k) \leq P(\xi \neq 0) = 0\\
        &\Rightarrow E \xi = \sum_{k = 1}^{n} x_k P(A_k) = 0
      \end{align*}

      Если $\xi \geq 0$ -- неотрицательная случайная величина, то 
      $E \xi = \sup\limits_{\eta \leq \xi} E \eta$, где $\eta$ -- простая неотрицательная с.в.\\
      Но для таких $\eta: 0 \leq \eta \leq \xi = 0 \Rightarrow \eta = 0$ п.н.

      Значит $E \eta = 0$\\

      Если $\xi$ -- произвольная случайная величина, то $\xi^+ = 0$ п.н., $\xi^- = 0$ п.н.

      По доказанному $E \xi^+ = E \xi^- = 0 \Rightarrow E \xi = E \xi^+ + E \xi^- = 0$
    \end{proof}

  \item
    Если $\xi = \eta$ п.н. и $E \eta$ - конечно, то $E \xi$ - конечно и $E \xi = E \eta$

    \begin{proof}
      Рассмотрим $A = \{\xi \neq \eta\}$. Тогда $I_A = 0$ п.н. $\Rightarrow \xi I_A = 0$ п.н.,
      $\eta I_A = 0$ п.н.
      \begin{align*}
        &\xi = \xi I_A + \xi I_{\,\comp{A}} = \xi I_A + \eta I_{\,\comp{A}} 
        \Rightarrow E \xi \text{ конечно и }\\
        &E \xi = E \xi I_A + E \eta I_{\,\comp{A}} = E \eta I_A + E \eta I_{\,\comp{A}} = E \eta
      \end{align*}
    \end{proof}

  \item
    Пусть $\xi \geq 0$ и $E \xi = 0$.

    Тогда $\xi = 0$ п.н.

    \begin{proof}
      Рассмотрим $A = \{ \xi > 0 \}$  и $A_n = \{ \xi > \frac{1}{n} \}$\\
      Тогда $A_n \uparrow A$. Но 
      \begin{align*}
        P(A_n) = E I_{A_n} \leq E (\xi_n) I_{A_n} \leq n E \xi = 0
      \end{align*}

      Отсюда в силу непрерывности вероятностной меры
      \begin{align*}
        P(A) = \lim_{n \to \infty} P(A_n) = 0
      \end{align*}
    \end{proof}

  \item
    Пусть $E \xi$ и $E \eta$ - конечно и для $\forall A \in \setF$ выполнено:
    \begin{align*}
      E (\xi I_A) \leq E (\eta I_A)
    \end{align*}

    Тогда $\xi \leq \eta$ п.н.

    \begin{proof}
      Рассмотрим $B \{ \xi > \eta \}$. Тогда $E \eta I_B \leq E \xi I_B \leq E \eta I_B$

      Тогда $E \xi I_B = E \eta I_B \Rightarrow E (\xi - \eta) I_B = 0$
      $\Rightarrow \expl{по свойству 8} \Rightarrow (\xi - \eta) I_B = 0$ п.н.

      Но $(\xi - \eta) I_B = 0 \Leftrightarrow I_B = 0$

      $\Rightarrow I_B = 0$  п.н. и, значит, $P(B) = 0$\\
    \end{proof}

\end{enumerate}

\mysection{Независимость случайных величин и векторов}

\begin{definition}
  Набор случайных векторов (величин) $\{ \xi_\alpha \}_{\alpha \in \mathfrak{A}}$ 
  называется \emph{независимым в совокупности}, если независимы в совокупности 
  $\braces{\setF_{\xi_\alpha}}_{\alpha \in \mathfrak{A}}$ сигма-алгебры, ими порожденные.
\end{definition}

\begin{corollary}
  Случайные величины $\xi_1 \ldots \xi_n$ - независимы в совокупности 
  $\Leftrightarrow \forall B_1 \ldots B_n \in B(\setR)$ события
  $\{ \xi_1 \in B_1 \} \ldots \{ \xi_n \in B_n \}$ - независимы в совокупности.
\end{corollary}

\begin{theorem}[критерий независимости для функции распределения]~

  Случайные величины $\xi_1 \ldots \xi_n$ -- независимы в совокупности 
  $\Leftrightarrow \forall x_1 \ldots, x_n \in \setR$
  \begin{align*}
    P(\xi_1 \leq x_1, \ldots, \xi_n \leq x_n) = P(\xi_1 \leq x_1) \ldots P(\xi_n \leq x_n)
  \end{align*}
  (функция распределения вектора $(\xi_1 \ldots \xi_n)$ 
  распадается в произведение функций распределения компонент)

\end{theorem}

\begin{proof}
  $\xi_1 \ldots \xi_n$ -- независимы в совокупности 
  $\Leftrightarrow \sigma$-алгебры $\setF_{\xi_1} \ldots \setF_{\xi_n}$ -- независимы в совокупности 
  $\Leftrightarrow \expl{критерий независ. $\sigma$-алгебр} \Leftrightarrow
  \pi$-системы порождающие эти $\sigma$-алгебры независимы.

  Для $\sigma$-алгебры $\setF_{\xi_i} = \condset{\{ \xi_i \in B \}}{B \in B(\setR) }$ 
  такой $\pi$-системой будет $\condset{\{ \xi_i \leq x \}}{x \in \setR}$.

  Это следует из того, что \quad $\sigma((-\infty; x] : x \in \setR) = B(\setR)$\\
  $\Leftrightarrow \pi$-системы $\condset{\{ \xi_i \leq x_i \}}{x_i \in \setR}$ 
  -- независимы\\
  $\Leftrightarrow \forall x_1 \ldots x_n$ - события.
  $\{ \xi_1 \leq x_1 \} \ldots \{ \xi_n \leq x_n \}$ независимы в совокупности   
  \begin{align*}
    \Leftrightarrow P(\xi_1 \leq x_1, \ldots, \xi_n \leq x_n) = 
    P(\xi_1 \leq x_1) \ldots P(\xi_n \leq x_n), \quad \forall x_1 \ldots x_n \in \setR
  \end{align*}
\end{proof}

\begin{theorem}[функции от независимых -- тоже независимы]~

  Пусть $\xi_1 \ldots \xi_m$ -- независимые случайные векторы, $\xi_i$ имеет размерность $n_i$.

  Пусть $f_i\colon \setR^{n_i} \to \setR^{k_i}$ -- борелевская функция, $\forall i = 1 \ldots n$

  Тогда $f_1 (\xi_1), \ldots, f_n(\xi_n)$ -- независимы в совокупности.

\end{theorem}

\begin{proof}
  Обозначим $\eta_i = f_i (\xi_i)$. 

  Тогда $\forall B \in B(\setR^{k_i}):$
  \begin{align*}
    \{ \eta_i \in B \} = \{ f_i (\xi_i) \in B \} = \{ \xi_i \in (f_i^{-1}) (B) \} \in \setF_{\xi_i}
  \end{align*}
  то есть $\setF_{\eta_i} \subset \setF_{\xi_i}$

  По условию $\setF_{\xi_1} \ldots \setF_{\xi_n}$ -- независимы
  $\Rightarrow \setF_{\eta_1} \ldots \setF_{\eta_n}$ -- тоже независимы.

  $\Leftrightarrow \eta_1 \ldots \eta_n$ -- независимы в совокупности.
\end{proof}

\begin{theorem}
  Пусть случайная величина $\xi$ и $\eta$ -- независимы, причем $E \xi$ и $E \eta$ -- конечны. 
  Тогда $E \xi \eta$ тоже конечно и $E \xi \eta = E \xi E \eta$
\end{theorem}

\begin{proof}
  Пусть $\xi$ и $\eta$ - простые случайные величины, \\
  $\xi$ - принимает значения $x_1 \ldots x_n$, \quad$\eta $ - принимает значения $y_1 \ldots y_m$. 

  Тогда по линейности:
  \begin{align*}
    &E \xi \eta = \sum_{k, j} x_k y_j P(\xi = x_k, \eta = y_j) = \expl{независимость} =
    \sum_{k, j} x_k y_j P(\xi = x_k) P (\eta = y_j) = \\
    &= \pars{\sum_{k = 1}^{n} x_k P(\xi = x_k))} \pars{\sum_{j = 1}^{m} y_j P(\eta = y_j)} 
    = E \xi E \eta
  \end{align*}

  Пусть теперь $\eta$ и $\xi$ -- неотрицательные случайные величины.

  Тогда по теореме о приближении простыми $\exists$ последовательность простых $\setF_\xi$ --
  измеримых неотрицательных случайных величин $\{ \xi_n, n \in \setN \}$, т.ч. $\xi_n \uparrow \xi$. 
  Аналогично $\exists \{ \eta_n, n \in \setN \}$ -- последовательных простых неотрицательных 
  $\setF_\eta$ - измеримых случайных величин, т.ч. $\eta_n \uparrow \eta$

  Тогда $\xi_n \eta_n \uparrow \xi \eta$ и $\forall n: \xi_n$  и $\eta_n$ -- независимы.
  \begin{align*}
    \Rightarrow E \xi \eta = \lim_{n \to \infty} E\xi_n \eta_n = 
    \expl{независимость $\xi_n$ и $\eta_n$} = \lim_{n \to \infty} E \xi_n E \eta_n = E\xi E\eta
  \end{align*}

  Пусть $\xi$ и $\eta$ - произвольные с.в. 
  Тогда $\xi^+, \xi^-$ -- функции от $\xi,\quad \eta^+, \eta^-$ -- функции от $\eta$
  $\Rightarrow \xi^+, \xi^-$ -- независимы c $\eta^+, \eta^-$

  Отсюда получаем 
  \begin{align*}
    &(\xi \eta)^+ = \xi^+ \eta^+ + \xi^- \eta^- 
    \Rightarrow E(\xi \eta)^+ = E (\xi^+ \eta^+) + E(\xi^- \eta^-) =\\
    &= \expl{независимость $\xi^+$ с $\eta^+$ и $\xi^-$ с $\eta^-$} 
    = E\xi^+ E\eta^+ + E\xi^- E\eta^-
  \end{align*}

  Аналогично $E(\xi \eta)^- = E\xi^+ E\eta^- + E\xi^- E\eta^+$

  $\Rightarrow E \xi \eta$ конечно и
  $E \xi \eta = E\xi^+ \eta^+ + E \xi^- E \eta^- - E \xi^+ \eta^- - E \xi^- E\eta^+ 
  = E \xi E \eta$
  
\end{proof}

\bigtitle{Дисперсия и ковариация}

\begin{definition}
  \emph{Дисперсией} с.в. $\xi$ называетют
  \begin{equation*}
    D\xi = E(\xi - E\xi)^2, \quad \text{если $E\xi$ существует}
  \end{equation*}
\end{definition}

\begin{definition}
  \emph{Ковариацией} случайных величин $\xi$ и $\eta$ называют
  \begin{align*}
    \cov(\xi, \eta) = E(\xi - E\xi) (\eta - E\eta)
  \end{align*}
  Если $\cov(\xi, \eta) = 0$, то $\xi$ и $\eta$ называются \emph{некоррелированными}.
\end{definition}

Если $D\xi$ и $D\eta$ -- конечны и положительны, то можно определить расстояние
\begin{align*}
  \rho(\xi, \eta) = \frac{\cov(\xi, \eta)}{\sqrt{D \xi D \eta}}
\end{align*}

которое называется \emph{коэффициентом корреляции} $\xi$ и $\eta$

\begin{lemma}[свойства дисперсии и ковариации]~

  Если все математические ожидания конечны, то 
  \begin{enumerate}
    \item 
      Ковариация билинейна.

    \item 
      $cov(\xi, \eta) = E \xi \eta - E\xi E\eta$

      $D\xi = cos(\xi, \xi) = E\xi^2 - (E \xi)^2$

    \item 
      $D(c\, \xi) = c^2 D \xi, D(\xi + c) = D \xi$

    \item 
      Неравенство Коши-Буняковского.
      \begin{align*}
        |E\xi \eta|^2 \leq E\xi^2 E\eta^2
      \end{align*}

    \item 
      $|\rho(\xi, \eta)| \leq 1$, 
      причем $\rho(\xi, \eta) = 1 \Leftrightarrow \xi$ и $\eta$ -- п.н. линейно зависимы.
  \end{enumerate}

\end{lemma}

\begin{proof}~

  Свойства $1) - 3)$ легко вытекают из свойств математического ожидания.

  \begin{enumerate}[start=4]
    \item 
      Рассмотрим для $\lambda \in \setR:$
      \begin{align*}
        f(\lambda) = E(\xi + \lambda \eta)^2 \geq 0
      \end{align*}
      Но $f(\lambda) = E \xi^2 + 2E \xi \eta \lambda + \lambda^2 E \eta^2 \geq 0$
      $\Leftrightarrow$ дискриминант $\leq 0$, т.е.
      $4 [(E \xi \eta)^2 - E \xi^2 E\eta^2] \leq 0$

    \item 
      Рассмотрим $\xi_1 = \xi - E \xi$, $\eta_1 = \eta - E \eta$

      Тогда $\cov(\xi, \eta) = E \xi_1 \eta_1,\quad D \xi = E \xi_1^2,\quad D \eta = E\eta_1^2$

      $\Rightarrow |\rho(\xi, \eta)| = 
      \walls{\frac{E \xi_1 \eta_1}{\sqrt{E \xi_1^2 E\eta_1^2}}} \leq 1$, 
      по нер-ву Коши-Буняковского.

      При этом $|\rho(\xi, \eta)| = 1 \Leftrightarrow$ дискриминант $= 0 
      \Leftrightarrow \exists ! \lambda_0 \in \setR$ т.ч. $f(\lambda_0) = 0$. 
      т.е. $E(\xi_1 + \lambda_0 \eta_1)^2 = 0$

      $\Rightarrow \xi_1 + \lambda_0 \eta_1 = 0$ п.н. т.е.
      \begin{align*}
        \xi = E \xi - \lambda_0 (\eta - E \eta) \text{ п.н.}
      \end{align*}
  \end{enumerate}
\end{proof}

\begin{corollary}
  Если $\xi_1, \ldots, \xi_n$ -- попарно некоррелируют, $D \xi_i < +\infty$, тогда
  \begin{align*}
    D(\xi_1 + \ldots \xi_n) = \sum_{k = 1}^{n} D\xi_k
  \end{align*}
\end{corollary}

\begin{proof}
  \begin{align*}
    &D(\xi_1 + \ldots + \xi_k) = \cov(\xi_1 + \ldots + \xi_k, \xi_1 + \ldots \xi_k) =\\
    &\sum_{i, j} \cov(\xi_i, \xi_j) = \sum_i \cov(\xi_i, \xi_i) = \sum_{i} D \xi_i
  \end{align*}
\end{proof}

\begin{corollary}
  $\xi_1 \ldots \xi_n$ -- независимы, $D \xi_i < +\infty$. 
  Тогда $D(\xi_1 + \ldots \xi_n) = \sum\limits_{k = 1}^{n} D \xi_k$
\end{corollary}

\begin{definition}
  Пусть $\xi = (\xi_1, \ldots, \xi_n)$ -- случ. вектор.

  Тогда его \emph{мат. ожиданием} называется вектор из мат. ожиданий его компонент:
  \begin{align*}
    E \xi = (E\xi_1, \ldots, E\xi_n)
  \end{align*}
\end{definition}

\begin{definition}
  \emph{Дисперсией} вектора $\xi$ называется его матрица ковариаций:
  \begin{align*}
    D\xi = \begin{Vmatrix}\cov(\xi_i, \xi_j)\end{Vmatrix}_{i, j = 1}^{n}\; 
    \text{ --- матрица $n \times n$}
  \end{align*}

\end{definition}

\begin{lemma}
  Матрица ковариаций случайного вектора является симметрической и неотрицательно определенной.
\end{lemma}

\begin{proof}
  $D \xi = \begin{Vmatrix}\cov(\xi_i, \xi_j)\end{Vmatrix}_{i, j = 1}^{n}$ -- симметричная 
  т.к $\cov(\xi_i, \xi_j) = \cov(\xi_j, \xi_i)$\\

  Пусть $x_1 \ldots x_n \in \setR,\quad x = (x_1, \ldots, x_n)$ -- вектор.
  \begin{align*}
    &\langle D\xi x, x \rangle = \sum_{i, j = 1}^{n} cov(\xi_i, \xi_j) x_i x_j 
    = \expl{линейность ковариации} = \sum_{i, j = 1}^n cov(x_i \xi_i, x_j \xi_j) =\\
    &= \expl{суммируем по $i$} = \sum_{j = 1}^n cov(x_1 \xi_1 + \ldots x_n \xi_n, x_j \xi_j) =\\
    &= \expl{суммируем по $j$} = cov(x_1 \xi_1 + \ldots x_n \xi_n, x_1 \xi_1 + \ldots + x_n \xi_n) 
    = D(x_1 \xi_1 + \ldots + x_n \xi_n) \geq 0
  \end{align*}
  $\Rightarrow$ неотр. определенная
\end{proof}
