
\mysection{Вероятностная мера на $(\setR, B_{\setR}$)}

\begin{theorem}[Каратеодори, о продолжении меры]
	Пусть $\Omega$ -- некоторое множество, $\setA$ - алгебра на нем, $P_{\sigma}$ -- 
	вероятностая мера на $(\Omega, \setA)$
	Тогда $\exists !$ вероятностная мера $P$ на $(\Omega, \sigma(\setA))$, 
	являющаяся продолжением меры $P_{\sigma}$, т.е 
	$\forall A \in \setA \hookrightarrow P_{\sigma}(\setA) = P(\setA)$
\end{theorem}

\begin{lemma}
	Пусть $(\Omega, \setF)$ -- измеримое пространство, $\setM$ -- $\pi$-система в 
	$\setF$, а $P$ и $Q$ -- две вероятностные меры на $(\Omega, \setF)$. 
	Тогда если $P |_{\setM} = Q |_{\setM}$, то 
	\begin{equation*}
		P |_{\sigma(\setM)} = Q |_{\sigma(\setM)}
	\end{equation*}
\end{lemma}

\begin{proof}
	Рассмотрим
	\begin{equation*}
		\setL = \condset{A \in \setF}{P(A) = Q(A)}
	\end{equation*}
	
	Покажем, что $\setL$ -- это $\lambda$-система.
	
	\begin{enumerate}
		\item $\Omega \in \setL : P(\Omega) = Q(\Omega)$
		\item 
			Пусть $A, B \in \setL$. $A \subset B \Rightarrow$
			\begin{align*}
				P(B \setminus A) = P(B) - P(A) = Q(B) - Q(A) = Q(B \setminus A) 
				\Rightarrow (B \setminus A) \in \setL
			\end{align*}
		\item 
			Пусть $A_n \uparrow A,\; A_n \in \setL \quad \forall n$. Тогда 
			\begin{align*}
				&P(A) = \expl{непрерывность вероятностной меры} = \lim_n P(A_n) = \lim_n Q(A_n) =\\
				&= \expl{непрерывность вероятностной меры} = Q(A)\\
				&\Rightarrow A \in \setL
			\end{align*}
	\end{enumerate}
			Доказали, что $\setL$ -- $\lambda$-система. По условию $\setM \subset \setL 
			\Rightarrow$ $\lambda (\setM) \subset \setL$. По теореме о монотонных классах получаем, 
			что $\sigma(\setM) \subset \setL$, т.е. 
			\begin{equation*}
				P |_{\sigma(\setM)} = Q |_{\sigma(\setM)}
			\end{equation*}				
\end{proof}

\begin{corollary}[единственность в теореме Каратеодори]~

	Пусть $P$ и $Q$ -- два продолжения $P_\sigma$ на $\sigma(\setA)$. 
	Но $\setA$ -- алгебра $\Rightarrow \pi$-система.
	\begin{align*}
		P |_\setA = P_\sigma = Q |_\setA
	\end{align*}
	$\Rightarrow$ по лемме получаем, что $\forall A \in \sigma(\setA)$\\
	$P(A) = Q(A)$, т.е продолжение единственно.\\
\end{corollary}

Пусть $P$ -- вероятностная мера на $(\setR, B(\setR))$

\begin{definition}
	Функция $F(x), x \in \setR$, заданная по правилу
	\begin{align*}
		F(x) = P((-\infty, x])
	\end{align*}
	называется \emph{функцией распределения вероятностной меры $P$}.
\end{definition}

\begin{lemma}[свойства функции распределения]~

	Пусть $F(x)$ -- функция распределения вероятностной меры $P$. Тогда
	\begin{enumerate}
		\item 
			$F(x)$ - неубывающая

		\item 
			$\lim\limits_{x \rightarrow -\infty} F(x) = 0, \, 
			\lim\limits_{x \rightarrow +\infty} F(x) = 1$

		\item 
			$F(x)$ непрерывная справа.
	\end{enumerate}
\end{lemma}

\begin{proof}~

	\begin{enumerate}
		\item 
			Пусть $y \geq x$. Тогда
			\begin{align*}
				F(y) - F(x) = P((-\infty; y]) - P((-\infty; x) = P((x, y]) \geq 0
			\end{align*}

		\item 
			Пусть $x_n \rightarrow -\infty$ при $n \rightarrow \infty$. 
			Тогда $(-\infty; x_n] \downarrow \emptyset\; \Rightarrow$ по непрерывности вероятностной меры.
			\begin{align*}
				F(x_n) = P((-\infty, x_n]) \todown{n \to \infty} P(\emptyset) = 0
			\end{align*}

			Аналогично, если $x_n \to +\infty$, то $(-\infty; x_n] \uparrow \setR\; \Rightarrow$ 
			в силу непрерывности вероятностой меры.
			\begin{align*}
				F(x_n) = P((-\infty; x_n])) \todown{n \to \infty} P(\setR) = 1
			\end{align*}

		\item 
			Пусть убывающая $x_n \to x + 0$ Тогда $(-\infty, x_n]) \downarrow (-\infty; x]\; \Rightarrow$ 
			в силу непрерывности вероятностой меры.
			\begin{align*}
				F(x_n) = P((-\infty; x_n]) \todown{n \to \infty} P((-\infty; x]) = F(x)
			\end{align*}
	\end{enumerate}
\end{proof}

\begin{corollary}
	Функция распределения имеет предел слева в каждой точке $x \in \setR$, при этом точек разрыва у нее не более чем счетное множество.

	\begin{proof}
		$$\lim_{x \to a-0} F(x) = P((-\infty; a))$$
		Каждая точка разрыва является скачком. Каждому скачку сопоставим отрезок.
		Отрезки скачков не пересекаются, так как функция монотонная. 
		В каждом из них найдется рациональная точка $\Rightarrow$ точек разрыва не более чем счетно.
	\end{proof}
\end{corollary}

\begin{definition}
	Функция $F(x)$ называется функцией распределения на $\setR$, если она удовлетворяет свойствам 
	1), 2), 3) из леммы.
\end{definition}

\begin{theorem}[взаимнооднозначное соответствие между функциями распределения и вероятностными мерами]
	$F(x)$ -- функция распределения на $\setR$. 
	Тогда существует единственная вероятностная мера $P$ на $(\setR, B(\setR))$, т.ч. $F(x)$ 
	является функцией распределния $P$, т.е. $\forall x \in \setR$
	\begin{align*}
		F(x) = P((-\infty; x])
	\end{align*}
\end{theorem}

\boldtitle{Идея доказательства}
Рассмотрим $\setA$ -- алгебру, состоящую из конечных объединений непересекающихся полуинтервалов вида $(a, b]$, т.е. $\forall A \in \setA$ имеет вид:
\begin{align*}
	A = \bigsqcup_{k = 1}^{n} (a_k, b_k] \quad (*)
\end{align*}
где $-\infty \leq a_1 < b_1 < a_2 < \ldots < b_n \leq +\infty$\\

Рассмотрим функцию $P_0$ на $\setA$, заданную по правилу: Если $A$ имеет вид $(*)$, то
\begin{align*}
	P_0 (A) = \sum_{k = 1}^{n} (F(b_k) - F(a_k))
\end{align*}

Легко видеть, что $P_0$ обладает свойствами 
\begin{enumerate}
	\item $P_0(A) \in [0, 1]\quad \forall A \in \setA$
	\item $P_0(\setR) = F(+\infty) - F(-\infty) = 1$
	\item 
		$P_0$ -- конечно-аддитивна, т.е. 
		$\forall A, B \in \setA\\
		 A \cap B = \emptyset \hookrightarrow P_0 (A \cup B) = P_0(A) + P_0(B)$
\end{enumerate}

Если бы удалось доказать, что $P_0$  счетно-аддитивна на $\setA$, 
то $P_0$ стала бы вероятностной мерой на $(\setR, \setA)$ и по теореме Каратеодори
 её можно было бы продолжить единственным образом до вероятностной меры $P$ 
 на $(\setR, \sigma(\setA))$, а $\sigma(\setA) = B(\setR)$.\\

Тогда бы $F(x)$ была функцией распределения меры $P$

\begin{align*}
	F(x) = P_0 ((-\infty; x]) = P((-\infty; x])
\end{align*}

\mysection{Классификация вероятностных мер и функций распределения на прямой}

\begin{enumerate}[label=\protect\circled{\arabic*},series=distributions]
	\item 
		\bigtitle{Дискретные распределения}

		Пусть $\mathcal{X} \subset \setR$ -- не более чем счетное множество.
		\begin{definition}
			Вероятностная мера $P$ на $(\setR, B(\setR))$, 
			удовлетворяющая свойству $P(\setR~\setminus~\mathcal{X})~=~0$, 
			называется дискретной вероятностной мерой на $\mathcal{X}$. 
			Её функция функция распределения называется дискретной.\\
			Пусть $\mathcal{X} = \braces{x_k}$ и положим $p_k = P(\braces{x_k})$\\
			Тогда $P(\mathcal{X}) = 1 = \sum\limits_{k} P(\braces{x_k}) = \sum\limits_k p_k$
		\end{definition}
\end{enumerate}

\begin{definition}
	Набор чисел $(p_0, p_1, \ldots)$ называется распределением вероятностей на $\mathcal{X}$.\\
	
	Как выглядит функция распределения дискретной верятностной меры $P$?\\
	$F(x)$ -- кусочно-постоянная разрывная в точках $x_k \in \mathcal{X}$. 
	При этом величина скачка равна
	\begin{align*}
		\Delta F(x_k) = F(x_k) - F(x_k - 0) = P(\braces{x_k}) = p_k
	\end{align*}
\end{definition}

\bigtitle{Примеры дискретных распределений}

\begin{enumerate}
	\item 
		Дискретное равномерное $\mathcal{X} = \braces{1, \ldots, N},\;
		k = 1, \ldots, N$ и $p_k = 1 / N$ для $\forall k \in \mathcal{X}$.
	
	\item 
		Бернуллиевское
		\begin{align*}
			&\mathcal{X} = \{ 0, 1 \}, k = 0, 1\\
			&p_k = p^k (1 - p)^{1 - k},
		\end{align*}
		где $p \in [0, 1]$ - параметр.
		
	\item 
		Биномиальное распределение
		\begin{align*}
			&\mathcal{X} = \{ 0, \ldots , n\}\\
			&p_k = \comb{n}{k}\, p^k (1 - p)^{n - k},
		\end{align*}
		где $p \in [0, 1]$ - параметр.

	\item
		Пуассоновское распределение
		\begin{align*}
			&\mathcal{X} = \setZ_+\\
			&k = 0, 1, 2, \ldots \\
			&p_k = \frac{\lambda^{k}}{k!} e^{-\lambda}, \lambda > 0 -- \text{параметр}
		\end{align*}

		Моделирование: биномиальное $\rightarrow$ пуассоновское\\
\end{enumerate}
