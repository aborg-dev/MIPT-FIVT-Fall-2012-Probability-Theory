
\begin{lemma}
  Пусть $\xi$ - неотр. с.в., $E\xi$ - конечно.
  Тогда
  \begin{align*}
    \sum_{n = 1}^{\infty} P(\xi \geq n) \leq E \xi \leq 1 + \sum_{n = 1}^{\infty} P(\xi \geq n) = 
    \sum_{n = 0}^{\infty} P(\xi \geq n)
  \end{align*}
\end{lemma}


\begin{proof}
  \begin{align*}
    &\sum_{n = 1}^{\infty} P(\xi \geq n)
    = \sum_{n = 1}^{\infty} \sum_{k = n}^{\infty} P(k \leq \xi < k + 1)
    = \sum_{k = 1}^{\infty} k P(k \leq \xi < k + 1) 
    = \sum_{k = 0}^{\infty} k P(k \leq \xi < k + 1) =\\
    &= \sum_{k = 0}^{\infty} E(k I \{ k \leq \xi < k + 1 \}) 
    \leq \sum_{k = 0}^{\infty} E(\xi I \{ k \leq \xi < k + 1 \})
    = E \pars{\sum_{k = 1}^{\infty} \xi I \{ k \leq \xi < k + 1 \}} = \\
    &= E \xi \leq E\pars{\sum_{k = 0}^{\infty} (k + 1) I \{ k \leq \xi < k + 1 \}}
    = \sum_{n = 1}^{\infty} P(\xi \geq n) + \sum_{n = 0}^{\infty} P(k \leq \xi < k + 1)  
    = \sum_{k = 1}^{\infty} P(\xi \geq n) + 1
  \end{align*}
\end{proof}

\begin{definition}
  Случайные величины $\xi$ и $\eta$ наз. \emph{одинаково распределенными}, 
  если у них совпадают функции распределения. 

  Обозначение: $\xi \equp{d} \eta$
\end{definition}

\begin{statement}
  Если $\xi \equp{d} \eta$, то для $\forall$ борелевской $g(x)$ т.ч. 
  $E g(\xi)$ конечно, выполнено:
  \begin{align*}
    E g(\xi) = E g(\eta)
  \end{align*}
\end{statement}

\begin{theorem}[Усиленный закон больших чисел в форме Колмогорова]~

  Пусть $\{ \xi_n,\; n \in \setN \}$ -- независимые одинаково распределенные случ. величины 
  (н.о.р.с.в), т.ч: $E |\xi_i| < +\infty$. 

  Тогда
  \begin{align*}
    \frac{\xi_1 + \ldots + \xi_n}{n} \toae m = E \xi_1
  \end{align*}

  \begin{proof}
    $E |\xi_i|$ -- конечно. 
    Тогда по доказанной выше лемме:
    \begin{align*}
      \sum_{n = 1}^{\infty} P(|\xi_1| \geq n) < +\infty
    \end{align*}

    В силу одинаковой распределенности:
    \begin{align*}
      \sum_{n = 1}^{\infty} P(|\xi_1| \geq n) = \sum_{n = 1}^{\infty} P(|\xi_n| \geq n) < +\infty
    \end{align*}

    Согласно лемме Бореля-Кантелли:
    \begin{align*}
      P(\{ |\xi_n| \geq n\} \text{ б.ч.}) = 0
    \end{align*}

    $\Rightarrow$ с вероятностью 1 $\forall n$, кроме конечного числа, 
    выполнено $\{ |\xi_n| \leq n \}$.

    Обозначим $\widetilde{\xi_n} = \xi_n\; I \{ |\xi_n| \leq n \}$. 

    Тогда с вероятностью 1, $\widetilde{\xi_n} = \xi_n$, кроме конечного числа элементов.

    Считаем, что $E \xi_i = 0$\\

    Получаем, что 
    \begin{align*}
      P\pars{\frac{\xi_1 + \ldots + \xi_n}{n} \to 0} = 
      P\pars{\frac{\widetilde{\xi_1} + \ldots + \widetilde{\xi_n}}{n} \to 0}
    \end{align*}

    Рассмотрим $E \widetilde{\xi_n}:$
    
    \begin{align*}
      E \widetilde{\xi_n} = E \xi_n\; I \{ |\xi_n| \leq n \} = 
      E \xi_1\; I \{ |\xi_1| < n \} \todown{n \to \infty} E \xi_1 = 0
    \end{align*}

    Согласно лемме Тёплица
    \begin{align*}
      \frac{1}{n} \sum_{k = 1}^{n} E \widetilde{\xi_k} \to 0,\quad \text{при } n \to \infty
    \end{align*}

    Значит
    \begin{align*}
      \frac{\widetilde{\xi_1} + \ldots + \widetilde{\xi_n}}{n} \toae 0
      \Leftrightarrow \frac{(\widetilde{\xi_1} - E \widetilde{\xi_1}) 
      + \ldots + (\widetilde{\xi_n} - E \widetilde{\xi_n})}{n} \toae 0
    \end{align*}

    Обозначим $\bar{\xi_n} = \widetilde{\xi_n} - E \widetilde{\xi_n}$. 

    Согласно лемме Кронекера, если сходится ряд
    \begin{align*}
      \sum_{k = 1}^{\infty} \frac{\bar{\xi_k}}{k}, \quad \text{то} 
      \frac{\bar{\xi_1} + \ldots + \bar{\xi_n}}{n} \to 0
    \end{align*}
    (для фикс. $\omega \in \Omega$)

    Остается проверить, что ряд $\sum\limits_{k = 1}^{\infty} \frac{\bar{\xi_k}}{k}$ 
    сходится с вероятностью 1.

    Согласно теореме Колмогорова-Хинчина для этого достаточно показать
    ($\bar{\xi_k}$ - нез., $E \bar{\xi_k} = 0$), что сходится ряд
    $\sum\limits_{k = 1}^{\infty} \dfrac{E \bar{\xi_k^2}}{k^2}$

    \begin{align*}
      &\sum_{k = 1}^{\infty} \frac{E \bar{\xi_k^2}}{k^2}
      = \sum_{k = 1}^{\infty} \frac{D \widetilde{\xi_k}}{k^2} 
      \leq \sum_{k = 1}^{\infty} \frac{E \widetilde{\xi_k^2}}{k^2}
      = \sum_{k = 1}^{\infty} \frac{1}{k^2} E \xi_k^2\; I \{ |\xi_k| \leq k \}
      = \sum_{k = 1}^{\infty} \frac{1}{k^2} E (\xi_1^2\; I \{ |\xi_1| \leq k \}) =\\
      &= \sum_{k = 1}^{\infty} \frac{1}{k^2} 
        E (\xi_1^2 \sum_{n = 1}^{k} I \{ n - 1 < |\xi_1| \leq n \})
      = \sum_{n = 1}^{\infty} E (\xi_1^2 I \{ n - 1 < |\xi_1| \leq n \})\cdot 
        \sum_{k = n}^{\infty} \frac{1}{k^2} \leq\\
      &\leq \sum_{n = 1}^{\infty} E\pars{\xi_1^2 I \{n - 1 < |\xi_1| \leq n \}\cdot \frac{2}{n}}
      \leq 2 \sum_{n = 1}^{\infty} E(|\xi_1| I \{ n - 1 < |\xi_1| \leq n \}) =\\
      &= 2 E \pars{\sum_{n = 1}^{\infty} |\xi_1| I \{ n - 1 < |\xi_1| \leq n \}} 
      = 2 E |\xi_1| < +\infty
    \end{align*}
    
  \end{proof}
\end{theorem}

\mysection{Замена переменных в интеграле Лебега}

Пусть $(\Omega, \setF, P)$ -- вероятностное пространство, 
$\xi$ -- с.в. на нем и $E \xi$ -- конечно.

\begin{designations}~

  \begin{enumerate}
    \item 
      $E \xi = \int\limits_{\Omega} \xi\, d P$ -- интеграл Лебега от $\xi$ по вер. мере $P$.

    \item
      $\int\limits_{A} \xi\, d P := E (\xi I_A)$ для $\forall A \in \setF$
  \end{enumerate}

\end{designations}

Напоминание: 
Распределение $P_{\xi}$ -- это вероятностная мера на $(\setR, B(\setR))$
($P_{\xi} = P(\xi \in B)$)

Для вер. пр-ва $(\setR, B(\setR), P_{\xi})$ тоже можно ввести мат. ожидание.

\begin{enumerate}
  \item
    $\int\limits_{\setR} g(x) P_{\xi}(dx)$ -- мат. ожидание с.в. $g(x)$ на таком пространстве.

  \item
    \begin{align*}
      \int\limits_{A} g(x) P_{\xi}(dx) := \int\limits_{\setR} g(x) I_{A}(x) P_{\xi}(dx),\quad 
      \forall A \in B(\setR)
    \end{align*}

  \item
    Если $F_{\xi}(x)$ -- ф.р. с.в. $\xi$, то
    \begin{align*}
      dF_{\xi}(x) := P_{\xi}(dx)
    \end{align*}
\end{enumerate}

$\underline{\text{Вопрос}}$: можно ли вычислить $E g(\xi)$, зная только ее распределение?

\begin{theorem}[замена переменных в интеграле Лебега]~

  Пусть $\xi = (\xi_1, \ldots, \xi_n)$ -- случайный вектор, 
  $g\colon \setRn \to \setR$ -- борелевская функция. 

  Тогда для $\forall B \in B(\setR)$ выполнено:
  \begin{align*}
    E(g(\xi)) I\{ \xi \in B \} \equp{\mathrm{def}} \int\limits_{\{ \xi \in B \}} g(\xi) dP = 
    \int\limits_B g(x) P_{\xi} (dx)
  \end{align*}

  \begin{proof}
    Пусть $g$ -- простая: $g(x) = I_{A}(x)$ для $A \in B(\setRn)$.

    Тогда
    \begin{align*}
      &E g(\xi) I \{ \xi \in B \} = E I \{ \xi \in A \} I \{ \xi \in B \} =
      E I \{ \xi \in A \cap B \} = \\
      &= \int\limits_{A \cap B} P_{\xi} (dx) = \int\limits_{B} I_{A} (x) P_{\xi} (dx) 
      = \int\limits_B g(x) P_{\xi} (dx)
    \end{align*}

    Если функция $g(x)$ -- простая неотрицательна, 
    то искомое равенство следует из линейности мат. ожидания.
    Если $g(x)$ -- произвольная неотрицательная, 
    то рассмотрим последовательность простых неотриц. $g_n(x)$ т.ч. $g_n(x) \uparrow g(x)$.

    Тогда по теореме о монотонной сходимости:
    \begin{align*}
      &E g_n(\xi) I \{ \xi \in B \} \todown{n \to \infty} E g(\xi) I \{ \xi \in B \}\\
      &\int\limits_B g_n (x) P_{\xi} (dx) \todown{n \to \infty} \int\limits_B g(x) P_{\xi} (dx)
    \end{align*}

    $\Rightarrow$ доказано для неотриц. $g(x)$.

    В общем случае, пользуемся разложением $g(x) = g^+(x) - g^-(x)$ и линейностью математического ожидания.
  \end{proof}
\end{theorem}

\begin{corollary}~

  \begin{enumerate}[label=\protect\circled{\arabic*},series=lebesgue_corollary]
    \item
      Для вычисления $E g(\xi)$ достаточно знать только распределение $\xi$.

    \item
      Для $\forall$ борелевской $g(x)\colon \setRn \to \setR$ и 
      $\forall$ случ. вектора $\xi$ из $\setRn$:
      \begin{align*}
        E g(\xi) = \int\limits_{\setRn} g(x) P_{\xi} (dx)
      \end{align*}

      \begin{proof}
        Достаточно положить $B = \setRn$ в теореме.
      \end{proof}

    \item
      Если $\xi$ -- с.в., то
      \begin{align*}
        E \xi = \int\limits_{\setR} x P_{\xi} (dx)
      \end{align*}

      \begin{proof}
        Достаточно положить $g(x) = x$ в \circled{2}
      \end{proof}

    \item
      Если $\xi \equp{d} \eta$ -- одинаково распределены, 
      то для $\forall$ борелевской $g(x): \quad E g(\xi) = E g(\eta)$

      \begin{proof}
        \begin{align*}
          E g(\xi) = \int\limits_{\setR} g(x) P_{\xi} (dx) = 
          \int\limits_{\setR} g(x) P_{\eta} (dx) = E g(\eta)
        \end{align*}
      \end{proof}

    \item 
      Пусть $\xi$ -- дискретная с.в. со значениями в $\mathcal{X} = \{ x_i \}_{i = 1}^{\infty}$. 

      Тогда для $\forall$ борелевской функции $g(x):$
      \begin{align*}
        E g(\xi) = \sum_{i = 1}^{\infty} g(x_i) P(\xi = x_i) 
        = \sum_{i = 1}^{\infty} g(x_i) P_{\xi}(\{x_i\})
      \end{align*}

      \begin{proof}
        Если $g(x) \geq 0$, 
        то $\sum\limits_{i = 1}^{n} g(x_i) I \{ \xi = x_i \} \uparrow g(\xi)$\\

        $\Rightarrow$ по теореме о монотонной сходимости:
        \begin{align*}
          E g(\xi) = \lim_{n \to \infty} \sum_{i = 1}^{n} g(x_i) P(\xi = x_i) = 
          \sum_{i = 1}^{\infty} g(x_i) P(\xi = x_i)
        \end{align*}

        В общем, раскладываем $g(x)$ на $g^+$ и $g^-$ и пользуемся линейностью мат. ожидания.
      \end{proof}

      \begin{corollary}
        если $P_{\xi}$ -- дискр. распр. на $\mathcal{X} = \{ x_i \}$, то 

        \begin{align*}
          \int\limits_{\setR} g(x) P_{\xi} (dx) = \sum_{i = 1}^{\infty} g(x_i) P_{\xi}(\{ x_i \}) 
          = \int\limits_{\setR} g(x) dF_{\xi} (x)
        \end{align*}

      \end{corollary}

      \begin{example}
        Пусть $\xi \sim Pois(\lambda)$. Найти $E\xi = ?$

        \begin{align*}
          \xi \sim Pois(\lambda) \Rightarrow P(\xi = k) = \frac{\lambda^k e^{-\lambda}}{k!},
          \quad\text{ для $\forall k \in \setZ_+$}
        \end{align*}
        Тогда
        \begin{align*}
          E \xi = \sum_{k = 0}^{\infty} k P(\xi = k) 
          = \sum_{k = 0}^{\infty} k \frac{\lambda^k e^{-\lambda}}{k!}
          = e^{-\lambda} \sum_{k = 1}^{\infty} k \frac{\lambda^k}{k!}
          = e^{-\lambda} \sum_{k = 1}^{\infty} \frac{\lambda^k}{(k - 1)!}
          = \lambda
        \end{align*}
        
      \end{example}

    \item
      Пусть $\xi$ -- абсолютно непрерывная с.в. с плотностью $f_{\xi} (x)$. 

      Тогда для $\forall g(x)$ -- борелевской функции:
      \begin{align*}
        E g(\xi) = \int\limits_{\setR} g(x) f_{\xi} (x) dx
      \end{align*}

      \begin{proof}
        Пусть $F_{\xi}$ -- ф.р. $\xi$. 
        Тогда по определению плотности,
        \begin{align*}
          P(\xi \leq x) = F_{\xi} (x) = \int\limits_{-\infty}^{x} f_{\xi} (y) dy
        \end{align*}

        С другой стороны,
        \begin{align*}
          &P(\xi \leq x) = P_{\xi} ((-\infty, x]) = \int\limits_{-\infty}^{x} P_{\xi} (dy)\\
          &\Rightarrow P_{\xi}(dy) = f_{\xi} (y) dy
        \end{align*}

        В итоге,
        \begin{align*}
          E g(\xi) = \int\limits_{\setR} g(x) P_{\xi} (dx) 
          = \int\limits_{\setR} g(x) f_{\xi} (x) dx
        \end{align*}

      \end{proof}

      \begin{example}
        Пусть $\xi \sim N(a, \sigma^2)$. Вычислить $E \xi$.

        Плотность $N(a, \sigma^2)$ равна:
        \begin{align*}
          f_\xi = \frac{1}{\sqrt{2 \pi \sigma^2}} e^{- \frac{(x - a)^2}{2\sigma^2}}
        \end{align*}

        Тогда
        \begin{align*}
          \Rightarrow E \xi = \int\limits_{R} x f_{\xi} (x) dx 
          = \int\limits_{\setR} x\, \frac{1}{\sqrt{2 \pi \sigma^2}}
            e^{- \frac{(x - a)^2}{2\sigma^2}} dx\\
          = \int\limits_{\setR} (x - a) \frac{1}{\sqrt{2 \pi \sigma^2}} 
            e^{- \frac{(x - a)^2}{2\sigma^2}} dx + a \int\limits_{\setR} f_{\xi} (x) dx = a
        \end{align*}
      \end{example}

  \end{enumerate}
\end{corollary}

\begin{remark}

  Если $\xi = (\xi_1, \ldots, \xi_n)$ -- случайный вектор 
  с плотностью $f_\xi (x_1, \ldots, x_n)$, то для 
  $\forall g\colon \setRn \to \setR$ -- борелевской функции:
  \begin{align*}
    E g(\xi_1, \ldots, \xi_n) 
    = \int\limits_{\setRn} g(x_1, \ldots, x_n) f_{\xi} (x_1, \ldots, x_n) dx_1 \ldots dx_n
  \end{align*}

  \begin{example}
    Если $(\xi, \eta)$ имеет плотность $f_{(\xi, \eta)} (x, y)$, то
    \begin{align*}
      E \xi \eta = \int\limits_{\setR^2} x y f_{(\xi, \eta)} (x, y) dx dy
    \end{align*}
  \end{example}

\end{remark}

\mysection{Прямое произведение вероятностных пространств}

\begin{definition}
  Пусть $(\Omega, \setF_1, P_1)$ и $(\Omega, \setF_2, P_2)$ -- два вероятностных пространства. 
  Тогда вероятностное пространство $(\Omega, \setF, P)$ наз. их \emph{прямым произведением}, 
  если 
  \begin{itemize}
    \item 
      $\Omega = \Omega_1 \times \Omega_2$

    \item 
      $\setF = \setF_1 \otimes \setF_2$, т.е.

      $\setF = \sigma(\condset{B_1 \times B_2}{B_1 \in \setF_1, B_2 \in \setF_2})$ 

    \item
      $P = P_1 \otimes  P_2$, т.е.

      $P$ -- это продолжение меры $P_1 \times P_2$, заданной на прямоугольниках $B_1 \times B_2$,
      $B_i \in \setF_i$ по правилу $P(B_1 \times B_2) = P_1(B_1) \cdot P_2(B_2)$
  \end{itemize}

  Такое продолжение $\exists$ и единственно по теореме Каратеодори.

\end{definition}

